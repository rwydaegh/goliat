\documentclass{IEEEtran}
\usepackage{graphicx}
\usepackage{booktabs}
\usepackage{float}
\usepackage{alphalph}

% Fix subsection counter to use extended alphabetic (aa, ab, ... after z) instead of \Alpha (limited to 26)
\makeatletter
\@addtoreset{subsection}{section}
\def\thesubsection{\alphalph{\value{subsection}}}
\def\thesubsectiondis{\thesection.\alphalph{\value{subsection}}}
\makeatother

\usepackage{hyperref}

\title{Far-Field Analysis Results}
\author{Robin Wydaeghe}
\date{\today}

\begin{document}

\maketitle

\tableofcontents
\newpage

\section{Bar Charts}

\subsection{SAR Bar Environmental}

The average normalized SAR - Environmental can be found in Figure~\ref{fig:0_1}. \\\\

\begin{figure}[H]
\centering
\includegraphics[width=\columnwidth]{../../../plots/far_field/duke/bar/average_sar_bar_environmental.pdf}
\caption{The bar chart shows average normalized SAR values (Head, Trunk, Brain, Skin, Eyes, Genitals) across frequencies for the \textit{Environmental} scenario for Duke. Error bars show boxplot whiskers (IQR-based range).}
\label{fig:0_1}
\end{figure}

The average normalized SAR - Environmental can be found in Figure~\ref{fig:0_2}. \\\\

\begin{figure}[H]
\centering
\includegraphics[width=\columnwidth]{../../../plots/far_field/eartha/bar/average_sar_bar_environmental.pdf}
\caption{The bar chart shows average normalized SAR values (Head, Trunk, Brain, Skin, Eyes, Genitals) across frequencies for the \textit{Environmental} scenario for Eartha. Error bars show boxplot whiskers (IQR-based range).}
\label{fig:0_2}
\end{figure}

The average normalized SAR - Environmental can be found in Figure~\ref{fig:0_3}. \\\\

\begin{figure}[H]
\centering
\includegraphics[width=\columnwidth]{../../../plots/far_field/ella/bar/average_sar_bar_environmental.pdf}
\caption{The bar chart shows average normalized SAR values (Head, Trunk, Brain, Skin, Eyes, Genitals) across frequencies for the \textit{Environmental} scenario for Ella. Error bars show boxplot whiskers (IQR-based range).}
\label{fig:0_3}
\end{figure}

The average normalized SAR - Environmental can be found in Figure~\ref{fig:0_4}. \\\\

\begin{figure}[H]
\centering
\includegraphics[width=\columnwidth]{../../../plots/far_field/thelonious/bar/average_sar_bar_environmental.pdf}
\caption{The bar chart shows average normalized SAR values (Head, Trunk, Brain, Skin, Eyes, Genitals) across frequencies for the \textit{Environmental} scenario for Thelonious. Error bars show boxplot whiskers (IQR-based range).}
\label{fig:0_4}
\end{figure}

\subsection{Whole Body SAR Bar}

The average whole body SAR (Duke) can be found in Figure~\ref{fig:0_5}. \\\\

\begin{figure}[H]
\centering
\includegraphics[width=\columnwidth]{../../../plots/far_field/duke/bar/average_whole_body_sar_bar.pdf}
\caption{The bar chart shows average normalized whole-body SAR values across frequencies for Duke.}
\label{fig:0_5}
\end{figure}

The average whole body SAR (Eartha) can be found in Figure~\ref{fig:0_6}. \\\\

\begin{figure}[H]
\centering
\includegraphics[width=\columnwidth]{../../../plots/far_field/eartha/bar/average_whole_body_sar_bar.pdf}
\caption{The bar chart shows average normalized whole-body SAR values across frequencies for Eartha.}
\label{fig:0_6}
\end{figure}

The average whole body SAR (Ella) can be found in Figure~\ref{fig:0_7}. \\\\

\begin{figure}[H]
\centering
\includegraphics[width=\columnwidth]{../../../plots/far_field/ella/bar/average_whole_body_sar_bar.pdf}
\caption{The bar chart shows average normalized whole-body SAR values across frequencies for Ella.}
\label{fig:0_7}
\end{figure}

The average whole body SAR (Thelonious) can be found in Figure~\ref{fig:0_8}. \\\\

\begin{figure}[H]
\centering
\includegraphics[width=\columnwidth]{../../../plots/far_field/thelonious/bar/average_whole_body_sar_bar.pdf}
\caption{The bar chart shows average normalized whole-body SAR values across frequencies for Thelonious.}
\label{fig:0_8}
\end{figure}

\subsection{psSAR Bar Environmental}

The average normalized psSAR10g - Environmental can be found in Figure~\ref{fig:0_9}. \\\\

\begin{figure}[H]
\centering
\includegraphics[width=\columnwidth]{../../../plots/far_field/duke/bar/average_pssar_bar_environmental.pdf}
\caption{The bar chart shows average normalized psSAR10g values (Eyes, Skin, Brain, Genitals, Whole Body) across frequencies for the \textit{Environmental} scenario for Duke. Error bars show boxplot whiskers (IQR-based range).}
\label{fig:0_9}
\end{figure}

The average normalized psSAR10g - Environmental can be found in Figure~\ref{fig:0_10}. \\\\

\begin{figure}[H]
\centering
\includegraphics[width=\columnwidth]{../../../plots/far_field/eartha/bar/average_pssar_bar_environmental.pdf}
\caption{The bar chart shows average normalized psSAR10g values (Eyes, Skin, Brain, Genitals, Whole Body) across frequencies for the \textit{Environmental} scenario for Eartha. Error bars show boxplot whiskers (IQR-based range).}
\label{fig:0_10}
\end{figure}

The average normalized psSAR10g - Environmental can be found in Figure~\ref{fig:0_11}. \\\\

\begin{figure}[H]
\centering
\includegraphics[width=\columnwidth]{../../../plots/far_field/ella/bar/average_pssar_bar_environmental.pdf}
\caption{The bar chart shows average normalized psSAR10g values (Eyes, Skin, Brain, Genitals, Whole Body) across frequencies for the \textit{Environmental} scenario for Ella. Error bars show boxplot whiskers (IQR-based range).}
\label{fig:0_11}
\end{figure}

The average normalized psSAR10g - Environmental can be found in Figure~\ref{fig:0_12}. \\\\

\begin{figure}[H]
\centering
\includegraphics[width=\columnwidth]{../../../plots/far_field/thelonious/bar/average_pssar_bar_environmental.pdf}
\caption{The bar chart shows average normalized psSAR10g values (Eyes, Skin, Brain, Genitals, Whole Body) across frequencies for the \textit{Environmental} scenario for Thelonious. Error bars show boxplot whiskers (IQR-based range).}
\label{fig:0_12}
\end{figure}

\newpage

\section{Boxplots}

\subsection{Peak SAR All}

The distribution of normalized Peak SAR 10g (Duke) can be found in Figure~\ref{fig:1_1}. \\\\

\begin{figure}[H]
\centering
\includegraphics[width=\columnwidth]{../../../plots/far_field/duke/boxplot/boxplot_peak_sar_all.pdf}
\caption{The boxplot shows the distribution of normalized Peak SAR (10g) values across different frequencies for Duke. Each box spans from the first quartile (Q1) to the third quartile (Q3), with the median line shown inside the box. The whiskers extend to show the range of the data, and points beyond the whiskers are outliers.}
\label{fig:1_1}
\end{figure}

The distribution of normalized Peak SAR 10g (Eartha) can be found in Figure~\ref{fig:1_2}. \\\\

\begin{figure}[H]
\centering
\includegraphics[width=\columnwidth]{../../../plots/far_field/eartha/boxplot/boxplot_peak_sar_all.pdf}
\caption{The boxplot shows the distribution of normalized Peak SAR (10g) values across different frequencies for Eartha. Each box spans from the first quartile (Q1) to the third quartile (Q3), with the median line shown inside the box. The whiskers extend to show the range of the data, and points beyond the whiskers are outliers.}
\label{fig:1_2}
\end{figure}

The distribution of normalized Peak SAR 10g (Ella) can be found in Figure~\ref{fig:1_3}. \\\\

\begin{figure}[H]
\centering
\includegraphics[width=\columnwidth]{../../../plots/far_field/ella/boxplot/boxplot_peak_sar_all.pdf}
\caption{The boxplot shows the distribution of normalized Peak SAR (10g) values across different frequencies for Ella. Each box spans from the first quartile (Q1) to the third quartile (Q3), with the median line shown inside the box. The whiskers extend to show the range of the data, and points beyond the whiskers are outliers.}
\label{fig:1_3}
\end{figure}

The distribution of normalized Peak SAR 10g (Thelonious) can be found in Figure~\ref{fig:1_4}. \\\\

\begin{figure}[H]
\centering
\includegraphics[width=\columnwidth]{../../../plots/far_field/thelonious/boxplot/boxplot_peak_sar_all.pdf}
\caption{The boxplot shows the distribution of normalized Peak SAR (10g) values across different frequencies for Thelonious. Each box spans from the first quartile (Q1) to the third quartile (Q3), with the median line shown inside the box. The whiskers extend to show the range of the data, and points beyond the whiskers are outliers.}
\label{fig:1_4}
\end{figure}

\subsection{Peak SAR Distribution}

The distribution of normalized Peak SAR 10g (Duke) can be found in Figure~\ref{fig:1_5}. \\\\

\begin{figure}[H]
\centering
\includegraphics[width=\columnwidth]{../../../plots/far_field/duke/boxplot/boxplot_peak_sar_distribution.pdf}
\caption{The boxplot shows the distribution of normalized Peak SAR (10g) values across different frequencies for Duke. Each box spans from the first quartile (Q1) to the third quartile (Q3), with the median line shown inside the box. The whiskers extend to show the range of the data, and points beyond the whiskers are outliers.}
\label{fig:1_5}
\end{figure}

The distribution of normalized Peak SAR 10g (Eartha) can be found in Figure~\ref{fig:1_6}. \\\\

\begin{figure}[H]
\centering
\includegraphics[width=\columnwidth]{../../../plots/far_field/eartha/boxplot/boxplot_peak_sar_distribution.pdf}
\caption{The boxplot shows the distribution of normalized Peak SAR (10g) values across different frequencies for Eartha. Each box spans from the first quartile (Q1) to the third quartile (Q3), with the median line shown inside the box. The whiskers extend to show the range of the data, and points beyond the whiskers are outliers.}
\label{fig:1_6}
\end{figure}

The distribution of normalized Peak SAR 10g (Ella) can be found in Figure~\ref{fig:1_7}. \\\\

\begin{figure}[H]
\centering
\includegraphics[width=\columnwidth]{../../../plots/far_field/ella/boxplot/boxplot_peak_sar_distribution.pdf}
\caption{The boxplot shows the distribution of normalized Peak SAR (10g) values across different frequencies for Ella. Each box spans from the first quartile (Q1) to the third quartile (Q3), with the median line shown inside the box. The whiskers extend to show the range of the data, and points beyond the whiskers are outliers.}
\label{fig:1_7}
\end{figure}

The distribution of normalized Peak SAR 10g (Thelonious) can be found in Figure~\ref{fig:1_8}. \\\\

\begin{figure}[H]
\centering
\includegraphics[width=\columnwidth]{../../../plots/far_field/thelonious/boxplot/boxplot_peak_sar_distribution.pdf}
\caption{The boxplot shows the distribution of normalized Peak SAR (10g) values across different frequencies for Thelonious. Each box spans from the first quartile (Q1) to the third quartile (Q3), with the median line shown inside the box. The whiskers extend to show the range of the data, and points beyond the whiskers are outliers.}
\label{fig:1_8}
\end{figure}

\subsection{SAR Brain All}

The distribution of normalized Brain SAR (Duke) can be found in Figure~\ref{fig:1_9}. \\\\

\begin{figure}[H]
\centering
\includegraphics[width=\columnwidth]{../../../plots/far_field/duke/boxplot/boxplot_SAR_brain_all.pdf}
\caption{The boxplot shows the distribution of normalized Brain SAR values across different frequencies for Duke. Each box spans from the first quartile (Q1) to the third quartile (Q3), with the median line shown inside the box. The whiskers extend to show the range of the data, and points beyond the whiskers are outliers.}
\label{fig:1_9}
\end{figure}

The distribution of normalized Brain SAR (Eartha) can be found in Figure~\ref{fig:1_10}. \\\\

\begin{figure}[H]
\centering
\includegraphics[width=\columnwidth]{../../../plots/far_field/eartha/boxplot/boxplot_SAR_brain_all.pdf}
\caption{The boxplot shows the distribution of normalized Brain SAR values across different frequencies for Eartha. Each box spans from the first quartile (Q1) to the third quartile (Q3), with the median line shown inside the box. The whiskers extend to show the range of the data, and points beyond the whiskers are outliers.}
\label{fig:1_10}
\end{figure}

The distribution of normalized Brain SAR (Ella) can be found in Figure~\ref{fig:1_11}. \\\\

\begin{figure}[H]
\centering
\includegraphics[width=\columnwidth]{../../../plots/far_field/ella/boxplot/boxplot_SAR_brain_all.pdf}
\caption{The boxplot shows the distribution of normalized Brain SAR values across different frequencies for Ella. Each box spans from the first quartile (Q1) to the third quartile (Q3), with the median line shown inside the box. The whiskers extend to show the range of the data, and points beyond the whiskers are outliers.}
\label{fig:1_11}
\end{figure}

The distribution of normalized Brain SAR (Thelonious) can be found in Figure~\ref{fig:1_12}. \\\\

\begin{figure}[H]
\centering
\includegraphics[width=\columnwidth]{../../../plots/far_field/thelonious/boxplot/boxplot_SAR_brain_all.pdf}
\caption{The boxplot shows the distribution of normalized Brain SAR values across different frequencies for Thelonious. Each box spans from the first quartile (Q1) to the third quartile (Q3), with the median line shown inside the box. The whiskers extend to show the range of the data, and points beyond the whiskers are outliers.}
\label{fig:1_12}
\end{figure}

\subsection{SAR Brain Distribution}

The distribution of normalized Brain SAR (Duke) can be found in Figure~\ref{fig:1_13}. \\\\

\begin{figure}[H]
\centering
\includegraphics[width=\columnwidth]{../../../plots/far_field/duke/boxplot/boxplot_SAR_brain_distribution.pdf}
\caption{The boxplot shows the distribution of normalized Brain SAR values across different frequencies for Duke. Each box spans from the first quartile (Q1) to the third quartile (Q3), with the median line shown inside the box. The whiskers extend to show the range of the data, and points beyond the whiskers are outliers.}
\label{fig:1_13}
\end{figure}

The distribution of normalized Brain SAR (Eartha) can be found in Figure~\ref{fig:1_14}. \\\\

\begin{figure}[H]
\centering
\includegraphics[width=\columnwidth]{../../../plots/far_field/eartha/boxplot/boxplot_SAR_brain_distribution.pdf}
\caption{The boxplot shows the distribution of normalized Brain SAR values across different frequencies for Eartha. Each box spans from the first quartile (Q1) to the third quartile (Q3), with the median line shown inside the box. The whiskers extend to show the range of the data, and points beyond the whiskers are outliers.}
\label{fig:1_14}
\end{figure}

The distribution of normalized Brain SAR (Ella) can be found in Figure~\ref{fig:1_15}. \\\\

\begin{figure}[H]
\centering
\includegraphics[width=\columnwidth]{../../../plots/far_field/ella/boxplot/boxplot_SAR_brain_distribution.pdf}
\caption{The boxplot shows the distribution of normalized Brain SAR values across different frequencies for Ella. Each box spans from the first quartile (Q1) to the third quartile (Q3), with the median line shown inside the box. The whiskers extend to show the range of the data, and points beyond the whiskers are outliers.}
\label{fig:1_15}
\end{figure}

The distribution of normalized Brain SAR (Thelonious) can be found in Figure~\ref{fig:1_16}. \\\\

\begin{figure}[H]
\centering
\includegraphics[width=\columnwidth]{../../../plots/far_field/thelonious/boxplot/boxplot_SAR_brain_distribution.pdf}
\caption{The boxplot shows the distribution of normalized Brain SAR values across different frequencies for Thelonious. Each box spans from the first quartile (Q1) to the third quartile (Q3), with the median line shown inside the box. The whiskers extend to show the range of the data, and points beyond the whiskers are outliers.}
\label{fig:1_16}
\end{figure}

\subsection{SAR Brain Environmental}

The distribution of normalized Brain SAR - Environmental can be found in Figure~\ref{fig:1_17}. \\\\

\begin{figure}[H]
\centering
\includegraphics[width=\columnwidth]{../../../plots/far_field/duke/boxplot/boxplot_SAR_brain_environmental.pdf}
\caption{The boxplot shows the distribution of normalized Brain SAR values across different frequencies for the \textit{Environmental} scenario for Duke. Each box spans from the first quartile (Q1) to the third quartile (Q3), with the median line shown inside the box. The whiskers extend to show the range of the data, and points beyond the whiskers are outliers.}
\label{fig:1_17}
\end{figure}

The distribution of normalized Brain SAR - Environmental can be found in Figure~\ref{fig:1_18}. \\\\

\begin{figure}[H]
\centering
\includegraphics[width=\columnwidth]{../../../plots/far_field/eartha/boxplot/boxplot_SAR_brain_environmental.pdf}
\caption{The boxplot shows the distribution of normalized Brain SAR values across different frequencies for the \textit{Environmental} scenario for Eartha. Each box spans from the first quartile (Q1) to the third quartile (Q3), with the median line shown inside the box. The whiskers extend to show the range of the data, and points beyond the whiskers are outliers.}
\label{fig:1_18}
\end{figure}

The distribution of normalized Brain SAR - Environmental can be found in Figure~\ref{fig:1_19}. \\\\

\begin{figure}[H]
\centering
\includegraphics[width=\columnwidth]{../../../plots/far_field/ella/boxplot/boxplot_SAR_brain_environmental.pdf}
\caption{The boxplot shows the distribution of normalized Brain SAR values across different frequencies for the \textit{Environmental} scenario for Ella. Each box spans from the first quartile (Q1) to the third quartile (Q3), with the median line shown inside the box. The whiskers extend to show the range of the data, and points beyond the whiskers are outliers.}
\label{fig:1_19}
\end{figure}

The distribution of normalized Brain SAR - Environmental can be found in Figure~\ref{fig:1_20}. \\\\

\begin{figure}[H]
\centering
\includegraphics[width=\columnwidth]{../../../plots/far_field/thelonious/boxplot/boxplot_SAR_brain_environmental.pdf}
\caption{The boxplot shows the distribution of normalized Brain SAR values across different frequencies for the \textit{Environmental} scenario for Thelonious. Each box spans from the first quartile (Q1) to the third quartile (Q3), with the median line shown inside the box. The whiskers extend to show the range of the data, and points beyond the whiskers are outliers.}
\label{fig:1_20}
\end{figure}

\subsection{SAR Eyes All}

The distribution of normalized Eyes SAR (Duke) can be found in Figure~\ref{fig:1_21}. \\\\

\begin{figure}[H]
\centering
\includegraphics[width=\columnwidth]{../../../plots/far_field/duke/boxplot/boxplot_SAR_eyes_all.pdf}
\caption{The boxplot shows the distribution of normalized Eyes SAR values across different frequencies for Duke. Each box spans from the first quartile (Q1) to the third quartile (Q3), with the median line shown inside the box. The whiskers extend to show the range of the data, and points beyond the whiskers are outliers.}
\label{fig:1_21}
\end{figure}

The distribution of normalized Eyes SAR (Eartha) can be found in Figure~\ref{fig:1_22}. \\\\

\begin{figure}[H]
\centering
\includegraphics[width=\columnwidth]{../../../plots/far_field/eartha/boxplot/boxplot_SAR_eyes_all.pdf}
\caption{The boxplot shows the distribution of normalized Eyes SAR values across different frequencies for Eartha. Each box spans from the first quartile (Q1) to the third quartile (Q3), with the median line shown inside the box. The whiskers extend to show the range of the data, and points beyond the whiskers are outliers.}
\label{fig:1_22}
\end{figure}

The distribution of normalized Eyes SAR (Ella) can be found in Figure~\ref{fig:1_23}. \\\\

\begin{figure}[H]
\centering
\includegraphics[width=\columnwidth]{../../../plots/far_field/ella/boxplot/boxplot_SAR_eyes_all.pdf}
\caption{The boxplot shows the distribution of normalized Eyes SAR values across different frequencies for Ella. Each box spans from the first quartile (Q1) to the third quartile (Q3), with the median line shown inside the box. The whiskers extend to show the range of the data, and points beyond the whiskers are outliers.}
\label{fig:1_23}
\end{figure}

The distribution of normalized Eyes SAR (Thelonious) can be found in Figure~\ref{fig:1_24}. \\\\

\begin{figure}[H]
\centering
\includegraphics[width=\columnwidth]{../../../plots/far_field/thelonious/boxplot/boxplot_SAR_eyes_all.pdf}
\caption{The boxplot shows the distribution of normalized Eyes SAR values across different frequencies for Thelonious. Each box spans from the first quartile (Q1) to the third quartile (Q3), with the median line shown inside the box. The whiskers extend to show the range of the data, and points beyond the whiskers are outliers.}
\label{fig:1_24}
\end{figure}

\subsection{SAR Eyes Distribution}

The distribution of normalized Eyes SAR (Duke) can be found in Figure~\ref{fig:1_25}. \\\\

\begin{figure}[H]
\centering
\includegraphics[width=\columnwidth]{../../../plots/far_field/duke/boxplot/boxplot_SAR_eyes_distribution.pdf}
\caption{The boxplot shows the distribution of normalized Eyes SAR values across different frequencies for Duke. Each box spans from the first quartile (Q1) to the third quartile (Q3), with the median line shown inside the box. The whiskers extend to show the range of the data, and points beyond the whiskers are outliers.}
\label{fig:1_25}
\end{figure}

The distribution of normalized Eyes SAR (Eartha) can be found in Figure~\ref{fig:1_26}. \\\\

\begin{figure}[H]
\centering
\includegraphics[width=\columnwidth]{../../../plots/far_field/eartha/boxplot/boxplot_SAR_eyes_distribution.pdf}
\caption{The boxplot shows the distribution of normalized Eyes SAR values across different frequencies for Eartha. Each box spans from the first quartile (Q1) to the third quartile (Q3), with the median line shown inside the box. The whiskers extend to show the range of the data, and points beyond the whiskers are outliers.}
\label{fig:1_26}
\end{figure}

The distribution of normalized Eyes SAR (Ella) can be found in Figure~\ref{fig:1_27}. \\\\

\begin{figure}[H]
\centering
\includegraphics[width=\columnwidth]{../../../plots/far_field/ella/boxplot/boxplot_SAR_eyes_distribution.pdf}
\caption{The boxplot shows the distribution of normalized Eyes SAR values across different frequencies for Ella. Each box spans from the first quartile (Q1) to the third quartile (Q3), with the median line shown inside the box. The whiskers extend to show the range of the data, and points beyond the whiskers are outliers.}
\label{fig:1_27}
\end{figure}

The distribution of normalized Eyes SAR (Thelonious) can be found in Figure~\ref{fig:1_28}. \\\\

\begin{figure}[H]
\centering
\includegraphics[width=\columnwidth]{../../../plots/far_field/thelonious/boxplot/boxplot_SAR_eyes_distribution.pdf}
\caption{The boxplot shows the distribution of normalized Eyes SAR values across different frequencies for Thelonious. Each box spans from the first quartile (Q1) to the third quartile (Q3), with the median line shown inside the box. The whiskers extend to show the range of the data, and points beyond the whiskers are outliers.}
\label{fig:1_28}
\end{figure}

\subsection{SAR Eyes Environmental}

The distribution of normalized Eyes SAR - Environmental can be found in Figure~\ref{fig:1_29}. \\\\

\begin{figure}[H]
\centering
\includegraphics[width=\columnwidth]{../../../plots/far_field/duke/boxplot/boxplot_SAR_eyes_environmental.pdf}
\caption{The boxplot shows the distribution of normalized Eyes SAR values across different frequencies for the \textit{Environmental} scenario for Duke. Each box spans from the first quartile (Q1) to the third quartile (Q3), with the median line shown inside the box. The whiskers extend to show the range of the data, and points beyond the whiskers are outliers.}
\label{fig:1_29}
\end{figure}

The distribution of normalized Eyes SAR - Environmental can be found in Figure~\ref{fig:1_30}. \\\\

\begin{figure}[H]
\centering
\includegraphics[width=\columnwidth]{../../../plots/far_field/eartha/boxplot/boxplot_SAR_eyes_environmental.pdf}
\caption{The boxplot shows the distribution of normalized Eyes SAR values across different frequencies for the \textit{Environmental} scenario for Eartha. Each box spans from the first quartile (Q1) to the third quartile (Q3), with the median line shown inside the box. The whiskers extend to show the range of the data, and points beyond the whiskers are outliers.}
\label{fig:1_30}
\end{figure}

The distribution of normalized Eyes SAR - Environmental can be found in Figure~\ref{fig:1_31}. \\\\

\begin{figure}[H]
\centering
\includegraphics[width=\columnwidth]{../../../plots/far_field/ella/boxplot/boxplot_SAR_eyes_environmental.pdf}
\caption{The boxplot shows the distribution of normalized Eyes SAR values across different frequencies for the \textit{Environmental} scenario for Ella. Each box spans from the first quartile (Q1) to the third quartile (Q3), with the median line shown inside the box. The whiskers extend to show the range of the data, and points beyond the whiskers are outliers.}
\label{fig:1_31}
\end{figure}

The distribution of normalized Eyes SAR - Environmental can be found in Figure~\ref{fig:1_32}. \\\\

\begin{figure}[H]
\centering
\includegraphics[width=\columnwidth]{../../../plots/far_field/thelonious/boxplot/boxplot_SAR_eyes_environmental.pdf}
\caption{The boxplot shows the distribution of normalized Eyes SAR values across different frequencies for the \textit{Environmental} scenario for Thelonious. Each box spans from the first quartile (Q1) to the third quartile (Q3), with the median line shown inside the box. The whiskers extend to show the range of the data, and points beyond the whiskers are outliers.}
\label{fig:1_32}
\end{figure}

\subsection{SAR Genitals All}

The distribution of normalized Genitals SAR (Duke) can be found in Figure~\ref{fig:1_33}. \\\\

\begin{figure}[H]
\centering
\includegraphics[width=\columnwidth]{../../../plots/far_field/duke/boxplot/boxplot_SAR_genitals_all.pdf}
\caption{The boxplot shows the distribution of normalized Genitals SAR values across different frequencies for Duke. Each box spans from the first quartile (Q1) to the third quartile (Q3), with the median line shown inside the box. The whiskers extend to show the range of the data, and points beyond the whiskers are outliers.}
\label{fig:1_33}
\end{figure}

The distribution of normalized Genitals SAR (Eartha) can be found in Figure~\ref{fig:1_34}. \\\\

\begin{figure}[H]
\centering
\includegraphics[width=\columnwidth]{../../../plots/far_field/eartha/boxplot/boxplot_SAR_genitals_all.pdf}
\caption{The boxplot shows the distribution of normalized Genitals SAR values across different frequencies for Eartha. Each box spans from the first quartile (Q1) to the third quartile (Q3), with the median line shown inside the box. The whiskers extend to show the range of the data, and points beyond the whiskers are outliers.}
\label{fig:1_34}
\end{figure}

The distribution of normalized Genitals SAR (Ella) can be found in Figure~\ref{fig:1_35}. \\\\

\begin{figure}[H]
\centering
\includegraphics[width=\columnwidth]{../../../plots/far_field/ella/boxplot/boxplot_SAR_genitals_all.pdf}
\caption{The boxplot shows the distribution of normalized Genitals SAR values across different frequencies for Ella. Each box spans from the first quartile (Q1) to the third quartile (Q3), with the median line shown inside the box. The whiskers extend to show the range of the data, and points beyond the whiskers are outliers.}
\label{fig:1_35}
\end{figure}

The distribution of normalized Genitals SAR (Thelonious) can be found in Figure~\ref{fig:1_36}. \\\\

\begin{figure}[H]
\centering
\includegraphics[width=\columnwidth]{../../../plots/far_field/thelonious/boxplot/boxplot_SAR_genitals_all.pdf}
\caption{The boxplot shows the distribution of normalized Genitals SAR values across different frequencies for Thelonious. Each box spans from the first quartile (Q1) to the third quartile (Q3), with the median line shown inside the box. The whiskers extend to show the range of the data, and points beyond the whiskers are outliers.}
\label{fig:1_36}
\end{figure}

\subsection{SAR Genitals Distribution}

The distribution of normalized Genitals SAR (Duke) can be found in Figure~\ref{fig:1_37}. \\\\

\begin{figure}[H]
\centering
\includegraphics[width=\columnwidth]{../../../plots/far_field/duke/boxplot/boxplot_SAR_genitals_distribution.pdf}
\caption{The boxplot shows the distribution of normalized Genitals SAR values across different frequencies for Duke. Each box spans from the first quartile (Q1) to the third quartile (Q3), with the median line shown inside the box. The whiskers extend to show the range of the data, and points beyond the whiskers are outliers.}
\label{fig:1_37}
\end{figure}

The distribution of normalized Genitals SAR (Eartha) can be found in Figure~\ref{fig:1_38}. \\\\

\begin{figure}[H]
\centering
\includegraphics[width=\columnwidth]{../../../plots/far_field/eartha/boxplot/boxplot_SAR_genitals_distribution.pdf}
\caption{The boxplot shows the distribution of normalized Genitals SAR values across different frequencies for Eartha. Each box spans from the first quartile (Q1) to the third quartile (Q3), with the median line shown inside the box. The whiskers extend to show the range of the data, and points beyond the whiskers are outliers.}
\label{fig:1_38}
\end{figure}

The distribution of normalized Genitals SAR (Ella) can be found in Figure~\ref{fig:1_39}. \\\\

\begin{figure}[H]
\centering
\includegraphics[width=\columnwidth]{../../../plots/far_field/ella/boxplot/boxplot_SAR_genitals_distribution.pdf}
\caption{The boxplot shows the distribution of normalized Genitals SAR values across different frequencies for Ella. Each box spans from the first quartile (Q1) to the third quartile (Q3), with the median line shown inside the box. The whiskers extend to show the range of the data, and points beyond the whiskers are outliers.}
\label{fig:1_39}
\end{figure}

The distribution of normalized Genitals SAR (Thelonious) can be found in Figure~\ref{fig:1_40}. \\\\

\begin{figure}[H]
\centering
\includegraphics[width=\columnwidth]{../../../plots/far_field/thelonious/boxplot/boxplot_SAR_genitals_distribution.pdf}
\caption{The boxplot shows the distribution of normalized Genitals SAR values across different frequencies for Thelonious. Each box spans from the first quartile (Q1) to the third quartile (Q3), with the median line shown inside the box. The whiskers extend to show the range of the data, and points beyond the whiskers are outliers.}
\label{fig:1_40}
\end{figure}

\subsection{SAR Genitals Environmental}

The distribution of normalized Genitals SAR - Environmental can be found in Figure~\ref{fig:1_41}. \\\\

\begin{figure}[H]
\centering
\includegraphics[width=\columnwidth]{../../../plots/far_field/duke/boxplot/boxplot_SAR_genitals_environmental.pdf}
\caption{The boxplot shows the distribution of normalized Genitals SAR values across different frequencies for the \textit{Environmental} scenario for Duke. Each box spans from the first quartile (Q1) to the third quartile (Q3), with the median line shown inside the box. The whiskers extend to show the range of the data, and points beyond the whiskers are outliers.}
\label{fig:1_41}
\end{figure}

The distribution of normalized Genitals SAR - Environmental can be found in Figure~\ref{fig:1_42}. \\\\

\begin{figure}[H]
\centering
\includegraphics[width=\columnwidth]{../../../plots/far_field/eartha/boxplot/boxplot_SAR_genitals_environmental.pdf}
\caption{The boxplot shows the distribution of normalized Genitals SAR values across different frequencies for the \textit{Environmental} scenario for Eartha. Each box spans from the first quartile (Q1) to the third quartile (Q3), with the median line shown inside the box. The whiskers extend to show the range of the data, and points beyond the whiskers are outliers.}
\label{fig:1_42}
\end{figure}

The distribution of normalized Genitals SAR - Environmental can be found in Figure~\ref{fig:1_43}. \\\\

\begin{figure}[H]
\centering
\includegraphics[width=\columnwidth]{../../../plots/far_field/ella/boxplot/boxplot_SAR_genitals_environmental.pdf}
\caption{The boxplot shows the distribution of normalized Genitals SAR values across different frequencies for the \textit{Environmental} scenario for Ella. Each box spans from the first quartile (Q1) to the third quartile (Q3), with the median line shown inside the box. The whiskers extend to show the range of the data, and points beyond the whiskers are outliers.}
\label{fig:1_43}
\end{figure}

The distribution of normalized Genitals SAR - Environmental can be found in Figure~\ref{fig:1_44}. \\\\

\begin{figure}[H]
\centering
\includegraphics[width=\columnwidth]{../../../plots/far_field/thelonious/boxplot/boxplot_SAR_genitals_environmental.pdf}
\caption{The boxplot shows the distribution of normalized Genitals SAR values across different frequencies for the \textit{Environmental} scenario for Thelonious. Each box spans from the first quartile (Q1) to the third quartile (Q3), with the median line shown inside the box. The whiskers extend to show the range of the data, and points beyond the whiskers are outliers.}
\label{fig:1_44}
\end{figure}

\subsection{SAR Skin All}

The distribution of normalized Skin SAR (Duke) can be found in Figure~\ref{fig:1_45}. \\\\

\begin{figure}[H]
\centering
\includegraphics[width=\columnwidth]{../../../plots/far_field/duke/boxplot/boxplot_SAR_skin_all.pdf}
\caption{The boxplot shows the distribution of normalized Skin SAR values across different frequencies for Duke. Each box spans from the first quartile (Q1) to the third quartile (Q3), with the median line shown inside the box. The whiskers extend to show the range of the data, and points beyond the whiskers are outliers.}
\label{fig:1_45}
\end{figure}

The distribution of normalized Skin SAR (Eartha) can be found in Figure~\ref{fig:1_46}. \\\\

\begin{figure}[H]
\centering
\includegraphics[width=\columnwidth]{../../../plots/far_field/eartha/boxplot/boxplot_SAR_skin_all.pdf}
\caption{The boxplot shows the distribution of normalized Skin SAR values across different frequencies for Eartha. Each box spans from the first quartile (Q1) to the third quartile (Q3), with the median line shown inside the box. The whiskers extend to show the range of the data, and points beyond the whiskers are outliers.}
\label{fig:1_46}
\end{figure}

The distribution of normalized Skin SAR (Ella) can be found in Figure~\ref{fig:1_47}. \\\\

\begin{figure}[H]
\centering
\includegraphics[width=\columnwidth]{../../../plots/far_field/ella/boxplot/boxplot_SAR_skin_all.pdf}
\caption{The boxplot shows the distribution of normalized Skin SAR values across different frequencies for Ella. Each box spans from the first quartile (Q1) to the third quartile (Q3), with the median line shown inside the box. The whiskers extend to show the range of the data, and points beyond the whiskers are outliers.}
\label{fig:1_47}
\end{figure}

The distribution of normalized Skin SAR (Thelonious) can be found in Figure~\ref{fig:1_48}. \\\\

\begin{figure}[H]
\centering
\includegraphics[width=\columnwidth]{../../../plots/far_field/thelonious/boxplot/boxplot_SAR_skin_all.pdf}
\caption{The boxplot shows the distribution of normalized Skin SAR values across different frequencies for Thelonious. Each box spans from the first quartile (Q1) to the third quartile (Q3), with the median line shown inside the box. The whiskers extend to show the range of the data, and points beyond the whiskers are outliers.}
\label{fig:1_48}
\end{figure}

\subsection{SAR Skin Distribution}

The distribution of normalized Skin SAR (Duke) can be found in Figure~\ref{fig:1_49}. \\\\

\begin{figure}[H]
\centering
\includegraphics[width=\columnwidth]{../../../plots/far_field/duke/boxplot/boxplot_SAR_skin_distribution.pdf}
\caption{The boxplot shows the distribution of normalized Skin SAR values across different frequencies for Duke. Each box spans from the first quartile (Q1) to the third quartile (Q3), with the median line shown inside the box. The whiskers extend to show the range of the data, and points beyond the whiskers are outliers.}
\label{fig:1_49}
\end{figure}

The distribution of normalized Skin SAR (Eartha) can be found in Figure~\ref{fig:1_50}. \\\\

\begin{figure}[H]
\centering
\includegraphics[width=\columnwidth]{../../../plots/far_field/eartha/boxplot/boxplot_SAR_skin_distribution.pdf}
\caption{The boxplot shows the distribution of normalized Skin SAR values across different frequencies for Eartha. Each box spans from the first quartile (Q1) to the third quartile (Q3), with the median line shown inside the box. The whiskers extend to show the range of the data, and points beyond the whiskers are outliers.}
\label{fig:1_50}
\end{figure}

The distribution of normalized Skin SAR (Ella) can be found in Figure~\ref{fig:1_51}. \\\\

\begin{figure}[H]
\centering
\includegraphics[width=\columnwidth]{../../../plots/far_field/ella/boxplot/boxplot_SAR_skin_distribution.pdf}
\caption{The boxplot shows the distribution of normalized Skin SAR values across different frequencies for Ella. Each box spans from the first quartile (Q1) to the third quartile (Q3), with the median line shown inside the box. The whiskers extend to show the range of the data, and points beyond the whiskers are outliers.}
\label{fig:1_51}
\end{figure}

The distribution of normalized Skin SAR (Thelonious) can be found in Figure~\ref{fig:1_52}. \\\\

\begin{figure}[H]
\centering
\includegraphics[width=\columnwidth]{../../../plots/far_field/thelonious/boxplot/boxplot_SAR_skin_distribution.pdf}
\caption{The boxplot shows the distribution of normalized Skin SAR values across different frequencies for Thelonious. Each box spans from the first quartile (Q1) to the third quartile (Q3), with the median line shown inside the box. The whiskers extend to show the range of the data, and points beyond the whiskers are outliers.}
\label{fig:1_52}
\end{figure}

\subsection{SAR Skin Environmental}

The distribution of normalized Skin SAR - Environmental can be found in Figure~\ref{fig:1_53}. \\\\

\begin{figure}[H]
\centering
\includegraphics[width=\columnwidth]{../../../plots/far_field/duke/boxplot/boxplot_SAR_skin_environmental.pdf}
\caption{The boxplot shows the distribution of normalized Skin SAR values across different frequencies for the \textit{Environmental} scenario for Duke. Each box spans from the first quartile (Q1) to the third quartile (Q3), with the median line shown inside the box. The whiskers extend to show the range of the data, and points beyond the whiskers are outliers.}
\label{fig:1_53}
\end{figure}

The distribution of normalized Skin SAR - Environmental can be found in Figure~\ref{fig:1_54}. \\\\

\begin{figure}[H]
\centering
\includegraphics[width=\columnwidth]{../../../plots/far_field/eartha/boxplot/boxplot_SAR_skin_environmental.pdf}
\caption{The boxplot shows the distribution of normalized Skin SAR values across different frequencies for the \textit{Environmental} scenario for Eartha. Each box spans from the first quartile (Q1) to the third quartile (Q3), with the median line shown inside the box. The whiskers extend to show the range of the data, and points beyond the whiskers are outliers.}
\label{fig:1_54}
\end{figure}

The distribution of normalized Skin SAR - Environmental can be found in Figure~\ref{fig:1_55}. \\\\

\begin{figure}[H]
\centering
\includegraphics[width=\columnwidth]{../../../plots/far_field/ella/boxplot/boxplot_SAR_skin_environmental.pdf}
\caption{The boxplot shows the distribution of normalized Skin SAR values across different frequencies for the \textit{Environmental} scenario for Ella. Each box spans from the first quartile (Q1) to the third quartile (Q3), with the median line shown inside the box. The whiskers extend to show the range of the data, and points beyond the whiskers are outliers.}
\label{fig:1_55}
\end{figure}

The distribution of normalized Skin SAR - Environmental can be found in Figure~\ref{fig:1_56}. \\\\

\begin{figure}[H]
\centering
\includegraphics[width=\columnwidth]{../../../plots/far_field/thelonious/boxplot/boxplot_SAR_skin_environmental.pdf}
\caption{The boxplot shows the distribution of normalized Skin SAR values across different frequencies for the \textit{Environmental} scenario for Thelonious. Each box spans from the first quartile (Q1) to the third quartile (Q3), with the median line shown inside the box. The whiskers extend to show the range of the data, and points beyond the whiskers are outliers.}
\label{fig:1_56}
\end{figure}

\subsection{SAR Whole Body All}

The distribution of normalized Whole Body SAR (Duke) can be found in Figure~\ref{fig:1_57}. \\\\

\begin{figure}[H]
\centering
\includegraphics[width=\columnwidth]{../../../plots/far_field/duke/boxplot/boxplot_SAR_whole_body_all.pdf}
\caption{The boxplot shows the distribution of normalized Whole-Body SAR values across different frequencies for Duke. Each box spans from the first quartile (Q1) to the third quartile (Q3), with the median line shown inside the box. The whiskers extend to show the range of the data, and points beyond the whiskers are outliers.}
\label{fig:1_57}
\end{figure}

The distribution of normalized Whole Body SAR (Eartha) can be found in Figure~\ref{fig:1_58}. \\\\

\begin{figure}[H]
\centering
\includegraphics[width=\columnwidth]{../../../plots/far_field/eartha/boxplot/boxplot_SAR_whole_body_all.pdf}
\caption{The boxplot shows the distribution of normalized Whole-Body SAR values across different frequencies for Eartha. Each box spans from the first quartile (Q1) to the third quartile (Q3), with the median line shown inside the box. The whiskers extend to show the range of the data, and points beyond the whiskers are outliers.}
\label{fig:1_58}
\end{figure}

The distribution of normalized Whole Body SAR (Ella) can be found in Figure~\ref{fig:1_59}. \\\\

\begin{figure}[H]
\centering
\includegraphics[width=\columnwidth]{../../../plots/far_field/ella/boxplot/boxplot_SAR_whole_body_all.pdf}
\caption{The boxplot shows the distribution of normalized Whole-Body SAR values across different frequencies for Ella. Each box spans from the first quartile (Q1) to the third quartile (Q3), with the median line shown inside the box. The whiskers extend to show the range of the data, and points beyond the whiskers are outliers.}
\label{fig:1_59}
\end{figure}

The distribution of normalized Whole Body SAR (Thelonious) can be found in Figure~\ref{fig:1_60}. \\\\

\begin{figure}[H]
\centering
\includegraphics[width=\columnwidth]{../../../plots/far_field/thelonious/boxplot/boxplot_SAR_whole_body_all.pdf}
\caption{The boxplot shows the distribution of normalized Whole-Body SAR values across different frequencies for Thelonious. Each box spans from the first quartile (Q1) to the third quartile (Q3), with the median line shown inside the box. The whiskers extend to show the range of the data, and points beyond the whiskers are outliers.}
\label{fig:1_60}
\end{figure}

\subsection{SAR Whole Body Distribution}

The distribution of normalized Whole Body SAR (Duke) can be found in Figure~\ref{fig:1_61}. \\\\

\begin{figure}[H]
\centering
\includegraphics[width=\columnwidth]{../../../plots/far_field/duke/boxplot/boxplot_SAR_whole_body_distribution.pdf}
\caption{The boxplot shows the distribution of normalized Whole-Body SAR values across different frequencies for Duke. Each box spans from the first quartile (Q1) to the third quartile (Q3), with the median line shown inside the box. The whiskers extend to show the range of the data, and points beyond the whiskers are outliers.}
\label{fig:1_61}
\end{figure}

The distribution of normalized Whole Body SAR (Eartha) can be found in Figure~\ref{fig:1_62}. \\\\

\begin{figure}[H]
\centering
\includegraphics[width=\columnwidth]{../../../plots/far_field/eartha/boxplot/boxplot_SAR_whole_body_distribution.pdf}
\caption{The boxplot shows the distribution of normalized Whole-Body SAR values across different frequencies for Eartha. Each box spans from the first quartile (Q1) to the third quartile (Q3), with the median line shown inside the box. The whiskers extend to show the range of the data, and points beyond the whiskers are outliers.}
\label{fig:1_62}
\end{figure}

The distribution of normalized Whole Body SAR (Ella) can be found in Figure~\ref{fig:1_63}. \\\\

\begin{figure}[H]
\centering
\includegraphics[width=\columnwidth]{../../../plots/far_field/ella/boxplot/boxplot_SAR_whole_body_distribution.pdf}
\caption{The boxplot shows the distribution of normalized Whole-Body SAR values across different frequencies for Ella. Each box spans from the first quartile (Q1) to the third quartile (Q3), with the median line shown inside the box. The whiskers extend to show the range of the data, and points beyond the whiskers are outliers.}
\label{fig:1_63}
\end{figure}

The distribution of normalized Whole Body SAR (Thelonious) can be found in Figure~\ref{fig:1_64}. \\\\

\begin{figure}[H]
\centering
\includegraphics[width=\columnwidth]{../../../plots/far_field/thelonious/boxplot/boxplot_SAR_whole_body_distribution.pdf}
\caption{The boxplot shows the distribution of normalized Whole-Body SAR values across different frequencies for Thelonious. Each box spans from the first quartile (Q1) to the third quartile (Q3), with the median line shown inside the box. The whiskers extend to show the range of the data, and points beyond the whiskers are outliers.}
\label{fig:1_64}
\end{figure}

\subsection{SAR Whole Body Environmental}

The distribution of normalized Whole Body SAR - Environmental can be found in Figure~\ref{fig:1_65}. \\\\

\begin{figure}[H]
\centering
\includegraphics[width=\columnwidth]{../../../plots/far_field/duke/boxplot/boxplot_SAR_whole_body_environmental.pdf}
\caption{The boxplot shows the distribution of normalized Whole-Body SAR values across different frequencies for the \textit{Environmental} scenario for Duke. Each box spans from the first quartile (Q1) to the third quartile (Q3), with the median line shown inside the box. The whiskers extend to show the range of the data, and points beyond the whiskers are outliers.}
\label{fig:1_65}
\end{figure}

The distribution of normalized Whole Body SAR - Environmental can be found in Figure~\ref{fig:1_66}. \\\\

\begin{figure}[H]
\centering
\includegraphics[width=\columnwidth]{../../../plots/far_field/eartha/boxplot/boxplot_SAR_whole_body_environmental.pdf}
\caption{The boxplot shows the distribution of normalized Whole-Body SAR values across different frequencies for the \textit{Environmental} scenario for Eartha. Each box spans from the first quartile (Q1) to the third quartile (Q3), with the median line shown inside the box. The whiskers extend to show the range of the data, and points beyond the whiskers are outliers.}
\label{fig:1_66}
\end{figure}

The distribution of normalized Whole Body SAR - Environmental can be found in Figure~\ref{fig:1_67}. \\\\

\begin{figure}[H]
\centering
\includegraphics[width=\columnwidth]{../../../plots/far_field/ella/boxplot/boxplot_SAR_whole_body_environmental.pdf}
\caption{The boxplot shows the distribution of normalized Whole-Body SAR values across different frequencies for the \textit{Environmental} scenario for Ella. Each box spans from the first quartile (Q1) to the third quartile (Q3), with the median line shown inside the box. The whiskers extend to show the range of the data, and points beyond the whiskers are outliers.}
\label{fig:1_67}
\end{figure}

The distribution of normalized Whole Body SAR - Environmental can be found in Figure~\ref{fig:1_68}. \\\\

\begin{figure}[H]
\centering
\includegraphics[width=\columnwidth]{../../../plots/far_field/thelonious/boxplot/boxplot_SAR_whole_body_environmental.pdf}
\caption{The boxplot shows the distribution of normalized Whole-Body SAR values across different frequencies for the \textit{Environmental} scenario for Thelonious. Each box spans from the first quartile (Q1) to the third quartile (Q3), with the median line shown inside the box. The whiskers extend to show the range of the data, and points beyond the whiskers are outliers.}
\label{fig:1_68}
\end{figure}

\subsection{psSAR10g Brain All}

The distribution of normalized psSAR10g Brain (Duke) can be found in Figure~\ref{fig:1_69}. \\\\

\begin{figure}[H]
\centering
\includegraphics[width=\columnwidth]{../../../plots/far_field/duke/boxplot/boxplot_psSAR10g_brain_all.pdf}
\caption{The boxplot shows the distribution of normalized psSAR10g Brain values across different frequencies for Duke. Each box spans from the first quartile (Q1) to the third quartile (Q3), with the median line shown inside the box. The whiskers extend to show the range of the data, and points beyond the whiskers are outliers.}
\label{fig:1_69}
\end{figure}

The distribution of normalized psSAR10g Brain (Eartha) can be found in Figure~\ref{fig:1_70}. \\\\

\begin{figure}[H]
\centering
\includegraphics[width=\columnwidth]{../../../plots/far_field/eartha/boxplot/boxplot_psSAR10g_brain_all.pdf}
\caption{The boxplot shows the distribution of normalized psSAR10g Brain values across different frequencies for Eartha. Each box spans from the first quartile (Q1) to the third quartile (Q3), with the median line shown inside the box. The whiskers extend to show the range of the data, and points beyond the whiskers are outliers.}
\label{fig:1_70}
\end{figure}

The distribution of normalized psSAR10g Brain (Ella) can be found in Figure~\ref{fig:1_71}. \\\\

\begin{figure}[H]
\centering
\includegraphics[width=\columnwidth]{../../../plots/far_field/ella/boxplot/boxplot_psSAR10g_brain_all.pdf}
\caption{The boxplot shows the distribution of normalized psSAR10g Brain values across different frequencies for Ella. Each box spans from the first quartile (Q1) to the third quartile (Q3), with the median line shown inside the box. The whiskers extend to show the range of the data, and points beyond the whiskers are outliers.}
\label{fig:1_71}
\end{figure}

The distribution of normalized psSAR10g Brain (Thelonious) can be found in Figure~\ref{fig:1_72}. \\\\

\begin{figure}[H]
\centering
\includegraphics[width=\columnwidth]{../../../plots/far_field/thelonious/boxplot/boxplot_psSAR10g_brain_all.pdf}
\caption{The boxplot shows the distribution of normalized psSAR10g Brain values across different frequencies for Thelonious. Each box spans from the first quartile (Q1) to the third quartile (Q3), with the median line shown inside the box. The whiskers extend to show the range of the data, and points beyond the whiskers are outliers.}
\label{fig:1_72}
\end{figure}

\subsection{psSAR10g Brain Environmental}

The distribution of normalized psSAR10g Brain - Environmental can be found in Figure~\ref{fig:1_73}. \\\\

\begin{figure}[H]
\centering
\includegraphics[width=\columnwidth]{../../../plots/far_field/duke/boxplot/boxplot_psSAR10g_brain_environmental.pdf}
\caption{The boxplot shows the distribution of normalized psSAR10g Brain values across different frequencies for the \textit{Environmental} scenario for Duke. Each box spans from the first quartile (Q1) to the third quartile (Q3), with the median line shown inside the box. The whiskers extend to show the range of the data, and points beyond the whiskers are outliers.}
\label{fig:1_73}
\end{figure}

The distribution of normalized psSAR10g Brain - Environmental can be found in Figure~\ref{fig:1_74}. \\\\

\begin{figure}[H]
\centering
\includegraphics[width=\columnwidth]{../../../plots/far_field/eartha/boxplot/boxplot_psSAR10g_brain_environmental.pdf}
\caption{The boxplot shows the distribution of normalized psSAR10g Brain values across different frequencies for the \textit{Environmental} scenario for Eartha. Each box spans from the first quartile (Q1) to the third quartile (Q3), with the median line shown inside the box. The whiskers extend to show the range of the data, and points beyond the whiskers are outliers.}
\label{fig:1_74}
\end{figure}

The distribution of normalized psSAR10g Brain - Environmental can be found in Figure~\ref{fig:1_75}. \\\\

\begin{figure}[H]
\centering
\includegraphics[width=\columnwidth]{../../../plots/far_field/ella/boxplot/boxplot_psSAR10g_brain_environmental.pdf}
\caption{The boxplot shows the distribution of normalized psSAR10g Brain values across different frequencies for the \textit{Environmental} scenario for Ella. Each box spans from the first quartile (Q1) to the third quartile (Q3), with the median line shown inside the box. The whiskers extend to show the range of the data, and points beyond the whiskers are outliers.}
\label{fig:1_75}
\end{figure}

The distribution of normalized psSAR10g Brain - Environmental can be found in Figure~\ref{fig:1_76}. \\\\

\begin{figure}[H]
\centering
\includegraphics[width=\columnwidth]{../../../plots/far_field/thelonious/boxplot/boxplot_psSAR10g_brain_environmental.pdf}
\caption{The boxplot shows the distribution of normalized psSAR10g Brain values across different frequencies for the \textit{Environmental} scenario for Thelonious. Each box spans from the first quartile (Q1) to the third quartile (Q3), with the median line shown inside the box. The whiskers extend to show the range of the data, and points beyond the whiskers are outliers.}
\label{fig:1_76}
\end{figure}

\subsection{psSAR10g Eyes All}

The distribution of normalized psSAR10g Eyes (Duke) can be found in Figure~\ref{fig:1_77}. \\\\

\begin{figure}[H]
\centering
\includegraphics[width=\columnwidth]{../../../plots/far_field/duke/boxplot/boxplot_psSAR10g_eyes_all.pdf}
\caption{The boxplot shows the distribution of normalized psSAR10g Eyes values across different frequencies for Duke. Each box spans from the first quartile (Q1) to the third quartile (Q3), with the median line shown inside the box. The whiskers extend to show the range of the data, and points beyond the whiskers are outliers.}
\label{fig:1_77}
\end{figure}

The distribution of normalized psSAR10g Eyes (Eartha) can be found in Figure~\ref{fig:1_78}. \\\\

\begin{figure}[H]
\centering
\includegraphics[width=\columnwidth]{../../../plots/far_field/eartha/boxplot/boxplot_psSAR10g_eyes_all.pdf}
\caption{The boxplot shows the distribution of normalized psSAR10g Eyes values across different frequencies for Eartha. Each box spans from the first quartile (Q1) to the third quartile (Q3), with the median line shown inside the box. The whiskers extend to show the range of the data, and points beyond the whiskers are outliers.}
\label{fig:1_78}
\end{figure}

The distribution of normalized psSAR10g Eyes (Ella) can be found in Figure~\ref{fig:1_79}. \\\\

\begin{figure}[H]
\centering
\includegraphics[width=\columnwidth]{../../../plots/far_field/ella/boxplot/boxplot_psSAR10g_eyes_all.pdf}
\caption{The boxplot shows the distribution of normalized psSAR10g Eyes values across different frequencies for Ella. Each box spans from the first quartile (Q1) to the third quartile (Q3), with the median line shown inside the box. The whiskers extend to show the range of the data, and points beyond the whiskers are outliers.}
\label{fig:1_79}
\end{figure}

The distribution of normalized psSAR10g Eyes (Thelonious) can be found in Figure~\ref{fig:1_80}. \\\\

\begin{figure}[H]
\centering
\includegraphics[width=\columnwidth]{../../../plots/far_field/thelonious/boxplot/boxplot_psSAR10g_eyes_all.pdf}
\caption{The boxplot shows the distribution of normalized psSAR10g Eyes values across different frequencies for Thelonious. Each box spans from the first quartile (Q1) to the third quartile (Q3), with the median line shown inside the box. The whiskers extend to show the range of the data, and points beyond the whiskers are outliers.}
\label{fig:1_80}
\end{figure}

\subsection{psSAR10g Eyes Environmental}

The distribution of normalized psSAR10g Eyes - Environmental can be found in Figure~\ref{fig:1_81}. \\\\

\begin{figure}[H]
\centering
\includegraphics[width=\columnwidth]{../../../plots/far_field/duke/boxplot/boxplot_psSAR10g_eyes_environmental.pdf}
\caption{The boxplot shows the distribution of normalized psSAR10g Eyes values across different frequencies for the \textit{Environmental} scenario for Duke. Each box spans from the first quartile (Q1) to the third quartile (Q3), with the median line shown inside the box. The whiskers extend to show the range of the data, and points beyond the whiskers are outliers.}
\label{fig:1_81}
\end{figure}

The distribution of normalized psSAR10g Eyes - Environmental can be found in Figure~\ref{fig:1_82}. \\\\

\begin{figure}[H]
\centering
\includegraphics[width=\columnwidth]{../../../plots/far_field/eartha/boxplot/boxplot_psSAR10g_eyes_environmental.pdf}
\caption{The boxplot shows the distribution of normalized psSAR10g Eyes values across different frequencies for the \textit{Environmental} scenario for Eartha. Each box spans from the first quartile (Q1) to the third quartile (Q3), with the median line shown inside the box. The whiskers extend to show the range of the data, and points beyond the whiskers are outliers.}
\label{fig:1_82}
\end{figure}

The distribution of normalized psSAR10g Eyes - Environmental can be found in Figure~\ref{fig:1_83}. \\\\

\begin{figure}[H]
\centering
\includegraphics[width=\columnwidth]{../../../plots/far_field/ella/boxplot/boxplot_psSAR10g_eyes_environmental.pdf}
\caption{The boxplot shows the distribution of normalized psSAR10g Eyes values across different frequencies for the \textit{Environmental} scenario for Ella. Each box spans from the first quartile (Q1) to the third quartile (Q3), with the median line shown inside the box. The whiskers extend to show the range of the data, and points beyond the whiskers are outliers.}
\label{fig:1_83}
\end{figure}

The distribution of normalized psSAR10g Eyes - Environmental can be found in Figure~\ref{fig:1_84}. \\\\

\begin{figure}[H]
\centering
\includegraphics[width=\columnwidth]{../../../plots/far_field/thelonious/boxplot/boxplot_psSAR10g_eyes_environmental.pdf}
\caption{The boxplot shows the distribution of normalized psSAR10g Eyes values across different frequencies for the \textit{Environmental} scenario for Thelonious. Each box spans from the first quartile (Q1) to the third quartile (Q3), with the median line shown inside the box. The whiskers extend to show the range of the data, and points beyond the whiskers are outliers.}
\label{fig:1_84}
\end{figure}

\subsection{psSAR10g Genitals All}

The distribution of normalized psSAR10g Genitals (Duke) can be found in Figure~\ref{fig:1_85}. \\\\

\begin{figure}[H]
\centering
\includegraphics[width=\columnwidth]{../../../plots/far_field/duke/boxplot/boxplot_psSAR10g_genitals_all.pdf}
\caption{The boxplot shows the distribution of normalized psSAR10g Genitals values across different frequencies for Duke. Each box spans from the first quartile (Q1) to the third quartile (Q3), with the median line shown inside the box. The whiskers extend to show the range of the data, and points beyond the whiskers are outliers.}
\label{fig:1_85}
\end{figure}

The distribution of normalized psSAR10g Genitals (Eartha) can be found in Figure~\ref{fig:1_86}. \\\\

\begin{figure}[H]
\centering
\includegraphics[width=\columnwidth]{../../../plots/far_field/eartha/boxplot/boxplot_psSAR10g_genitals_all.pdf}
\caption{The boxplot shows the distribution of normalized psSAR10g Genitals values across different frequencies for Eartha. Each box spans from the first quartile (Q1) to the third quartile (Q3), with the median line shown inside the box. The whiskers extend to show the range of the data, and points beyond the whiskers are outliers.}
\label{fig:1_86}
\end{figure}

The distribution of normalized psSAR10g Genitals (Ella) can be found in Figure~\ref{fig:1_87}. \\\\

\begin{figure}[H]
\centering
\includegraphics[width=\columnwidth]{../../../plots/far_field/ella/boxplot/boxplot_psSAR10g_genitals_all.pdf}
\caption{The boxplot shows the distribution of normalized psSAR10g Genitals values across different frequencies for Ella. Each box spans from the first quartile (Q1) to the third quartile (Q3), with the median line shown inside the box. The whiskers extend to show the range of the data, and points beyond the whiskers are outliers.}
\label{fig:1_87}
\end{figure}

The distribution of normalized psSAR10g Genitals (Thelonious) can be found in Figure~\ref{fig:1_88}. \\\\

\begin{figure}[H]
\centering
\includegraphics[width=\columnwidth]{../../../plots/far_field/thelonious/boxplot/boxplot_psSAR10g_genitals_all.pdf}
\caption{The boxplot shows the distribution of normalized psSAR10g Genitals values across different frequencies for Thelonious. Each box spans from the first quartile (Q1) to the third quartile (Q3), with the median line shown inside the box. The whiskers extend to show the range of the data, and points beyond the whiskers are outliers.}
\label{fig:1_88}
\end{figure}

\subsection{psSAR10g Genitals Environmental}

The distribution of normalized psSAR10g Genitals - Environmental can be found in Figure~\ref{fig:1_89}. \\\\

\begin{figure}[H]
\centering
\includegraphics[width=\columnwidth]{../../../plots/far_field/duke/boxplot/boxplot_psSAR10g_genitals_environmental.pdf}
\caption{The boxplot shows the distribution of normalized psSAR10g Genitals values across different frequencies for the \textit{Environmental} scenario for Duke. Each box spans from the first quartile (Q1) to the third quartile (Q3), with the median line shown inside the box. The whiskers extend to show the range of the data, and points beyond the whiskers are outliers.}
\label{fig:1_89}
\end{figure}

The distribution of normalized psSAR10g Genitals - Environmental can be found in Figure~\ref{fig:1_90}. \\\\

\begin{figure}[H]
\centering
\includegraphics[width=\columnwidth]{../../../plots/far_field/eartha/boxplot/boxplot_psSAR10g_genitals_environmental.pdf}
\caption{The boxplot shows the distribution of normalized psSAR10g Genitals values across different frequencies for the \textit{Environmental} scenario for Eartha. Each box spans from the first quartile (Q1) to the third quartile (Q3), with the median line shown inside the box. The whiskers extend to show the range of the data, and points beyond the whiskers are outliers.}
\label{fig:1_90}
\end{figure}

The distribution of normalized psSAR10g Genitals - Environmental can be found in Figure~\ref{fig:1_91}. \\\\

\begin{figure}[H]
\centering
\includegraphics[width=\columnwidth]{../../../plots/far_field/ella/boxplot/boxplot_psSAR10g_genitals_environmental.pdf}
\caption{The boxplot shows the distribution of normalized psSAR10g Genitals values across different frequencies for the \textit{Environmental} scenario for Ella. Each box spans from the first quartile (Q1) to the third quartile (Q3), with the median line shown inside the box. The whiskers extend to show the range of the data, and points beyond the whiskers are outliers.}
\label{fig:1_91}
\end{figure}

The distribution of normalized psSAR10g Genitals - Environmental can be found in Figure~\ref{fig:1_92}. \\\\

\begin{figure}[H]
\centering
\includegraphics[width=\columnwidth]{../../../plots/far_field/thelonious/boxplot/boxplot_psSAR10g_genitals_environmental.pdf}
\caption{The boxplot shows the distribution of normalized psSAR10g Genitals values across different frequencies for the \textit{Environmental} scenario for Thelonious. Each box spans from the first quartile (Q1) to the third quartile (Q3), with the median line shown inside the box. The whiskers extend to show the range of the data, and points beyond the whiskers are outliers.}
\label{fig:1_92}
\end{figure}

\subsection{psSAR10g Skin All}

The distribution of normalized psSAR10g Skin (Duke) can be found in Figure~\ref{fig:1_93}. \\\\

\begin{figure}[H]
\centering
\includegraphics[width=\columnwidth]{../../../plots/far_field/duke/boxplot/boxplot_psSAR10g_skin_all.pdf}
\caption{The boxplot shows the distribution of normalized psSAR10g Skin values across different frequencies for Duke. Each box spans from the first quartile (Q1) to the third quartile (Q3), with the median line shown inside the box. The whiskers extend to show the range of the data, and points beyond the whiskers are outliers.}
\label{fig:1_93}
\end{figure}

The distribution of normalized psSAR10g Skin (Eartha) can be found in Figure~\ref{fig:1_94}. \\\\

\begin{figure}[H]
\centering
\includegraphics[width=\columnwidth]{../../../plots/far_field/eartha/boxplot/boxplot_psSAR10g_skin_all.pdf}
\caption{The boxplot shows the distribution of normalized psSAR10g Skin values across different frequencies for Eartha. Each box spans from the first quartile (Q1) to the third quartile (Q3), with the median line shown inside the box. The whiskers extend to show the range of the data, and points beyond the whiskers are outliers.}
\label{fig:1_94}
\end{figure}

The distribution of normalized psSAR10g Skin (Ella) can be found in Figure~\ref{fig:1_95}. \\\\

\begin{figure}[H]
\centering
\includegraphics[width=\columnwidth]{../../../plots/far_field/ella/boxplot/boxplot_psSAR10g_skin_all.pdf}
\caption{The boxplot shows the distribution of normalized psSAR10g Skin values across different frequencies for Ella. Each box spans from the first quartile (Q1) to the third quartile (Q3), with the median line shown inside the box. The whiskers extend to show the range of the data, and points beyond the whiskers are outliers.}
\label{fig:1_95}
\end{figure}

The distribution of normalized psSAR10g Skin (Thelonious) can be found in Figure~\ref{fig:1_96}. \\\\

\begin{figure}[H]
\centering
\includegraphics[width=\columnwidth]{../../../plots/far_field/thelonious/boxplot/boxplot_psSAR10g_skin_all.pdf}
\caption{The boxplot shows the distribution of normalized psSAR10g Skin values across different frequencies for Thelonious. Each box spans from the first quartile (Q1) to the third quartile (Q3), with the median line shown inside the box. The whiskers extend to show the range of the data, and points beyond the whiskers are outliers.}
\label{fig:1_96}
\end{figure}

\subsection{psSAR10g Skin Environmental}

The distribution of normalized psSAR10g Skin - Environmental can be found in Figure~\ref{fig:1_97}. \\\\

\begin{figure}[H]
\centering
\includegraphics[width=\columnwidth]{../../../plots/far_field/duke/boxplot/boxplot_psSAR10g_skin_environmental.pdf}
\caption{The boxplot shows the distribution of normalized psSAR10g Skin values across different frequencies for the \textit{Environmental} scenario for Duke. Each box spans from the first quartile (Q1) to the third quartile (Q3), with the median line shown inside the box. The whiskers extend to show the range of the data, and points beyond the whiskers are outliers.}
\label{fig:1_97}
\end{figure}

The distribution of normalized psSAR10g Skin - Environmental can be found in Figure~\ref{fig:1_98}. \\\\

\begin{figure}[H]
\centering
\includegraphics[width=\columnwidth]{../../../plots/far_field/eartha/boxplot/boxplot_psSAR10g_skin_environmental.pdf}
\caption{The boxplot shows the distribution of normalized psSAR10g Skin values across different frequencies for the \textit{Environmental} scenario for Eartha. Each box spans from the first quartile (Q1) to the third quartile (Q3), with the median line shown inside the box. The whiskers extend to show the range of the data, and points beyond the whiskers are outliers.}
\label{fig:1_98}
\end{figure}

The distribution of normalized psSAR10g Skin - Environmental can be found in Figure~\ref{fig:1_99}. \\\\

\begin{figure}[H]
\centering
\includegraphics[width=\columnwidth]{../../../plots/far_field/ella/boxplot/boxplot_psSAR10g_skin_environmental.pdf}
\caption{The boxplot shows the distribution of normalized psSAR10g Skin values across different frequencies for the \textit{Environmental} scenario for Ella. Each box spans from the first quartile (Q1) to the third quartile (Q3), with the median line shown inside the box. The whiskers extend to show the range of the data, and points beyond the whiskers are outliers.}
\label{fig:1_99}
\end{figure}

The distribution of normalized psSAR10g Skin - Environmental can be found in Figure~\ref{fig:1_100}. \\\\

\begin{figure}[H]
\centering
\includegraphics[width=\columnwidth]{../../../plots/far_field/thelonious/boxplot/boxplot_psSAR10g_skin_environmental.pdf}
\caption{The boxplot shows the distribution of normalized psSAR10g Skin values across different frequencies for the \textit{Environmental} scenario for Thelonious. Each box spans from the first quartile (Q1) to the third quartile (Q3), with the median line shown inside the box. The whiskers extend to show the range of the data, and points beyond the whiskers are outliers.}
\label{fig:1_100}
\end{figure}

\newpage

\section{Line Plots}

\subsection{Cross Phantom SAR Whole Body}

The cross Phantom Comparison Whole Body SAR (Duke) can be found in Figure~\ref{fig:2_1}. \\\\

\begin{figure}[H]
\centering
\includegraphics[width=\columnwidth]{../../../plots/far_field/duke/line/line_cross_phantom_SAR_whole_body.pdf}
\caption{Cross-phantom comparison of whole-body sar as a function of frequency. Lines show means across all incident directions and polarizations. Whiskers indicate 25th-75th percentile range. Children show approximately 1.5-2× higher absorption than adults.}
\label{fig:2_1}
\end{figure}

\subsection{Direction Polarization Peak SAR by Polarization}

The far field Peak SAR 10g vs frequency by direction grouped by polarization (Duke) can be found in Figure~\ref{fig:2_2}. \\\\

\begin{figure*}[htbp]
\centering
\includegraphics[width=\textwidth]{../../../plots/far_field/duke/line/line_direction_polarization_peak_sar_by_polarization.pdf}
\caption{The dual-panel line plot shows normalized Peak SAR (10g) values for Duke across frequencies. Left panel shows Theta polarization, right panel shows Phi polarization. Each line represents a different incident direction (from left/right, front/back, above/below).}
\label{fig:2_2}
\end{figure*}

The far field Peak SAR 10g vs frequency by direction grouped by polarization (Eartha) can be found in Figure~\ref{fig:2_3}. \\\\

\begin{figure*}[htbp]
\centering
\includegraphics[width=\textwidth]{../../../plots/far_field/eartha/line/line_direction_polarization_peak_sar_by_polarization.pdf}
\caption{The dual-panel line plot shows normalized Peak SAR (10g) values for Eartha across frequencies. Left panel shows Theta polarization, right panel shows Phi polarization. Each line represents a different incident direction (from left/right, front/back, above/below).}
\label{fig:2_3}
\end{figure*}

The far field Peak SAR 10g vs frequency by direction grouped by polarization (Ella) can be found in Figure~\ref{fig:2_4}. \\\\

\begin{figure*}[htbp]
\centering
\includegraphics[width=\textwidth]{../../../plots/far_field/ella/line/line_direction_polarization_peak_sar_by_polarization.pdf}
\caption{The dual-panel line plot shows normalized Peak SAR (10g) values for Ella across frequencies. Left panel shows Theta polarization, right panel shows Phi polarization. Each line represents a different incident direction (from left/right, front/back, above/below).}
\label{fig:2_4}
\end{figure*}

The far field Peak SAR 10g vs frequency by direction grouped by polarization (Thelonious) can be found in Figure~\ref{fig:2_5}. \\\\

\begin{figure*}[htbp]
\centering
\includegraphics[width=\textwidth]{../../../plots/far_field/thelonious/line/line_direction_polarization_peak_sar_by_polarization.pdf}
\caption{The dual-panel line plot shows normalized Peak SAR (10g) values for Thelonious across frequencies. Left panel shows Theta polarization, right panel shows Phi polarization. Each line represents a different incident direction (from left/right, front/back, above/below).}
\label{fig:2_5}
\end{figure*}

\subsection{Direction Polarization SAR Brain by Polarization}

The far field Brain SAR vs frequency by direction grouped by polarization (Duke) can be found in Figure~\ref{fig:2_6}. \\\\

\begin{figure*}[htbp]
\centering
\includegraphics[width=\textwidth]{../../../plots/far_field/duke/line/line_direction_polarization_SAR_brain_by_polarization.pdf}
\caption{The dual-panel line plot shows normalized Brain SAR values for Duke across frequencies. Left panel shows Theta polarization, right panel shows Phi polarization. Each line represents a different incident direction (from left/right, front/back, above/below).}
\label{fig:2_6}
\end{figure*}

The far field Brain SAR vs frequency by direction grouped by polarization (Eartha) can be found in Figure~\ref{fig:2_7}. \\\\

\begin{figure*}[htbp]
\centering
\includegraphics[width=\textwidth]{../../../plots/far_field/eartha/line/line_direction_polarization_SAR_brain_by_polarization.pdf}
\caption{The dual-panel line plot shows normalized Brain SAR values for Eartha across frequencies. Left panel shows Theta polarization, right panel shows Phi polarization. Each line represents a different incident direction (from left/right, front/back, above/below).}
\label{fig:2_7}
\end{figure*}

The far field Brain SAR vs frequency by direction grouped by polarization (Ella) can be found in Figure~\ref{fig:2_8}. \\\\

\begin{figure*}[htbp]
\centering
\includegraphics[width=\textwidth]{../../../plots/far_field/ella/line/line_direction_polarization_SAR_brain_by_polarization.pdf}
\caption{The dual-panel line plot shows normalized Brain SAR values for Ella across frequencies. Left panel shows Theta polarization, right panel shows Phi polarization. Each line represents a different incident direction (from left/right, front/back, above/below).}
\label{fig:2_8}
\end{figure*}

The far field Brain SAR vs frequency by direction grouped by polarization (Thelonious) can be found in Figure~\ref{fig:2_9}. \\\\

\begin{figure*}[htbp]
\centering
\includegraphics[width=\textwidth]{../../../plots/far_field/thelonious/line/line_direction_polarization_SAR_brain_by_polarization.pdf}
\caption{The dual-panel line plot shows normalized Brain SAR values for Thelonious across frequencies. Left panel shows Theta polarization, right panel shows Phi polarization. Each line represents a different incident direction (from left/right, front/back, above/below).}
\label{fig:2_9}
\end{figure*}

\subsection{Direction Polarization SAR Eyes by Polarization}

The far field Eyes SAR vs frequency by direction grouped by polarization (Duke) can be found in Figure~\ref{fig:2_10}. \\\\

\begin{figure*}[htbp]
\centering
\includegraphics[width=\textwidth]{../../../plots/far_field/duke/line/line_direction_polarization_SAR_eyes_by_polarization.pdf}
\caption{The dual-panel line plot shows normalized Eyes SAR values for Duke across frequencies. Left panel shows Theta polarization, right panel shows Phi polarization. Each line represents a different incident direction (from left/right, front/back, above/below).}
\label{fig:2_10}
\end{figure*}

The far field Eyes SAR vs frequency by direction grouped by polarization (Eartha) can be found in Figure~\ref{fig:2_11}. \\\\

\begin{figure*}[htbp]
\centering
\includegraphics[width=\textwidth]{../../../plots/far_field/eartha/line/line_direction_polarization_SAR_eyes_by_polarization.pdf}
\caption{The dual-panel line plot shows normalized Eyes SAR values for Eartha across frequencies. Left panel shows Theta polarization, right panel shows Phi polarization. Each line represents a different incident direction (from left/right, front/back, above/below).}
\label{fig:2_11}
\end{figure*}

The far field Eyes SAR vs frequency by direction grouped by polarization (Ella) can be found in Figure~\ref{fig:2_12}. \\\\

\begin{figure*}[htbp]
\centering
\includegraphics[width=\textwidth]{../../../plots/far_field/ella/line/line_direction_polarization_SAR_eyes_by_polarization.pdf}
\caption{The dual-panel line plot shows normalized Eyes SAR values for Ella across frequencies. Left panel shows Theta polarization, right panel shows Phi polarization. Each line represents a different incident direction (from left/right, front/back, above/below).}
\label{fig:2_12}
\end{figure*}

The far field Eyes SAR vs frequency by direction grouped by polarization (Thelonious) can be found in Figure~\ref{fig:2_13}. \\\\

\begin{figure*}[htbp]
\centering
\includegraphics[width=\textwidth]{../../../plots/far_field/thelonious/line/line_direction_polarization_SAR_eyes_by_polarization.pdf}
\caption{The dual-panel line plot shows normalized Eyes SAR values for Thelonious across frequencies. Left panel shows Theta polarization, right panel shows Phi polarization. Each line represents a different incident direction (from left/right, front/back, above/below).}
\label{fig:2_13}
\end{figure*}

\subsection{Direction Polarization SAR Genitals by Polarization}

The far field Genitals SAR vs frequency by direction grouped by polarization (Duke) can be found in Figure~\ref{fig:2_14}. \\\\

\begin{figure*}[htbp]
\centering
\includegraphics[width=\textwidth]{../../../plots/far_field/duke/line/line_direction_polarization_SAR_genitals_by_polarization.pdf}
\caption{The dual-panel line plot shows normalized Genitals SAR values for Duke across frequencies. Left panel shows Theta polarization, right panel shows Phi polarization. Each line represents a different incident direction (from left/right, front/back, above/below).}
\label{fig:2_14}
\end{figure*}

The far field Genitals SAR vs frequency by direction grouped by polarization (Eartha) can be found in Figure~\ref{fig:2_15}. \\\\

\begin{figure*}[htbp]
\centering
\includegraphics[width=\textwidth]{../../../plots/far_field/eartha/line/line_direction_polarization_SAR_genitals_by_polarization.pdf}
\caption{The dual-panel line plot shows normalized Genitals SAR values for Eartha across frequencies. Left panel shows Theta polarization, right panel shows Phi polarization. Each line represents a different incident direction (from left/right, front/back, above/below).}
\label{fig:2_15}
\end{figure*}

The far field Genitals SAR vs frequency by direction grouped by polarization (Ella) can be found in Figure~\ref{fig:2_16}. \\\\

\begin{figure*}[htbp]
\centering
\includegraphics[width=\textwidth]{../../../plots/far_field/ella/line/line_direction_polarization_SAR_genitals_by_polarization.pdf}
\caption{The dual-panel line plot shows normalized Genitals SAR values for Ella across frequencies. Left panel shows Theta polarization, right panel shows Phi polarization. Each line represents a different incident direction (from left/right, front/back, above/below).}
\label{fig:2_16}
\end{figure*}

The far field Genitals SAR vs frequency by direction grouped by polarization (Thelonious) can be found in Figure~\ref{fig:2_17}. \\\\

\begin{figure*}[htbp]
\centering
\includegraphics[width=\textwidth]{../../../plots/far_field/thelonious/line/line_direction_polarization_SAR_genitals_by_polarization.pdf}
\caption{The dual-panel line plot shows normalized Genitals SAR values for Thelonious across frequencies. Left panel shows Theta polarization, right panel shows Phi polarization. Each line represents a different incident direction (from left/right, front/back, above/below).}
\label{fig:2_17}
\end{figure*}

\subsection{Direction Polarization SAR Skin by Polarization}

The far field Skin SAR vs frequency by direction grouped by polarization (Duke) can be found in Figure~\ref{fig:2_18}. \\\\

\begin{figure*}[htbp]
\centering
\includegraphics[width=\textwidth]{../../../plots/far_field/duke/line/line_direction_polarization_SAR_skin_by_polarization.pdf}
\caption{The dual-panel line plot shows normalized Skin SAR values for Duke across frequencies. Left panel shows Theta polarization, right panel shows Phi polarization. Each line represents a different incident direction (from left/right, front/back, above/below).}
\label{fig:2_18}
\end{figure*}

The far field Skin SAR vs frequency by direction grouped by polarization (Eartha) can be found in Figure~\ref{fig:2_19}. \\\\

\begin{figure*}[htbp]
\centering
\includegraphics[width=\textwidth]{../../../plots/far_field/eartha/line/line_direction_polarization_SAR_skin_by_polarization.pdf}
\caption{The dual-panel line plot shows normalized Skin SAR values for Eartha across frequencies. Left panel shows Theta polarization, right panel shows Phi polarization. Each line represents a different incident direction (from left/right, front/back, above/below).}
\label{fig:2_19}
\end{figure*}

The far field Skin SAR vs frequency by direction grouped by polarization (Ella) can be found in Figure~\ref{fig:2_20}. \\\\

\begin{figure*}[htbp]
\centering
\includegraphics[width=\textwidth]{../../../plots/far_field/ella/line/line_direction_polarization_SAR_skin_by_polarization.pdf}
\caption{The dual-panel line plot shows normalized Skin SAR values for Ella across frequencies. Left panel shows Theta polarization, right panel shows Phi polarization. Each line represents a different incident direction (from left/right, front/back, above/below).}
\label{fig:2_20}
\end{figure*}

The far field Skin SAR vs frequency by direction grouped by polarization (Thelonious) can be found in Figure~\ref{fig:2_21}. \\\\

\begin{figure*}[htbp]
\centering
\includegraphics[width=\textwidth]{../../../plots/far_field/thelonious/line/line_direction_polarization_SAR_skin_by_polarization.pdf}
\caption{The dual-panel line plot shows normalized Skin SAR values for Thelonious across frequencies. Left panel shows Theta polarization, right panel shows Phi polarization. Each line represents a different incident direction (from left/right, front/back, above/below).}
\label{fig:2_21}
\end{figure*}

\subsection{Direction Polarization SAR Whole Body by Polarization}

The far field Whole Body SAR vs frequency by direction grouped by polarization (Duke) can be found in Figure~\ref{fig:2_22}. \\\\

\begin{figure*}[htbp]
\centering
\includegraphics[width=\textwidth]{../../../plots/far_field/duke/line/line_direction_polarization_SAR_whole_body_by_polarization.pdf}
\caption{The dual-panel line plot shows normalized Whole-Body SAR values for Duke across frequencies. Left panel shows Theta polarization, right panel shows Phi polarization. Each line represents a different incident direction (from left/right, front/back, above/below).}
\label{fig:2_22}
\end{figure*}

The far field Whole Body SAR vs frequency by direction grouped by polarization (Eartha) can be found in Figure~\ref{fig:2_23}. \\\\

\begin{figure*}[htbp]
\centering
\includegraphics[width=\textwidth]{../../../plots/far_field/eartha/line/line_direction_polarization_SAR_whole_body_by_polarization.pdf}
\caption{The dual-panel line plot shows normalized Whole-Body SAR values for Eartha across frequencies. Left panel shows Theta polarization, right panel shows Phi polarization. Each line represents a different incident direction (from left/right, front/back, above/below).}
\label{fig:2_23}
\end{figure*}

The far field Whole Body SAR vs frequency by direction grouped by polarization (Ella) can be found in Figure~\ref{fig:2_24}. \\\\

\begin{figure*}[htbp]
\centering
\includegraphics[width=\textwidth]{../../../plots/far_field/ella/line/line_direction_polarization_SAR_whole_body_by_polarization.pdf}
\caption{The dual-panel line plot shows normalized Whole-Body SAR values for Ella across frequencies. Left panel shows Theta polarization, right panel shows Phi polarization. Each line represents a different incident direction (from left/right, front/back, above/below).}
\label{fig:2_24}
\end{figure*}

The far field Whole Body SAR vs frequency by direction grouped by polarization (Thelonious) can be found in Figure~\ref{fig:2_25}. \\\\

\begin{figure*}[htbp]
\centering
\includegraphics[width=\textwidth]{../../../plots/far_field/thelonious/line/line_direction_polarization_SAR_whole_body_by_polarization.pdf}
\caption{The dual-panel line plot shows normalized Whole-Body SAR values for Thelonious across frequencies. Left panel shows Theta polarization, right panel shows Phi polarization. Each line represents a different incident direction (from left/right, front/back, above/below).}
\label{fig:2_25}
\end{figure*}

\subsection{Direction Polarization psSAR10g Brain by Polarization}

The far field psSAR10g Brain vs frequency by direction grouped by polarization (Duke) can be found in Figure~\ref{fig:2_26}. \\\\

\begin{figure*}[htbp]
\centering
\includegraphics[width=\textwidth]{../../../plots/far_field/duke/line/line_direction_polarization_psSAR10g_brain_by_polarization.pdf}
\caption{The dual-panel line plot shows normalized psSAR10g Brain values for Duke across frequencies. Left panel shows Theta polarization, right panel shows Phi polarization. Each line represents a different incident direction (from left/right, front/back, above/below).}
\label{fig:2_26}
\end{figure*}

The far field psSAR10g Brain vs frequency by direction grouped by polarization (Eartha) can be found in Figure~\ref{fig:2_27}. \\\\

\begin{figure*}[htbp]
\centering
\includegraphics[width=\textwidth]{../../../plots/far_field/eartha/line/line_direction_polarization_psSAR10g_brain_by_polarization.pdf}
\caption{The dual-panel line plot shows normalized psSAR10g Brain values for Eartha across frequencies. Left panel shows Theta polarization, right panel shows Phi polarization. Each line represents a different incident direction (from left/right, front/back, above/below).}
\label{fig:2_27}
\end{figure*}

The far field psSAR10g Brain vs frequency by direction grouped by polarization (Ella) can be found in Figure~\ref{fig:2_28}. \\\\

\begin{figure*}[htbp]
\centering
\includegraphics[width=\textwidth]{../../../plots/far_field/ella/line/line_direction_polarization_psSAR10g_brain_by_polarization.pdf}
\caption{The dual-panel line plot shows normalized psSAR10g Brain values for Ella across frequencies. Left panel shows Theta polarization, right panel shows Phi polarization. Each line represents a different incident direction (from left/right, front/back, above/below).}
\label{fig:2_28}
\end{figure*}

The far field psSAR10g Brain vs frequency by direction grouped by polarization (Thelonious) can be found in Figure~\ref{fig:2_29}. \\\\

\begin{figure*}[htbp]
\centering
\includegraphics[width=\textwidth]{../../../plots/far_field/thelonious/line/line_direction_polarization_psSAR10g_brain_by_polarization.pdf}
\caption{The dual-panel line plot shows normalized psSAR10g Brain values for Thelonious across frequencies. Left panel shows Theta polarization, right panel shows Phi polarization. Each line represents a different incident direction (from left/right, front/back, above/below).}
\label{fig:2_29}
\end{figure*}

\subsection{Direction Polarization psSAR10g Eyes by Polarization}

The far field psSAR10g Eyes vs frequency by direction grouped by polarization (Duke) can be found in Figure~\ref{fig:2_30}. \\\\

\begin{figure*}[htbp]
\centering
\includegraphics[width=\textwidth]{../../../plots/far_field/duke/line/line_direction_polarization_psSAR10g_eyes_by_polarization.pdf}
\caption{The dual-panel line plot shows normalized psSAR10g Eyes values for Duke across frequencies. Left panel shows Theta polarization, right panel shows Phi polarization. Each line represents a different incident direction (from left/right, front/back, above/below).}
\label{fig:2_30}
\end{figure*}

The far field psSAR10g Eyes vs frequency by direction grouped by polarization (Eartha) can be found in Figure~\ref{fig:2_31}. \\\\

\begin{figure*}[htbp]
\centering
\includegraphics[width=\textwidth]{../../../plots/far_field/eartha/line/line_direction_polarization_psSAR10g_eyes_by_polarization.pdf}
\caption{The dual-panel line plot shows normalized psSAR10g Eyes values for Eartha across frequencies. Left panel shows Theta polarization, right panel shows Phi polarization. Each line represents a different incident direction (from left/right, front/back, above/below).}
\label{fig:2_31}
\end{figure*}

The far field psSAR10g Eyes vs frequency by direction grouped by polarization (Ella) can be found in Figure~\ref{fig:2_32}. \\\\

\begin{figure*}[htbp]
\centering
\includegraphics[width=\textwidth]{../../../plots/far_field/ella/line/line_direction_polarization_psSAR10g_eyes_by_polarization.pdf}
\caption{The dual-panel line plot shows normalized psSAR10g Eyes values for Ella across frequencies. Left panel shows Theta polarization, right panel shows Phi polarization. Each line represents a different incident direction (from left/right, front/back, above/below).}
\label{fig:2_32}
\end{figure*}

The far field psSAR10g Eyes vs frequency by direction grouped by polarization (Thelonious) can be found in Figure~\ref{fig:2_33}. \\\\

\begin{figure*}[htbp]
\centering
\includegraphics[width=\textwidth]{../../../plots/far_field/thelonious/line/line_direction_polarization_psSAR10g_eyes_by_polarization.pdf}
\caption{The dual-panel line plot shows normalized psSAR10g Eyes values for Thelonious across frequencies. Left panel shows Theta polarization, right panel shows Phi polarization. Each line represents a different incident direction (from left/right, front/back, above/below).}
\label{fig:2_33}
\end{figure*}

\subsection{Direction Polarization psSAR10g Genitals by Polarization}

The far field psSAR10g Genitals vs frequency by direction grouped by polarization (Duke) can be found in Figure~\ref{fig:2_34}. \\\\

\begin{figure*}[htbp]
\centering
\includegraphics[width=\textwidth]{../../../plots/far_field/duke/line/line_direction_polarization_psSAR10g_genitals_by_polarization.pdf}
\caption{The dual-panel line plot shows normalized psSAR10g Genitals values for Duke across frequencies. Left panel shows Theta polarization, right panel shows Phi polarization. Each line represents a different incident direction (from left/right, front/back, above/below).}
\label{fig:2_34}
\end{figure*}

The far field psSAR10g Genitals vs frequency by direction grouped by polarization (Eartha) can be found in Figure~\ref{fig:2_35}. \\\\

\begin{figure*}[htbp]
\centering
\includegraphics[width=\textwidth]{../../../plots/far_field/eartha/line/line_direction_polarization_psSAR10g_genitals_by_polarization.pdf}
\caption{The dual-panel line plot shows normalized psSAR10g Genitals values for Eartha across frequencies. Left panel shows Theta polarization, right panel shows Phi polarization. Each line represents a different incident direction (from left/right, front/back, above/below).}
\label{fig:2_35}
\end{figure*}

The far field psSAR10g Genitals vs frequency by direction grouped by polarization (Ella) can be found in Figure~\ref{fig:2_36}. \\\\

\begin{figure*}[htbp]
\centering
\includegraphics[width=\textwidth]{../../../plots/far_field/ella/line/line_direction_polarization_psSAR10g_genitals_by_polarization.pdf}
\caption{The dual-panel line plot shows normalized psSAR10g Genitals values for Ella across frequencies. Left panel shows Theta polarization, right panel shows Phi polarization. Each line represents a different incident direction (from left/right, front/back, above/below).}
\label{fig:2_36}
\end{figure*}

The far field psSAR10g Genitals vs frequency by direction grouped by polarization (Thelonious) can be found in Figure~\ref{fig:2_37}. \\\\

\begin{figure*}[htbp]
\centering
\includegraphics[width=\textwidth]{../../../plots/far_field/thelonious/line/line_direction_polarization_psSAR10g_genitals_by_polarization.pdf}
\caption{The dual-panel line plot shows normalized psSAR10g Genitals values for Thelonious across frequencies. Left panel shows Theta polarization, right panel shows Phi polarization. Each line represents a different incident direction (from left/right, front/back, above/below).}
\label{fig:2_37}
\end{figure*}

\subsection{Direction Polarization psSAR10g Skin by Polarization}

The far field psSAR10g Skin vs frequency by direction grouped by polarization (Duke) can be found in Figure~\ref{fig:2_38}. \\\\

\begin{figure*}[htbp]
\centering
\includegraphics[width=\textwidth]{../../../plots/far_field/duke/line/line_direction_polarization_psSAR10g_skin_by_polarization.pdf}
\caption{The dual-panel line plot shows normalized psSAR10g Skin values for Duke across frequencies. Left panel shows Theta polarization, right panel shows Phi polarization. Each line represents a different incident direction (from left/right, front/back, above/below).}
\label{fig:2_38}
\end{figure*}

The far field psSAR10g Skin vs frequency by direction grouped by polarization (Eartha) can be found in Figure~\ref{fig:2_39}. \\\\

\begin{figure*}[htbp]
\centering
\includegraphics[width=\textwidth]{../../../plots/far_field/eartha/line/line_direction_polarization_psSAR10g_skin_by_polarization.pdf}
\caption{The dual-panel line plot shows normalized psSAR10g Skin values for Eartha across frequencies. Left panel shows Theta polarization, right panel shows Phi polarization. Each line represents a different incident direction (from left/right, front/back, above/below).}
\label{fig:2_39}
\end{figure*}

The far field psSAR10g Skin vs frequency by direction grouped by polarization (Ella) can be found in Figure~\ref{fig:2_40}. \\\\

\begin{figure*}[htbp]
\centering
\includegraphics[width=\textwidth]{../../../plots/far_field/ella/line/line_direction_polarization_psSAR10g_skin_by_polarization.pdf}
\caption{The dual-panel line plot shows normalized psSAR10g Skin values for Ella across frequencies. Left panel shows Theta polarization, right panel shows Phi polarization. Each line represents a different incident direction (from left/right, front/back, above/below).}
\label{fig:2_40}
\end{figure*}

The far field psSAR10g Skin vs frequency by direction grouped by polarization (Thelonious) can be found in Figure~\ref{fig:2_41}. \\\\

\begin{figure*}[htbp]
\centering
\includegraphics[width=\textwidth]{../../../plots/far_field/thelonious/line/line_direction_polarization_psSAR10g_skin_by_polarization.pdf}
\caption{The dual-panel line plot shows normalized psSAR10g Skin values for Thelonious across frequencies. Left panel shows Theta polarization, right panel shows Phi polarization. Each line represents a different incident direction (from left/right, front/back, above/below).}
\label{fig:2_41}
\end{figure*}

\subsection{Environmental}

The average normalized psSAR10g - Environmental can be found in Figure~\ref{fig:2_42}. \\\\

\begin{figure}[H]
\centering
\includegraphics[width=\columnwidth]{../../../plots/far_field/duke/line/pssar10g_line_environmental.pdf}
\caption{The line plot shows average normalized psSAR10g trends for different tissue groups (Eyes, Skin, Brain, Genitals, Whole Body) across frequencies for the \textit{Environmental} scenario for Duke.}
\label{fig:2_42}
\end{figure}

The average normalized SAR - Environmental can be found in Figure~\ref{fig:2_43}. \\\\

\begin{figure}[H]
\centering
\includegraphics[width=\columnwidth]{../../../plots/far_field/duke/line/sar_line_environmental.pdf}
\caption{The line plot shows average normalized SAR trends for different tissue groups (Head, Trunk, Whole-Body, Brain, Skin, Eyes, Genitals) across frequencies for the \textit{Environmental} scenario for Duke.}
\label{fig:2_43}
\end{figure}

The average normalized psSAR10g - Environmental can be found in Figure~\ref{fig:2_44}. \\\\

\begin{figure}[H]
\centering
\includegraphics[width=\columnwidth]{../../../plots/far_field/eartha/line/pssar10g_line_environmental.pdf}
\caption{The line plot shows average normalized psSAR10g trends for different tissue groups (Eyes, Skin, Brain, Genitals, Whole Body) across frequencies for the \textit{Environmental} scenario for Eartha.}
\label{fig:2_44}
\end{figure}

The average normalized SAR - Environmental can be found in Figure~\ref{fig:2_45}. \\\\

\begin{figure}[H]
\centering
\includegraphics[width=\columnwidth]{../../../plots/far_field/eartha/line/sar_line_environmental.pdf}
\caption{The line plot shows average normalized SAR trends for different tissue groups (Head, Trunk, Whole-Body, Brain, Skin, Eyes, Genitals) across frequencies for the \textit{Environmental} scenario for Eartha.}
\label{fig:2_45}
\end{figure}

The average normalized psSAR10g - Environmental can be found in Figure~\ref{fig:2_46}. \\\\

\begin{figure}[H]
\centering
\includegraphics[width=\columnwidth]{../../../plots/far_field/ella/line/pssar10g_line_environmental.pdf}
\caption{The line plot shows average normalized psSAR10g trends for different tissue groups (Eyes, Skin, Brain, Genitals, Whole Body) across frequencies for the \textit{Environmental} scenario for Ella.}
\label{fig:2_46}
\end{figure}

The average normalized SAR - Environmental can be found in Figure~\ref{fig:2_47}. \\\\

\begin{figure}[H]
\centering
\includegraphics[width=\columnwidth]{../../../plots/far_field/ella/line/sar_line_environmental.pdf}
\caption{The line plot shows average normalized SAR trends for different tissue groups (Head, Trunk, Whole-Body, Brain, Skin, Eyes, Genitals) across frequencies for the \textit{Environmental} scenario for Ella.}
\label{fig:2_47}
\end{figure}

The average normalized psSAR10g - Environmental can be found in Figure~\ref{fig:2_48}. \\\\

\begin{figure}[H]
\centering
\includegraphics[width=\columnwidth]{../../../plots/far_field/thelonious/line/pssar10g_line_environmental.pdf}
\caption{The line plot shows average normalized psSAR10g trends for different tissue groups (Eyes, Skin, Brain, Genitals, Whole Body) across frequencies for the \textit{Environmental} scenario for Thelonious.}
\label{fig:2_48}
\end{figure}

The average normalized SAR - Environmental can be found in Figure~\ref{fig:2_49}. \\\\

\begin{figure}[H]
\centering
\includegraphics[width=\columnwidth]{../../../plots/far_field/thelonious/line/sar_line_environmental.pdf}
\caption{The line plot shows average normalized SAR trends for different tissue groups (Head, Trunk, Whole-Body, Brain, Skin, Eyes, Genitals) across frequencies for the \textit{Environmental} scenario for Thelonious.}
\label{fig:2_49}
\end{figure}

\subsection{Peak SAR Summary}

The average peak SAR 10g across all tissues (Duke) can be found in Figure~\ref{fig:2_50}. \\\\

\begin{figure}[H]
\centering
\includegraphics[width=\columnwidth]{../../../plots/far_field/duke/line/line_peak_sar_summary.pdf}
\caption{The line plot shows the trend of average peak SAR (10g) values across all tissues as a function of frequency for Duke.}
\label{fig:2_50}
\end{figure}

The average peak SAR 10g across all tissues (Eartha) can be found in Figure~\ref{fig:2_51}. \\\\

\begin{figure}[H]
\centering
\includegraphics[width=\columnwidth]{../../../plots/far_field/eartha/line/line_peak_sar_summary.pdf}
\caption{The line plot shows the trend of average peak SAR (10g) values across all tissues as a function of frequency for Eartha.}
\label{fig:2_51}
\end{figure}

The average peak SAR 10g across all tissues (Ella) can be found in Figure~\ref{fig:2_52}. \\\\

\begin{figure}[H]
\centering
\includegraphics[width=\columnwidth]{../../../plots/far_field/ella/line/line_peak_sar_summary.pdf}
\caption{The line plot shows the trend of average peak SAR (10g) values across all tissues as a function of frequency for Ella.}
\label{fig:2_52}
\end{figure}

The average peak SAR 10g across all tissues (Thelonious) can be found in Figure~\ref{fig:2_53}. \\\\

\begin{figure}[H]
\centering
\includegraphics[width=\columnwidth]{../../../plots/far_field/thelonious/line/line_peak_sar_summary.pdf}
\caption{The line plot shows the trend of average peak SAR (10g) values across all tissues as a function of frequency for Thelonious.}
\label{fig:2_53}
\end{figure}

\subsection{Polarization Ratio by Direction}

The polarization Ratio vs Frequency by Direction (Duke) can be found in Figure~\ref{fig:2_54}. \\\\

\begin{figure*}[htbp]
\centering
\includegraphics[width=\textwidth]{../../../plots/far_field/duke/line/line_polarization_ratio_by_direction.pdf}
\caption{Theta/phi polarization ratio as a function of frequency for each incident direction. Ratio > 1.0 indicates theta polarization gives higher SAR; ratio < 1.0 indicates phi dominates. The gray dashed line marks the equal polarization reference (ratio = 1.0). Significant frequency-dependent variations indicate complex polarization sensitivity.}
\label{fig:2_54}
\end{figure*}

The polarization Ratio vs Frequency by Direction (Eartha) can be found in Figure~\ref{fig:2_55}. \\\\

\begin{figure*}[htbp]
\centering
\includegraphics[width=\textwidth]{../../../plots/far_field/eartha/line/line_polarization_ratio_by_direction.pdf}
\caption{Theta/phi polarization ratio as a function of frequency for each incident direction. Ratio > 1.0 indicates theta polarization gives higher SAR; ratio < 1.0 indicates phi dominates. The gray dashed line marks the equal polarization reference (ratio = 1.0). Significant frequency-dependent variations indicate complex polarization sensitivity.}
\label{fig:2_55}
\end{figure*}

The polarization Ratio vs Frequency by Direction (Ella) can be found in Figure~\ref{fig:2_56}. \\\\

\begin{figure*}[htbp]
\centering
\includegraphics[width=\textwidth]{../../../plots/far_field/ella/line/line_polarization_ratio_by_direction.pdf}
\caption{Theta/phi polarization ratio as a function of frequency for each incident direction. Ratio > 1.0 indicates theta polarization gives higher SAR; ratio < 1.0 indicates phi dominates. The gray dashed line marks the equal polarization reference (ratio = 1.0). Significant frequency-dependent variations indicate complex polarization sensitivity.}
\label{fig:2_56}
\end{figure*}

The polarization Ratio vs Frequency by Direction (Thelonious) can be found in Figure~\ref{fig:2_57}. \\\\

\begin{figure*}[htbp]
\centering
\includegraphics[width=\textwidth]{../../../plots/far_field/thelonious/line/line_polarization_ratio_by_direction.pdf}
\caption{Theta/phi polarization ratio as a function of frequency for each incident direction. Ratio > 1.0 indicates theta polarization gives higher SAR; ratio < 1.0 indicates phi dominates. The gray dashed line marks the equal polarization reference (ratio = 1.0). Significant frequency-dependent variations indicate complex polarization sensitivity.}
\label{fig:2_57}
\end{figure*}

\newpage

\section{Heatmaps}

\subsection{Avg SAR Summary}

The average SAR mW kg$^{ 1}$ per tissue (Duke) can be found in Figure~\ref{fig:3_1}. \\\\

\begin{figure}[H]
\centering
\includegraphics[width=\columnwidth]{../../../plots/far_field/duke/heatmap/heatmap_avg_sar_summary.pdf}
\caption{The heatmap shows Average SAR values per tissue across frequencies for Duke. The top panel shows individual tissues, and the bottom panel shows organ group summaries. Tissues are colored by group (red=eyes, green=skin, blue=brain, purple=genitals).}
\label{fig:3_1}
\end{figure}

The average SAR mW kg$^{ 1}$ per tissue (Eartha) can be found in Figure~\ref{fig:3_2}. \\\\

\begin{figure}[H]
\centering
\includegraphics[width=\columnwidth]{../../../plots/far_field/eartha/heatmap/heatmap_avg_sar_summary.pdf}
\caption{The heatmap shows Average SAR values per tissue across frequencies for Eartha. The top panel shows individual tissues, and the bottom panel shows organ group summaries. Tissues are colored by group (red=eyes, green=skin, blue=brain, purple=genitals).}
\label{fig:3_2}
\end{figure}

The average SAR mW kg$^{ 1}$ per tissue (Ella) can be found in Figure~\ref{fig:3_3}. \\\\

\begin{figure}[H]
\centering
\includegraphics[width=\columnwidth]{../../../plots/far_field/ella/heatmap/heatmap_avg_sar_summary.pdf}
\caption{The heatmap shows Average SAR values per tissue across frequencies for Ella. The top panel shows individual tissues, and the bottom panel shows organ group summaries. Tissues are colored by group (red=eyes, green=skin, blue=brain, purple=genitals).}
\label{fig:3_3}
\end{figure}

The average SAR mW kg$^{ 1}$ per tissue (Thelonious) can be found in Figure~\ref{fig:3_4}. \\\\

\begin{figure}[H]
\centering
\includegraphics[width=\columnwidth]{../../../plots/far_field/thelonious/heatmap/heatmap_avg_sar_summary.pdf}
\caption{The heatmap shows Average SAR values per tissue across frequencies for Thelonious. The top panel shows individual tissues, and the bottom panel shows organ group summaries. Tissues are colored by group (red=eyes, green=skin, blue=brain, purple=genitals).}
\label{fig:3_4}
\end{figure}

\subsection{Direction Polarization Peak SAR All MHz}

The far field Peak SAR 10g by incident direction and polarization averaged across all frequencies (Duke) can be found in Figure~\ref{fig:3_5}. \\\\

\begin{figure}[H]
\centering
\includegraphics[width=\columnwidth]{../../../plots/far_field/duke/heatmap/heatmap_direction_polarization_peak_sar_allMHz.pdf}
\caption{The heatmap compares normalized Peak SAR (10g) values for Duke across different far-field incident wave directions (from left, right, front, back, above, below) and polarizations (Theta, Phi) averaged across all frequencies. Red indicates higher SAR absorption, green indicates lower SAR absorption. This visualization helps identify which exposure configurations result in the highest and lowest SAR values.}
\label{fig:3_5}
\end{figure}

The far field Peak SAR 10g by incident direction and polarization averaged across all frequencies (Eartha) can be found in Figure~\ref{fig:3_6}. \\\\

\begin{figure}[H]
\centering
\includegraphics[width=\columnwidth]{../../../plots/far_field/eartha/heatmap/heatmap_direction_polarization_peak_sar_allMHz.pdf}
\caption{The heatmap compares normalized Peak SAR (10g) values for Eartha across different far-field incident wave directions (from left, right, front, back, above, below) and polarizations (Theta, Phi) averaged across all frequencies. Red indicates higher SAR absorption, green indicates lower SAR absorption. This visualization helps identify which exposure configurations result in the highest and lowest SAR values.}
\label{fig:3_6}
\end{figure}

The far field Peak SAR 10g by incident direction and polarization averaged across all frequencies (Ella) can be found in Figure~\ref{fig:3_7}. \\\\

\begin{figure}[H]
\centering
\includegraphics[width=\columnwidth]{../../../plots/far_field/ella/heatmap/heatmap_direction_polarization_peak_sar_allMHz.pdf}
\caption{The heatmap compares normalized Peak SAR (10g) values for Ella across different far-field incident wave directions (from left, right, front, back, above, below) and polarizations (Theta, Phi) averaged across all frequencies. Red indicates higher SAR absorption, green indicates lower SAR absorption. This visualization helps identify which exposure configurations result in the highest and lowest SAR values.}
\label{fig:3_7}
\end{figure}

The far field Peak SAR 10g by incident direction and polarization averaged across all frequencies (Thelonious) can be found in Figure~\ref{fig:3_8}. \\\\

\begin{figure}[H]
\centering
\includegraphics[width=\columnwidth]{../../../plots/far_field/thelonious/heatmap/heatmap_direction_polarization_peak_sar_allMHz.pdf}
\caption{The heatmap compares normalized Peak SAR (10g) values for Thelonious across different far-field incident wave directions (from left, right, front, back, above, below) and polarizations (Theta, Phi) averaged across all frequencies. Red indicates higher SAR absorption, green indicates lower SAR absorption. This visualization helps identify which exposure configurations result in the highest and lowest SAR values.}
\label{fig:3_8}
\end{figure}

\subsection{Direction Polarization SAR Brain All MHz}

The far field Brain SAR by incident direction and polarization averaged across all frequencies (Duke) can be found in Figure~\ref{fig:3_9}. \\\\

\begin{figure}[H]
\centering
\includegraphics[width=\columnwidth]{../../../plots/far_field/duke/heatmap/heatmap_direction_polarization_SAR_brain_allMHz.pdf}
\caption{The heatmap compares normalized Brain SAR values for Duke across different far-field incident wave directions (from left, right, front, back, above, below) and polarizations (Theta, Phi) averaged across all frequencies. Red indicates higher SAR absorption, green indicates lower SAR absorption. This visualization helps identify which exposure configurations result in the highest and lowest SAR values.}
\label{fig:3_9}
\end{figure}

The far field Brain SAR by incident direction and polarization averaged across all frequencies (Eartha) can be found in Figure~\ref{fig:3_10}. \\\\

\begin{figure}[H]
\centering
\includegraphics[width=\columnwidth]{../../../plots/far_field/eartha/heatmap/heatmap_direction_polarization_SAR_brain_allMHz.pdf}
\caption{The heatmap compares normalized Brain SAR values for Eartha across different far-field incident wave directions (from left, right, front, back, above, below) and polarizations (Theta, Phi) averaged across all frequencies. Red indicates higher SAR absorption, green indicates lower SAR absorption. This visualization helps identify which exposure configurations result in the highest and lowest SAR values.}
\label{fig:3_10}
\end{figure}

The far field Brain SAR by incident direction and polarization averaged across all frequencies (Ella) can be found in Figure~\ref{fig:3_11}. \\\\

\begin{figure}[H]
\centering
\includegraphics[width=\columnwidth]{../../../plots/far_field/ella/heatmap/heatmap_direction_polarization_SAR_brain_allMHz.pdf}
\caption{The heatmap compares normalized Brain SAR values for Ella across different far-field incident wave directions (from left, right, front, back, above, below) and polarizations (Theta, Phi) averaged across all frequencies. Red indicates higher SAR absorption, green indicates lower SAR absorption. This visualization helps identify which exposure configurations result in the highest and lowest SAR values.}
\label{fig:3_11}
\end{figure}

The far field Brain SAR by incident direction and polarization averaged across all frequencies (Thelonious) can be found in Figure~\ref{fig:3_12}. \\\\

\begin{figure}[H]
\centering
\includegraphics[width=\columnwidth]{../../../plots/far_field/thelonious/heatmap/heatmap_direction_polarization_SAR_brain_allMHz.pdf}
\caption{The heatmap compares normalized Brain SAR values for Thelonious across different far-field incident wave directions (from left, right, front, back, above, below) and polarizations (Theta, Phi) averaged across all frequencies. Red indicates higher SAR absorption, green indicates lower SAR absorption. This visualization helps identify which exposure configurations result in the highest and lowest SAR values.}
\label{fig:3_12}
\end{figure}

\subsection{Direction Polarization SAR Eyes All MHz}

The far field Eyes SAR by incident direction and polarization averaged across all frequencies (Duke) can be found in Figure~\ref{fig:3_13}. \\\\

\begin{figure}[H]
\centering
\includegraphics[width=\columnwidth]{../../../plots/far_field/duke/heatmap/heatmap_direction_polarization_SAR_eyes_allMHz.pdf}
\caption{The heatmap compares normalized Eyes SAR values for Duke across different far-field incident wave directions (from left, right, front, back, above, below) and polarizations (Theta, Phi) averaged across all frequencies. Red indicates higher SAR absorption, green indicates lower SAR absorption. This visualization helps identify which exposure configurations result in the highest and lowest SAR values.}
\label{fig:3_13}
\end{figure}

The far field Eyes SAR by incident direction and polarization averaged across all frequencies (Eartha) can be found in Figure~\ref{fig:3_14}. \\\\

\begin{figure}[H]
\centering
\includegraphics[width=\columnwidth]{../../../plots/far_field/eartha/heatmap/heatmap_direction_polarization_SAR_eyes_allMHz.pdf}
\caption{The heatmap compares normalized Eyes SAR values for Eartha across different far-field incident wave directions (from left, right, front, back, above, below) and polarizations (Theta, Phi) averaged across all frequencies. Red indicates higher SAR absorption, green indicates lower SAR absorption. This visualization helps identify which exposure configurations result in the highest and lowest SAR values.}
\label{fig:3_14}
\end{figure}

The far field Eyes SAR by incident direction and polarization averaged across all frequencies (Ella) can be found in Figure~\ref{fig:3_15}. \\\\

\begin{figure}[H]
\centering
\includegraphics[width=\columnwidth]{../../../plots/far_field/ella/heatmap/heatmap_direction_polarization_SAR_eyes_allMHz.pdf}
\caption{The heatmap compares normalized Eyes SAR values for Ella across different far-field incident wave directions (from left, right, front, back, above, below) and polarizations (Theta, Phi) averaged across all frequencies. Red indicates higher SAR absorption, green indicates lower SAR absorption. This visualization helps identify which exposure configurations result in the highest and lowest SAR values.}
\label{fig:3_15}
\end{figure}

The far field Eyes SAR by incident direction and polarization averaged across all frequencies (Thelonious) can be found in Figure~\ref{fig:3_16}. \\\\

\begin{figure}[H]
\centering
\includegraphics[width=\columnwidth]{../../../plots/far_field/thelonious/heatmap/heatmap_direction_polarization_SAR_eyes_allMHz.pdf}
\caption{The heatmap compares normalized Eyes SAR values for Thelonious across different far-field incident wave directions (from left, right, front, back, above, below) and polarizations (Theta, Phi) averaged across all frequencies. Red indicates higher SAR absorption, green indicates lower SAR absorption. This visualization helps identify which exposure configurations result in the highest and lowest SAR values.}
\label{fig:3_16}
\end{figure}

\subsection{Direction Polarization SAR Genitals All MHz}

The far field Genitals SAR by incident direction and polarization averaged across all frequencies (Duke) can be found in Figure~\ref{fig:3_17}. \\\\

\begin{figure}[H]
\centering
\includegraphics[width=\columnwidth]{../../../plots/far_field/duke/heatmap/heatmap_direction_polarization_SAR_genitals_allMHz.pdf}
\caption{The heatmap compares normalized Genitals SAR values for Duke across different far-field incident wave directions (from left, right, front, back, above, below) and polarizations (Theta, Phi) averaged across all frequencies. Red indicates higher SAR absorption, green indicates lower SAR absorption. This visualization helps identify which exposure configurations result in the highest and lowest SAR values.}
\label{fig:3_17}
\end{figure}

The far field Genitals SAR by incident direction and polarization averaged across all frequencies (Eartha) can be found in Figure~\ref{fig:3_18}. \\\\

\begin{figure}[H]
\centering
\includegraphics[width=\columnwidth]{../../../plots/far_field/eartha/heatmap/heatmap_direction_polarization_SAR_genitals_allMHz.pdf}
\caption{The heatmap compares normalized Genitals SAR values for Eartha across different far-field incident wave directions (from left, right, front, back, above, below) and polarizations (Theta, Phi) averaged across all frequencies. Red indicates higher SAR absorption, green indicates lower SAR absorption. This visualization helps identify which exposure configurations result in the highest and lowest SAR values.}
\label{fig:3_18}
\end{figure}

The far field Genitals SAR by incident direction and polarization averaged across all frequencies (Ella) can be found in Figure~\ref{fig:3_19}. \\\\

\begin{figure}[H]
\centering
\includegraphics[width=\columnwidth]{../../../plots/far_field/ella/heatmap/heatmap_direction_polarization_SAR_genitals_allMHz.pdf}
\caption{The heatmap compares normalized Genitals SAR values for Ella across different far-field incident wave directions (from left, right, front, back, above, below) and polarizations (Theta, Phi) averaged across all frequencies. Red indicates higher SAR absorption, green indicates lower SAR absorption. This visualization helps identify which exposure configurations result in the highest and lowest SAR values.}
\label{fig:3_19}
\end{figure}

The far field Genitals SAR by incident direction and polarization averaged across all frequencies (Thelonious) can be found in Figure~\ref{fig:3_20}. \\\\

\begin{figure}[H]
\centering
\includegraphics[width=\columnwidth]{../../../plots/far_field/thelonious/heatmap/heatmap_direction_polarization_SAR_genitals_allMHz.pdf}
\caption{The heatmap compares normalized Genitals SAR values for Thelonious across different far-field incident wave directions (from left, right, front, back, above, below) and polarizations (Theta, Phi) averaged across all frequencies. Red indicates higher SAR absorption, green indicates lower SAR absorption. This visualization helps identify which exposure configurations result in the highest and lowest SAR values.}
\label{fig:3_20}
\end{figure}

\subsection{Direction Polarization SAR Skin All MHz}

The far field Skin SAR by incident direction and polarization averaged across all frequencies (Duke) can be found in Figure~\ref{fig:3_21}. \\\\

\begin{figure}[H]
\centering
\includegraphics[width=\columnwidth]{../../../plots/far_field/duke/heatmap/heatmap_direction_polarization_SAR_skin_allMHz.pdf}
\caption{The heatmap compares normalized Skin SAR values for Duke across different far-field incident wave directions (from left, right, front, back, above, below) and polarizations (Theta, Phi) averaged across all frequencies. Red indicates higher SAR absorption, green indicates lower SAR absorption. This visualization helps identify which exposure configurations result in the highest and lowest SAR values.}
\label{fig:3_21}
\end{figure}

The far field Skin SAR by incident direction and polarization averaged across all frequencies (Eartha) can be found in Figure~\ref{fig:3_22}. \\\\

\begin{figure}[H]
\centering
\includegraphics[width=\columnwidth]{../../../plots/far_field/eartha/heatmap/heatmap_direction_polarization_SAR_skin_allMHz.pdf}
\caption{The heatmap compares normalized Skin SAR values for Eartha across different far-field incident wave directions (from left, right, front, back, above, below) and polarizations (Theta, Phi) averaged across all frequencies. Red indicates higher SAR absorption, green indicates lower SAR absorption. This visualization helps identify which exposure configurations result in the highest and lowest SAR values.}
\label{fig:3_22}
\end{figure}

The far field Skin SAR by incident direction and polarization averaged across all frequencies (Ella) can be found in Figure~\ref{fig:3_23}. \\\\

\begin{figure}[H]
\centering
\includegraphics[width=\columnwidth]{../../../plots/far_field/ella/heatmap/heatmap_direction_polarization_SAR_skin_allMHz.pdf}
\caption{The heatmap compares normalized Skin SAR values for Ella across different far-field incident wave directions (from left, right, front, back, above, below) and polarizations (Theta, Phi) averaged across all frequencies. Red indicates higher SAR absorption, green indicates lower SAR absorption. This visualization helps identify which exposure configurations result in the highest and lowest SAR values.}
\label{fig:3_23}
\end{figure}

The far field Skin SAR by incident direction and polarization averaged across all frequencies (Thelonious) can be found in Figure~\ref{fig:3_24}. \\\\

\begin{figure}[H]
\centering
\includegraphics[width=\columnwidth]{../../../plots/far_field/thelonious/heatmap/heatmap_direction_polarization_SAR_skin_allMHz.pdf}
\caption{The heatmap compares normalized Skin SAR values for Thelonious across different far-field incident wave directions (from left, right, front, back, above, below) and polarizations (Theta, Phi) averaged across all frequencies. Red indicates higher SAR absorption, green indicates lower SAR absorption. This visualization helps identify which exposure configurations result in the highest and lowest SAR values.}
\label{fig:3_24}
\end{figure}

\subsection{Direction Polarization SAR Whole Body All MHz}

The far field Whole Body SAR by incident direction and polarization averaged across all frequencies (Duke) can be found in Figure~\ref{fig:3_25}. \\\\

\begin{figure}[H]
\centering
\includegraphics[width=\columnwidth]{../../../plots/far_field/duke/heatmap/heatmap_direction_polarization_SAR_whole_body_allMHz.pdf}
\caption{The heatmap compares normalized Whole-Body SAR values for Duke across different far-field incident wave directions (from left, right, front, back, above, below) and polarizations (Theta, Phi) averaged across all frequencies. Red indicates higher SAR absorption, green indicates lower SAR absorption. This visualization helps identify which exposure configurations result in the highest and lowest SAR values.}
\label{fig:3_25}
\end{figure}

The far field Whole Body SAR by incident direction and polarization averaged across all frequencies (Eartha) can be found in Figure~\ref{fig:3_26}. \\\\

\begin{figure}[H]
\centering
\includegraphics[width=\columnwidth]{../../../plots/far_field/eartha/heatmap/heatmap_direction_polarization_SAR_whole_body_allMHz.pdf}
\caption{The heatmap compares normalized Whole-Body SAR values for Eartha across different far-field incident wave directions (from left, right, front, back, above, below) and polarizations (Theta, Phi) averaged across all frequencies. Red indicates higher SAR absorption, green indicates lower SAR absorption. This visualization helps identify which exposure configurations result in the highest and lowest SAR values.}
\label{fig:3_26}
\end{figure}

The far field Whole Body SAR by incident direction and polarization averaged across all frequencies (Ella) can be found in Figure~\ref{fig:3_27}. \\\\

\begin{figure}[H]
\centering
\includegraphics[width=\columnwidth]{../../../plots/far_field/ella/heatmap/heatmap_direction_polarization_SAR_whole_body_allMHz.pdf}
\caption{The heatmap compares normalized Whole-Body SAR values for Ella across different far-field incident wave directions (from left, right, front, back, above, below) and polarizations (Theta, Phi) averaged across all frequencies. Red indicates higher SAR absorption, green indicates lower SAR absorption. This visualization helps identify which exposure configurations result in the highest and lowest SAR values.}
\label{fig:3_27}
\end{figure}

The far field Whole Body SAR by incident direction and polarization averaged across all frequencies (Thelonious) can be found in Figure~\ref{fig:3_28}. \\\\

\begin{figure}[H]
\centering
\includegraphics[width=\columnwidth]{../../../plots/far_field/thelonious/heatmap/heatmap_direction_polarization_SAR_whole_body_allMHz.pdf}
\caption{The heatmap compares normalized Whole-Body SAR values for Thelonious across different far-field incident wave directions (from left, right, front, back, above, below) and polarizations (Theta, Phi) averaged across all frequencies. Red indicates higher SAR absorption, green indicates lower SAR absorption. This visualization helps identify which exposure configurations result in the highest and lowest SAR values.}
\label{fig:3_28}
\end{figure}

\subsection{Direction Polarization Summary}

The far field SAR comparison by incident direction and polarization (Duke) can be found in Figure~\ref{fig:3_29}. \\\\

\begin{figure*}[htbp]
\centering
\includegraphics[width=\textwidth]{../../../plots/far_field/duke/heatmap/heatmap_direction_polarization_summary.pdf}
\caption{The combined heatmap compares normalized SAR values for Duke across different far-field incident wave directions (from left/right, front/back, above/below) and polarizations (Theta, Phi). Each panel shows a different SAR metric. Red indicates higher SAR absorption, green indicates lower.}
\label{fig:3_29}
\end{figure*}

The far field SAR comparison by incident direction and polarization (Eartha) can be found in Figure~\ref{fig:3_30}. \\\\

\begin{figure*}[htbp]
\centering
\includegraphics[width=\textwidth]{../../../plots/far_field/eartha/heatmap/heatmap_direction_polarization_summary.pdf}
\caption{The combined heatmap compares normalized SAR values for Eartha across different far-field incident wave directions (from left/right, front/back, above/below) and polarizations (Theta, Phi). Each panel shows a different SAR metric. Red indicates higher SAR absorption, green indicates lower.}
\label{fig:3_30}
\end{figure*}

The far field SAR comparison by incident direction and polarization (Ella) can be found in Figure~\ref{fig:3_31}. \\\\

\begin{figure*}[htbp]
\centering
\includegraphics[width=\textwidth]{../../../plots/far_field/ella/heatmap/heatmap_direction_polarization_summary.pdf}
\caption{The combined heatmap compares normalized SAR values for Ella across different far-field incident wave directions (from left/right, front/back, above/below) and polarizations (Theta, Phi). Each panel shows a different SAR metric. Red indicates higher SAR absorption, green indicates lower.}
\label{fig:3_31}
\end{figure*}

The far field SAR comparison by incident direction and polarization (Thelonious) can be found in Figure~\ref{fig:3_32}. \\\\

\begin{figure*}[htbp]
\centering
\includegraphics[width=\textwidth]{../../../plots/far_field/thelonious/heatmap/heatmap_direction_polarization_summary.pdf}
\caption{The combined heatmap compares normalized SAR values for Thelonious across different far-field incident wave directions (from left/right, front/back, above/below) and polarizations (Theta, Phi). Each panel shows a different SAR metric. Red indicates higher SAR absorption, green indicates lower.}
\label{fig:3_32}
\end{figure*}

\subsection{Direction Polarization psSAR10g Brain All MHz}

The far field psSAR10g Brain by incident direction and polarization averaged across all frequencies (Duke) can be found in Figure~\ref{fig:3_33}. \\\\

\begin{figure}[H]
\centering
\includegraphics[width=\columnwidth]{../../../plots/far_field/duke/heatmap/heatmap_direction_polarization_psSAR10g_brain_allMHz.pdf}
\caption{The heatmap compares normalized psSAR10g Brain values for Duke across different far-field incident wave directions (from left, right, front, back, above, below) and polarizations (Theta, Phi) averaged across all frequencies. Red indicates higher SAR absorption, green indicates lower SAR absorption. This visualization helps identify which exposure configurations result in the highest and lowest SAR values.}
\label{fig:3_33}
\end{figure}

The far field psSAR10g Brain by incident direction and polarization averaged across all frequencies (Eartha) can be found in Figure~\ref{fig:3_34}. \\\\

\begin{figure}[H]
\centering
\includegraphics[width=\columnwidth]{../../../plots/far_field/eartha/heatmap/heatmap_direction_polarization_psSAR10g_brain_allMHz.pdf}
\caption{The heatmap compares normalized psSAR10g Brain values for Eartha across different far-field incident wave directions (from left, right, front, back, above, below) and polarizations (Theta, Phi) averaged across all frequencies. Red indicates higher SAR absorption, green indicates lower SAR absorption. This visualization helps identify which exposure configurations result in the highest and lowest SAR values.}
\label{fig:3_34}
\end{figure}

The far field psSAR10g Brain by incident direction and polarization averaged across all frequencies (Ella) can be found in Figure~\ref{fig:3_35}. \\\\

\begin{figure}[H]
\centering
\includegraphics[width=\columnwidth]{../../../plots/far_field/ella/heatmap/heatmap_direction_polarization_psSAR10g_brain_allMHz.pdf}
\caption{The heatmap compares normalized psSAR10g Brain values for Ella across different far-field incident wave directions (from left, right, front, back, above, below) and polarizations (Theta, Phi) averaged across all frequencies. Red indicates higher SAR absorption, green indicates lower SAR absorption. This visualization helps identify which exposure configurations result in the highest and lowest SAR values.}
\label{fig:3_35}
\end{figure}

The far field psSAR10g Brain by incident direction and polarization averaged across all frequencies (Thelonious) can be found in Figure~\ref{fig:3_36}. \\\\

\begin{figure}[H]
\centering
\includegraphics[width=\columnwidth]{../../../plots/far_field/thelonious/heatmap/heatmap_direction_polarization_psSAR10g_brain_allMHz.pdf}
\caption{The heatmap compares normalized psSAR10g Brain values for Thelonious across different far-field incident wave directions (from left, right, front, back, above, below) and polarizations (Theta, Phi) averaged across all frequencies. Red indicates higher SAR absorption, green indicates lower SAR absorption. This visualization helps identify which exposure configurations result in the highest and lowest SAR values.}
\label{fig:3_36}
\end{figure}

\subsection{Direction Polarization psSAR10g Eyes All MHz}

The far field psSAR10g Eyes by incident direction and polarization averaged across all frequencies (Duke) can be found in Figure~\ref{fig:3_37}. \\\\

\begin{figure}[H]
\centering
\includegraphics[width=\columnwidth]{../../../plots/far_field/duke/heatmap/heatmap_direction_polarization_psSAR10g_eyes_allMHz.pdf}
\caption{The heatmap compares normalized psSAR10g Eyes values for Duke across different far-field incident wave directions (from left, right, front, back, above, below) and polarizations (Theta, Phi) averaged across all frequencies. Red indicates higher SAR absorption, green indicates lower SAR absorption. This visualization helps identify which exposure configurations result in the highest and lowest SAR values.}
\label{fig:3_37}
\end{figure}

The far field psSAR10g Eyes by incident direction and polarization averaged across all frequencies (Eartha) can be found in Figure~\ref{fig:3_38}. \\\\

\begin{figure}[H]
\centering
\includegraphics[width=\columnwidth]{../../../plots/far_field/eartha/heatmap/heatmap_direction_polarization_psSAR10g_eyes_allMHz.pdf}
\caption{The heatmap compares normalized psSAR10g Eyes values for Eartha across different far-field incident wave directions (from left, right, front, back, above, below) and polarizations (Theta, Phi) averaged across all frequencies. Red indicates higher SAR absorption, green indicates lower SAR absorption. This visualization helps identify which exposure configurations result in the highest and lowest SAR values.}
\label{fig:3_38}
\end{figure}

The far field psSAR10g Eyes by incident direction and polarization averaged across all frequencies (Ella) can be found in Figure~\ref{fig:3_39}. \\\\

\begin{figure}[H]
\centering
\includegraphics[width=\columnwidth]{../../../plots/far_field/ella/heatmap/heatmap_direction_polarization_psSAR10g_eyes_allMHz.pdf}
\caption{The heatmap compares normalized psSAR10g Eyes values for Ella across different far-field incident wave directions (from left, right, front, back, above, below) and polarizations (Theta, Phi) averaged across all frequencies. Red indicates higher SAR absorption, green indicates lower SAR absorption. This visualization helps identify which exposure configurations result in the highest and lowest SAR values.}
\label{fig:3_39}
\end{figure}

The far field psSAR10g Eyes by incident direction and polarization averaged across all frequencies (Thelonious) can be found in Figure~\ref{fig:3_40}. \\\\

\begin{figure}[H]
\centering
\includegraphics[width=\columnwidth]{../../../plots/far_field/thelonious/heatmap/heatmap_direction_polarization_psSAR10g_eyes_allMHz.pdf}
\caption{The heatmap compares normalized psSAR10g Eyes values for Thelonious across different far-field incident wave directions (from left, right, front, back, above, below) and polarizations (Theta, Phi) averaged across all frequencies. Red indicates higher SAR absorption, green indicates lower SAR absorption. This visualization helps identify which exposure configurations result in the highest and lowest SAR values.}
\label{fig:3_40}
\end{figure}

\subsection{Direction Polarization psSAR10g Genitals All MHz}

The far field psSAR10g Genitals by incident direction and polarization averaged across all frequencies (Duke) can be found in Figure~\ref{fig:3_41}. \\\\

\begin{figure}[H]
\centering
\includegraphics[width=\columnwidth]{../../../plots/far_field/duke/heatmap/heatmap_direction_polarization_psSAR10g_genitals_allMHz.pdf}
\caption{The heatmap compares normalized psSAR10g Genitals values for Duke across different far-field incident wave directions (from left, right, front, back, above, below) and polarizations (Theta, Phi) averaged across all frequencies. Red indicates higher SAR absorption, green indicates lower SAR absorption. This visualization helps identify which exposure configurations result in the highest and lowest SAR values.}
\label{fig:3_41}
\end{figure}

The far field psSAR10g Genitals by incident direction and polarization averaged across all frequencies (Eartha) can be found in Figure~\ref{fig:3_42}. \\\\

\begin{figure}[H]
\centering
\includegraphics[width=\columnwidth]{../../../plots/far_field/eartha/heatmap/heatmap_direction_polarization_psSAR10g_genitals_allMHz.pdf}
\caption{The heatmap compares normalized psSAR10g Genitals values for Eartha across different far-field incident wave directions (from left, right, front, back, above, below) and polarizations (Theta, Phi) averaged across all frequencies. Red indicates higher SAR absorption, green indicates lower SAR absorption. This visualization helps identify which exposure configurations result in the highest and lowest SAR values.}
\label{fig:3_42}
\end{figure}

The far field psSAR10g Genitals by incident direction and polarization averaged across all frequencies (Ella) can be found in Figure~\ref{fig:3_43}. \\\\

\begin{figure}[H]
\centering
\includegraphics[width=\columnwidth]{../../../plots/far_field/ella/heatmap/heatmap_direction_polarization_psSAR10g_genitals_allMHz.pdf}
\caption{The heatmap compares normalized psSAR10g Genitals values for Ella across different far-field incident wave directions (from left, right, front, back, above, below) and polarizations (Theta, Phi) averaged across all frequencies. Red indicates higher SAR absorption, green indicates lower SAR absorption. This visualization helps identify which exposure configurations result in the highest and lowest SAR values.}
\label{fig:3_43}
\end{figure}

The far field psSAR10g Genitals by incident direction and polarization averaged across all frequencies (Thelonious) can be found in Figure~\ref{fig:3_44}. \\\\

\begin{figure}[H]
\centering
\includegraphics[width=\columnwidth]{../../../plots/far_field/thelonious/heatmap/heatmap_direction_polarization_psSAR10g_genitals_allMHz.pdf}
\caption{The heatmap compares normalized psSAR10g Genitals values for Thelonious across different far-field incident wave directions (from left, right, front, back, above, below) and polarizations (Theta, Phi) averaged across all frequencies. Red indicates higher SAR absorption, green indicates lower SAR absorption. This visualization helps identify which exposure configurations result in the highest and lowest SAR values.}
\label{fig:3_44}
\end{figure}

\subsection{Direction Polarization psSAR10g Skin All MHz}

The far field psSAR10g Skin by incident direction and polarization averaged across all frequencies (Duke) can be found in Figure~\ref{fig:3_45}. \\\\

\begin{figure}[H]
\centering
\includegraphics[width=\columnwidth]{../../../plots/far_field/duke/heatmap/heatmap_direction_polarization_psSAR10g_skin_allMHz.pdf}
\caption{The heatmap compares normalized psSAR10g Skin values for Duke across different far-field incident wave directions (from left, right, front, back, above, below) and polarizations (Theta, Phi) averaged across all frequencies. Red indicates higher SAR absorption, green indicates lower SAR absorption. This visualization helps identify which exposure configurations result in the highest and lowest SAR values.}
\label{fig:3_45}
\end{figure}

The far field psSAR10g Skin by incident direction and polarization averaged across all frequencies (Eartha) can be found in Figure~\ref{fig:3_46}. \\\\

\begin{figure}[H]
\centering
\includegraphics[width=\columnwidth]{../../../plots/far_field/eartha/heatmap/heatmap_direction_polarization_psSAR10g_skin_allMHz.pdf}
\caption{The heatmap compares normalized psSAR10g Skin values for Eartha across different far-field incident wave directions (from left, right, front, back, above, below) and polarizations (Theta, Phi) averaged across all frequencies. Red indicates higher SAR absorption, green indicates lower SAR absorption. This visualization helps identify which exposure configurations result in the highest and lowest SAR values.}
\label{fig:3_46}
\end{figure}

The far field psSAR10g Skin by incident direction and polarization averaged across all frequencies (Ella) can be found in Figure~\ref{fig:3_47}. \\\\

\begin{figure}[H]
\centering
\includegraphics[width=\columnwidth]{../../../plots/far_field/ella/heatmap/heatmap_direction_polarization_psSAR10g_skin_allMHz.pdf}
\caption{The heatmap compares normalized psSAR10g Skin values for Ella across different far-field incident wave directions (from left, right, front, back, above, below) and polarizations (Theta, Phi) averaged across all frequencies. Red indicates higher SAR absorption, green indicates lower SAR absorption. This visualization helps identify which exposure configurations result in the highest and lowest SAR values.}
\label{fig:3_47}
\end{figure}

The far field psSAR10g Skin by incident direction and polarization averaged across all frequencies (Thelonious) can be found in Figure~\ref{fig:3_48}. \\\\

\begin{figure}[H]
\centering
\includegraphics[width=\columnwidth]{../../../plots/far_field/thelonious/heatmap/heatmap_direction_polarization_psSAR10g_skin_allMHz.pdf}
\caption{The heatmap compares normalized psSAR10g Skin values for Thelonious across different far-field incident wave directions (from left, right, front, back, above, below) and polarizations (Theta, Phi) averaged across all frequencies. Red indicates higher SAR absorption, green indicates lower SAR absorption. This visualization helps identify which exposure configurations result in the highest and lowest SAR values.}
\label{fig:3_48}
\end{figure}

\subsection{Peak SAR 10g mW kg Summary}

The peak SAR 10g mW kg$^{ 1}$ per tissue (Duke) can be found in Figure~\ref{fig:3_49}. \\\\

\begin{figure}[H]
\centering
\includegraphics[width=\columnwidth]{../../../plots/far_field/duke/heatmap/heatmap_peak_sar_10g_mw_kg_summary.pdf}
\caption{The heatmap shows Peak SAR 10g values per tissue across frequencies for Duke. The top panel shows individual tissues, and the bottom panel shows organ group summaries. Tissues are colored by group (red=eyes, green=skin, blue=brain, purple=genitals).}
\label{fig:3_49}
\end{figure}

The peak SAR 10g mW kg$^{ 1}$ per tissue (Eartha) can be found in Figure~\ref{fig:3_50}. \\\\

\begin{figure}[H]
\centering
\includegraphics[width=\columnwidth]{../../../plots/far_field/eartha/heatmap/heatmap_peak_sar_10g_mw_kg_summary.pdf}
\caption{The heatmap shows Peak SAR 10g values per tissue across frequencies for Eartha. The top panel shows individual tissues, and the bottom panel shows organ group summaries. Tissues are colored by group (red=eyes, green=skin, blue=brain, purple=genitals).}
\label{fig:3_50}
\end{figure}

The peak SAR 10g mW kg$^{ 1}$ per tissue (Ella) can be found in Figure~\ref{fig:3_51}. \\\\

\begin{figure}[H]
\centering
\includegraphics[width=\columnwidth]{../../../plots/far_field/ella/heatmap/heatmap_peak_sar_10g_mw_kg_summary.pdf}
\caption{The heatmap shows Peak SAR 10g values per tissue across frequencies for Ella. The top panel shows individual tissues, and the bottom panel shows organ group summaries. Tissues are colored by group (red=eyes, green=skin, blue=brain, purple=genitals).}
\label{fig:3_51}
\end{figure}

The peak SAR 10g mW kg$^{ 1}$ per tissue (Thelonious) can be found in Figure~\ref{fig:3_52}. \\\\

\begin{figure}[H]
\centering
\includegraphics[width=\columnwidth]{../../../plots/far_field/thelonious/heatmap/heatmap_peak_sar_10g_mw_kg_summary.pdf}
\caption{The heatmap shows Peak SAR 10g values per tissue across frequencies for Thelonious. The top panel shows individual tissues, and the bottom panel shows organ group summaries. Tissues are colored by group (red=eyes, green=skin, blue=brain, purple=genitals).}
\label{fig:3_52}
\end{figure}

\subsection{Polarization Ratio}

The polarization Ratio Theta/Phi by Direction (Duke) can be found in Figure~\ref{fig:3_53}. \\\\

\begin{figure}[H]
\centering
\includegraphics[width=\columnwidth]{../../../plots/far_field/duke/heatmap/heatmap_polarization_ratio.pdf}
\caption{Heatmap showing the theta/phi polarization ratio for Duke, averaged across all frequencies. Ratio > 1.0 (red) indicates theta polarization gives higher SAR. Ratio < 1.0 (blue) indicates phi polarization dominates. Note: These are frequency-averaged values; significant frequency-dependent variations exist.}
\label{fig:3_53}
\end{figure}

The polarization Ratio Theta/Phi by Direction (Eartha) can be found in Figure~\ref{fig:3_54}. \\\\

\begin{figure}[H]
\centering
\includegraphics[width=\columnwidth]{../../../plots/far_field/eartha/heatmap/heatmap_polarization_ratio.pdf}
\caption{Heatmap showing the theta/phi polarization ratio for Eartha, averaged across all frequencies. Ratio > 1.0 (red) indicates theta polarization gives higher SAR. Ratio < 1.0 (blue) indicates phi polarization dominates. Note: These are frequency-averaged values; significant frequency-dependent variations exist.}
\label{fig:3_54}
\end{figure}

The polarization Ratio Theta/Phi by Direction (Ella) can be found in Figure~\ref{fig:3_55}. \\\\

\begin{figure}[H]
\centering
\includegraphics[width=\columnwidth]{../../../plots/far_field/ella/heatmap/heatmap_polarization_ratio.pdf}
\caption{Heatmap showing the theta/phi polarization ratio for Ella, averaged across all frequencies. Ratio > 1.0 (red) indicates theta polarization gives higher SAR. Ratio < 1.0 (blue) indicates phi polarization dominates. Note: These are frequency-averaged values; significant frequency-dependent variations exist.}
\label{fig:3_55}
\end{figure}

The polarization Ratio Theta/Phi by Direction (Thelonious) can be found in Figure~\ref{fig:3_56}. \\\\

\begin{figure}[H]
\centering
\includegraphics[width=\columnwidth]{../../../plots/far_field/thelonious/heatmap/heatmap_polarization_ratio.pdf}
\caption{Heatmap showing the theta/phi polarization ratio for Thelonious, averaged across all frequencies. Ratio > 1.0 (red) indicates theta polarization gives higher SAR. Ratio < 1.0 (blue) indicates phi polarization dominates. Note: These are frequency-averaged values; significant frequency-dependent variations exist.}
\label{fig:3_56}
\end{figure}

\subsection{Polarization Ratio 1450 MHz}

heatmap polarization Ratio 1450Mhz can be found in Figure~\ref{fig:3_57}. \\\\

\begin{figure}[H]
\centering
\includegraphics[width=\columnwidth]{../../../plots/far_field/duke/heatmap/heatmap_polarization_ratio_1450MHz.pdf}
\caption{Heatmap Polarization Ratio 1450Mhz}
\label{fig:3_57}
\end{figure}

heatmap polarization Ratio 1450Mhz can be found in Figure~\ref{fig:3_58}. \\\\

\begin{figure}[H]
\centering
\includegraphics[width=\columnwidth]{../../../plots/far_field/eartha/heatmap/heatmap_polarization_ratio_1450MHz.pdf}
\caption{Heatmap Polarization Ratio 1450Mhz}
\label{fig:3_58}
\end{figure}

heatmap polarization Ratio 1450Mhz can be found in Figure~\ref{fig:3_59}. \\\\

\begin{figure}[H]
\centering
\includegraphics[width=\columnwidth]{../../../plots/far_field/ella/heatmap/heatmap_polarization_ratio_1450MHz.pdf}
\caption{Heatmap Polarization Ratio 1450Mhz}
\label{fig:3_59}
\end{figure}

heatmap polarization Ratio 1450Mhz can be found in Figure~\ref{fig:3_60}. \\\\

\begin{figure}[H]
\centering
\includegraphics[width=\columnwidth]{../../../plots/far_field/thelonious/heatmap/heatmap_polarization_ratio_1450MHz.pdf}
\caption{Heatmap Polarization Ratio 1450Mhz}
\label{fig:3_60}
\end{figure}

\subsection{Polarization Ratio 2140 MHz}

heatmap polarization Ratio 2140Mhz can be found in Figure~\ref{fig:3_61}. \\\\

\begin{figure}[H]
\centering
\includegraphics[width=\columnwidth]{../../../plots/far_field/duke/heatmap/heatmap_polarization_ratio_2140MHz.pdf}
\caption{Heatmap Polarization Ratio 2140Mhz}
\label{fig:3_61}
\end{figure}

heatmap polarization Ratio 2140Mhz can be found in Figure~\ref{fig:3_62}. \\\\

\begin{figure}[H]
\centering
\includegraphics[width=\columnwidth]{../../../plots/far_field/eartha/heatmap/heatmap_polarization_ratio_2140MHz.pdf}
\caption{Heatmap Polarization Ratio 2140Mhz}
\label{fig:3_62}
\end{figure}

heatmap polarization Ratio 2140Mhz can be found in Figure~\ref{fig:3_63}. \\\\

\begin{figure}[H]
\centering
\includegraphics[width=\columnwidth]{../../../plots/far_field/ella/heatmap/heatmap_polarization_ratio_2140MHz.pdf}
\caption{Heatmap Polarization Ratio 2140Mhz}
\label{fig:3_63}
\end{figure}

heatmap polarization Ratio 2140Mhz can be found in Figure~\ref{fig:3_64}. \\\\

\begin{figure}[H]
\centering
\includegraphics[width=\columnwidth]{../../../plots/far_field/thelonious/heatmap/heatmap_polarization_ratio_2140MHz.pdf}
\caption{Heatmap Polarization Ratio 2140Mhz}
\label{fig:3_64}
\end{figure}

\subsection{Polarization Ratio 2450 MHz}

heatmap polarization Ratio 2450Mhz can be found in Figure~\ref{fig:3_65}. \\\\

\begin{figure}[H]
\centering
\includegraphics[width=\columnwidth]{../../../plots/far_field/duke/heatmap/heatmap_polarization_ratio_2450MHz.pdf}
\caption{Heatmap Polarization Ratio 2450Mhz}
\label{fig:3_65}
\end{figure}

heatmap polarization Ratio 2450Mhz can be found in Figure~\ref{fig:3_66}. \\\\

\begin{figure}[H]
\centering
\includegraphics[width=\columnwidth]{../../../plots/far_field/eartha/heatmap/heatmap_polarization_ratio_2450MHz.pdf}
\caption{Heatmap Polarization Ratio 2450Mhz}
\label{fig:3_66}
\end{figure}

heatmap polarization Ratio 2450Mhz can be found in Figure~\ref{fig:3_67}. \\\\

\begin{figure}[H]
\centering
\includegraphics[width=\columnwidth]{../../../plots/far_field/ella/heatmap/heatmap_polarization_ratio_2450MHz.pdf}
\caption{Heatmap Polarization Ratio 2450Mhz}
\label{fig:3_67}
\end{figure}

heatmap polarization Ratio 2450Mhz can be found in Figure~\ref{fig:3_68}. \\\\

\begin{figure}[H]
\centering
\includegraphics[width=\columnwidth]{../../../plots/far_field/thelonious/heatmap/heatmap_polarization_ratio_2450MHz.pdf}
\caption{Heatmap Polarization Ratio 2450Mhz}
\label{fig:3_68}
\end{figure}

\subsection{Polarization Ratio 3500 MHz}

heatmap polarization Ratio 3500Mhz can be found in Figure~\ref{fig:3_69}. \\\\

\begin{figure}[H]
\centering
\includegraphics[width=\columnwidth]{../../../plots/far_field/duke/heatmap/heatmap_polarization_ratio_3500MHz.pdf}
\caption{Heatmap Polarization Ratio 3500Mhz}
\label{fig:3_69}
\end{figure}

heatmap polarization Ratio 3500Mhz can be found in Figure~\ref{fig:3_70}. \\\\

\begin{figure}[H]
\centering
\includegraphics[width=\columnwidth]{../../../plots/far_field/eartha/heatmap/heatmap_polarization_ratio_3500MHz.pdf}
\caption{Heatmap Polarization Ratio 3500Mhz}
\label{fig:3_70}
\end{figure}

heatmap polarization Ratio 3500Mhz can be found in Figure~\ref{fig:3_71}. \\\\

\begin{figure}[H]
\centering
\includegraphics[width=\columnwidth]{../../../plots/far_field/ella/heatmap/heatmap_polarization_ratio_3500MHz.pdf}
\caption{Heatmap Polarization Ratio 3500Mhz}
\label{fig:3_71}
\end{figure}

heatmap polarization Ratio 3500Mhz can be found in Figure~\ref{fig:3_72}. \\\\

\begin{figure}[H]
\centering
\includegraphics[width=\columnwidth]{../../../plots/far_field/thelonious/heatmap/heatmap_polarization_ratio_3500MHz.pdf}
\caption{Heatmap Polarization Ratio 3500Mhz}
\label{fig:3_72}
\end{figure}

\subsection{Polarization Ratio 450 MHz}

heatmap polarization Ratio 450Mhz can be found in Figure~\ref{fig:3_73}. \\\\

\begin{figure}[H]
\centering
\includegraphics[width=\columnwidth]{../../../plots/far_field/duke/heatmap/heatmap_polarization_ratio_450MHz.pdf}
\caption{Heatmap Polarization Ratio 450Mhz}
\label{fig:3_73}
\end{figure}

heatmap polarization Ratio 450Mhz can be found in Figure~\ref{fig:3_74}. \\\\

\begin{figure}[H]
\centering
\includegraphics[width=\columnwidth]{../../../plots/far_field/eartha/heatmap/heatmap_polarization_ratio_450MHz.pdf}
\caption{Heatmap Polarization Ratio 450Mhz}
\label{fig:3_74}
\end{figure}

heatmap polarization Ratio 450Mhz can be found in Figure~\ref{fig:3_75}. \\\\

\begin{figure}[H]
\centering
\includegraphics[width=\columnwidth]{../../../plots/far_field/ella/heatmap/heatmap_polarization_ratio_450MHz.pdf}
\caption{Heatmap Polarization Ratio 450Mhz}
\label{fig:3_75}
\end{figure}

heatmap polarization Ratio 450Mhz can be found in Figure~\ref{fig:3_76}. \\\\

\begin{figure}[H]
\centering
\includegraphics[width=\columnwidth]{../../../plots/far_field/thelonious/heatmap/heatmap_polarization_ratio_450MHz.pdf}
\caption{Heatmap Polarization Ratio 450Mhz}
\label{fig:3_76}
\end{figure}

\subsection{Polarization Ratio 5200 MHz}

heatmap polarization Ratio 5200Mhz can be found in Figure~\ref{fig:3_77}. \\\\

\begin{figure}[H]
\centering
\includegraphics[width=\columnwidth]{../../../plots/far_field/duke/heatmap/heatmap_polarization_ratio_5200MHz.pdf}
\caption{Heatmap Polarization Ratio 5200Mhz}
\label{fig:3_77}
\end{figure}

heatmap polarization Ratio 5200Mhz can be found in Figure~\ref{fig:3_78}. \\\\

\begin{figure}[H]
\centering
\includegraphics[width=\columnwidth]{../../../plots/far_field/eartha/heatmap/heatmap_polarization_ratio_5200MHz.pdf}
\caption{Heatmap Polarization Ratio 5200Mhz}
\label{fig:3_78}
\end{figure}

heatmap polarization Ratio 5200Mhz can be found in Figure~\ref{fig:3_79}. \\\\

\begin{figure}[H]
\centering
\includegraphics[width=\columnwidth]{../../../plots/far_field/ella/heatmap/heatmap_polarization_ratio_5200MHz.pdf}
\caption{Heatmap Polarization Ratio 5200Mhz}
\label{fig:3_79}
\end{figure}

heatmap polarization Ratio 5200Mhz can be found in Figure~\ref{fig:3_80}. \\\\

\begin{figure}[H]
\centering
\includegraphics[width=\columnwidth]{../../../plots/far_field/thelonious/heatmap/heatmap_polarization_ratio_5200MHz.pdf}
\caption{Heatmap Polarization Ratio 5200Mhz}
\label{fig:3_80}
\end{figure}

\subsection{Polarization Ratio 5800 MHz}

heatmap polarization Ratio 5800Mhz can be found in Figure~\ref{fig:3_81}. \\\\

\begin{figure}[H]
\centering
\includegraphics[width=\columnwidth]{../../../plots/far_field/duke/heatmap/heatmap_polarization_ratio_5800MHz.pdf}
\caption{Heatmap Polarization Ratio 5800Mhz}
\label{fig:3_81}
\end{figure}

heatmap polarization Ratio 5800Mhz can be found in Figure~\ref{fig:3_82}. \\\\

\begin{figure}[H]
\centering
\includegraphics[width=\columnwidth]{../../../plots/far_field/eartha/heatmap/heatmap_polarization_ratio_5800MHz.pdf}
\caption{Heatmap Polarization Ratio 5800Mhz}
\label{fig:3_82}
\end{figure}

heatmap polarization Ratio 5800Mhz can be found in Figure~\ref{fig:3_83}. \\\\

\begin{figure}[H]
\centering
\includegraphics[width=\columnwidth]{../../../plots/far_field/ella/heatmap/heatmap_polarization_ratio_5800MHz.pdf}
\caption{Heatmap Polarization Ratio 5800Mhz}
\label{fig:3_83}
\end{figure}

heatmap polarization Ratio 5800Mhz can be found in Figure~\ref{fig:3_84}. \\\\

\begin{figure}[H]
\centering
\includegraphics[width=\columnwidth]{../../../plots/far_field/thelonious/heatmap/heatmap_polarization_ratio_5800MHz.pdf}
\caption{Heatmap Polarization Ratio 5800Mhz}
\label{fig:3_84}
\end{figure}

\subsection{Polarization Ratio 700 MHz}

heatmap polarization Ratio 700Mhz can be found in Figure~\ref{fig:3_85}. \\\\

\begin{figure}[H]
\centering
\includegraphics[width=\columnwidth]{../../../plots/far_field/duke/heatmap/heatmap_polarization_ratio_700MHz.pdf}
\caption{Heatmap Polarization Ratio 700Mhz}
\label{fig:3_85}
\end{figure}

heatmap polarization Ratio 700Mhz can be found in Figure~\ref{fig:3_86}. \\\\

\begin{figure}[H]
\centering
\includegraphics[width=\columnwidth]{../../../plots/far_field/eartha/heatmap/heatmap_polarization_ratio_700MHz.pdf}
\caption{Heatmap Polarization Ratio 700Mhz}
\label{fig:3_86}
\end{figure}

heatmap polarization Ratio 700Mhz can be found in Figure~\ref{fig:3_87}. \\\\

\begin{figure}[H]
\centering
\includegraphics[width=\columnwidth]{../../../plots/far_field/ella/heatmap/heatmap_polarization_ratio_700MHz.pdf}
\caption{Heatmap Polarization Ratio 700Mhz}
\label{fig:3_87}
\end{figure}

heatmap polarization Ratio 700Mhz can be found in Figure~\ref{fig:3_88}. \\\\

\begin{figure}[H]
\centering
\includegraphics[width=\columnwidth]{../../../plots/far_field/thelonious/heatmap/heatmap_polarization_ratio_700MHz.pdf}
\caption{Heatmap Polarization Ratio 700Mhz}
\label{fig:3_88}
\end{figure}

\subsection{Polarization Ratio 835 MHz}

heatmap polarization Ratio 835Mhz can be found in Figure~\ref{fig:3_89}. \\\\

\begin{figure}[H]
\centering
\includegraphics[width=\columnwidth]{../../../plots/far_field/duke/heatmap/heatmap_polarization_ratio_835MHz.pdf}
\caption{Heatmap Polarization Ratio 835Mhz}
\label{fig:3_89}
\end{figure}

heatmap polarization Ratio 835Mhz can be found in Figure~\ref{fig:3_90}. \\\\

\begin{figure}[H]
\centering
\includegraphics[width=\columnwidth]{../../../plots/far_field/eartha/heatmap/heatmap_polarization_ratio_835MHz.pdf}
\caption{Heatmap Polarization Ratio 835Mhz}
\label{fig:3_90}
\end{figure}

heatmap polarization Ratio 835Mhz can be found in Figure~\ref{fig:3_91}. \\\\

\begin{figure}[H]
\centering
\includegraphics[width=\columnwidth]{../../../plots/far_field/ella/heatmap/heatmap_polarization_ratio_835MHz.pdf}
\caption{Heatmap Polarization Ratio 835Mhz}
\label{fig:3_91}
\end{figure}

heatmap polarization Ratio 835Mhz can be found in Figure~\ref{fig:3_92}. \\\\

\begin{figure}[H]
\centering
\includegraphics[width=\columnwidth]{../../../plots/far_field/thelonious/heatmap/heatmap_polarization_ratio_835MHz.pdf}
\caption{Heatmap Polarization Ratio 835Mhz}
\label{fig:3_92}
\end{figure}

\subsection{SAR Avg}

The avg SAR mW kg$^{ 1}$ per tissue (Duke) can be found in Figure~\ref{fig:3_93}. \\\\

\begin{figure}[H]
\centering
\includegraphics[width=\columnwidth]{../../../plots/far_field/duke/heatmap/heatmap_sar_avg.pdf}
\caption{The interactive heatmap shows avg SAR values per tissue across frequencies for Duke. The top panel shows individual tissues, and the bottom panel shows organ group summaries.}
\label{fig:3_93}
\end{figure}

The avg SAR mW kg$^{ 1}$ per tissue (Eartha) can be found in Figure~\ref{fig:3_94}. \\\\

\begin{figure}[H]
\centering
\includegraphics[width=\columnwidth]{../../../plots/far_field/eartha/heatmap/heatmap_sar_avg.pdf}
\caption{The interactive heatmap shows avg SAR values per tissue across frequencies for Eartha. The top panel shows individual tissues, and the bottom panel shows organ group summaries.}
\label{fig:3_94}
\end{figure}

The avg SAR mW kg$^{ 1}$ per tissue (Ella) can be found in Figure~\ref{fig:3_95}. \\\\

\begin{figure}[H]
\centering
\includegraphics[width=\columnwidth]{../../../plots/far_field/ella/heatmap/heatmap_sar_avg.pdf}
\caption{The interactive heatmap shows avg SAR values per tissue across frequencies for Ella. The top panel shows individual tissues, and the bottom panel shows organ group summaries.}
\label{fig:3_95}
\end{figure}

The avg SAR mW kg$^{ 1}$ per tissue (Thelonious) can be found in Figure~\ref{fig:3_96}. \\\\

\begin{figure}[H]
\centering
\includegraphics[width=\columnwidth]{../../../plots/far_field/thelonious/heatmap/heatmap_sar_avg.pdf}
\caption{The interactive heatmap shows avg SAR values per tissue across frequencies for Thelonious. The top panel shows individual tissues, and the bottom panel shows organ group summaries.}
\label{fig:3_96}
\end{figure}

\subsection{SAR Max}

The max SAR mW kg$^{ 1}$ per tissue (Duke) can be found in Figure~\ref{fig:3_97}. \\\\

\begin{figure}[H]
\centering
\includegraphics[width=\columnwidth]{../../../plots/far_field/duke/heatmap/heatmap_sar_max.pdf}
\caption{The interactive heatmap shows max SAR values per tissue across frequencies for Duke. The top panel shows individual tissues, and the bottom panel shows organ group summaries.}
\label{fig:3_97}
\end{figure}

The max SAR mW kg$^{ 1}$ per tissue (Eartha) can be found in Figure~\ref{fig:3_98}. \\\\

\begin{figure}[H]
\centering
\includegraphics[width=\columnwidth]{../../../plots/far_field/eartha/heatmap/heatmap_sar_max.pdf}
\caption{The interactive heatmap shows max SAR values per tissue across frequencies for Eartha. The top panel shows individual tissues, and the bottom panel shows organ group summaries.}
\label{fig:3_98}
\end{figure}

The max SAR mW kg$^{ 1}$ per tissue (Ella) can be found in Figure~\ref{fig:3_99}. \\\\

\begin{figure}[H]
\centering
\includegraphics[width=\columnwidth]{../../../plots/far_field/ella/heatmap/heatmap_sar_max.pdf}
\caption{The interactive heatmap shows max SAR values per tissue across frequencies for Ella. The top panel shows individual tissues, and the bottom panel shows organ group summaries.}
\label{fig:3_99}
\end{figure}

The max SAR mW kg$^{ 1}$ per tissue (Thelonious) can be found in Figure~\ref{fig:3_100}. \\\\

\begin{figure}[H]
\centering
\includegraphics[width=\columnwidth]{../../../plots/far_field/thelonious/heatmap/heatmap_sar_max.pdf}
\caption{The interactive heatmap shows max SAR values per tissue across frequencies for Thelonious. The top panel shows individual tissues, and the bottom panel shows organ group summaries.}
\label{fig:3_100}
\end{figure}

\subsection{SAR Min}

The min SAR mW kg$^{ 1}$ per tissue (Duke) can be found in Figure~\ref{fig:3_101}. \\\\

\begin{figure}[H]
\centering
\includegraphics[width=\columnwidth]{../../../plots/far_field/duke/heatmap/heatmap_sar_min.pdf}
\caption{The interactive heatmap shows min SAR values per tissue across frequencies for Duke. The top panel shows individual tissues, and the bottom panel shows organ group summaries.}
\label{fig:3_101}
\end{figure}

The min SAR mW kg$^{ 1}$ per tissue (Eartha) can be found in Figure~\ref{fig:3_102}. \\\\

\begin{figure}[H]
\centering
\includegraphics[width=\columnwidth]{../../../plots/far_field/eartha/heatmap/heatmap_sar_min.pdf}
\caption{The interactive heatmap shows min SAR values per tissue across frequencies for Eartha. The top panel shows individual tissues, and the bottom panel shows organ group summaries.}
\label{fig:3_102}
\end{figure}

The min SAR mW kg$^{ 1}$ per tissue (Ella) can be found in Figure~\ref{fig:3_103}. \\\\

\begin{figure}[H]
\centering
\includegraphics[width=\columnwidth]{../../../plots/far_field/ella/heatmap/heatmap_sar_min.pdf}
\caption{The interactive heatmap shows min SAR values per tissue across frequencies for Ella. The top panel shows individual tissues, and the bottom panel shows organ group summaries.}
\label{fig:3_103}
\end{figure}

The min SAR mW kg$^{ 1}$ per tissue (Thelonious) can be found in Figure~\ref{fig:3_104}. \\\\

\begin{figure}[H]
\centering
\includegraphics[width=\columnwidth]{../../../plots/far_field/thelonious/heatmap/heatmap_sar_min.pdf}
\caption{The interactive heatmap shows min SAR values per tissue across frequencies for Thelonious. The top panel shows individual tissues, and the bottom panel shows organ group summaries.}
\label{fig:3_104}
\end{figure}

\newpage

\section{Bubble Plots}

\subsection{Mass vs Mass Avg SAR All 1450 MHz Log}

The tissue Mass vs SAR Bubble Plot Log Scale (Duke) can be found in Figure~\ref{fig:4_1}. \\\\

\begin{figure}[H]
\centering
\includegraphics[width=\columnwidth]{../../../plots/far_field/duke/bubble/bubble_mass_vs_mass_avg_sar_all_1450MHz_log.pdf}
\caption{The bubble plot shows tissue mass vs Mass Avg SAR with bubble size proportional to tissue volume (log scale on both axes) for the all scenarios scenario at 1450 MHz for Duke.}
\label{fig:4_1}
\end{figure}

The tissue Mass vs SAR Bubble Plot Log Scale (Eartha) can be found in Figure~\ref{fig:4_2}. \\\\

\begin{figure}[H]
\centering
\includegraphics[width=\columnwidth]{../../../plots/far_field/eartha/bubble/bubble_mass_vs_mass_avg_sar_all_1450MHz_log.pdf}
\caption{The bubble plot shows tissue mass vs Mass Avg SAR with bubble size proportional to tissue volume (log scale on both axes) for the all scenarios scenario at 1450 MHz for Eartha.}
\label{fig:4_2}
\end{figure}

The tissue Mass vs SAR Bubble Plot Log Scale (Ella) can be found in Figure~\ref{fig:4_3}. \\\\

\begin{figure}[H]
\centering
\includegraphics[width=\columnwidth]{../../../plots/far_field/ella/bubble/bubble_mass_vs_mass_avg_sar_all_1450MHz_log.pdf}
\caption{The bubble plot shows tissue mass vs Mass Avg SAR with bubble size proportional to tissue volume (log scale on both axes) for the all scenarios scenario at 1450 MHz for Ella.}
\label{fig:4_3}
\end{figure}

The tissue Mass vs SAR Bubble Plot Log Scale (Thelonious) can be found in Figure~\ref{fig:4_4}. \\\\

\begin{figure}[H]
\centering
\includegraphics[width=\columnwidth]{../../../plots/far_field/thelonious/bubble/bubble_mass_vs_mass_avg_sar_all_1450MHz_log.pdf}
\caption{The bubble plot shows tissue mass vs Mass Avg SAR with bubble size proportional to tissue volume (log scale on both axes) for the all scenarios scenario at 1450 MHz for Thelonious.}
\label{fig:4_4}
\end{figure}

\subsection{Mass vs Mass Avg SAR All 2140 MHz Log}

The tissue Mass vs SAR Bubble Plot Log Scale (Duke) can be found in Figure~\ref{fig:4_5}. \\\\

\begin{figure}[H]
\centering
\includegraphics[width=\columnwidth]{../../../plots/far_field/duke/bubble/bubble_mass_vs_mass_avg_sar_all_2140MHz_log.pdf}
\caption{The bubble plot shows tissue mass vs Mass Avg SAR with bubble size proportional to tissue volume (log scale on both axes) for the all scenarios scenario at 2140 MHz for Duke.}
\label{fig:4_5}
\end{figure}

The tissue Mass vs SAR Bubble Plot Log Scale (Eartha) can be found in Figure~\ref{fig:4_6}. \\\\

\begin{figure}[H]
\centering
\includegraphics[width=\columnwidth]{../../../plots/far_field/eartha/bubble/bubble_mass_vs_mass_avg_sar_all_2140MHz_log.pdf}
\caption{The bubble plot shows tissue mass vs Mass Avg SAR with bubble size proportional to tissue volume (log scale on both axes) for the all scenarios scenario at 2140 MHz for Eartha.}
\label{fig:4_6}
\end{figure}

The tissue Mass vs SAR Bubble Plot Log Scale (Ella) can be found in Figure~\ref{fig:4_7}. \\\\

\begin{figure}[H]
\centering
\includegraphics[width=\columnwidth]{../../../plots/far_field/ella/bubble/bubble_mass_vs_mass_avg_sar_all_2140MHz_log.pdf}
\caption{The bubble plot shows tissue mass vs Mass Avg SAR with bubble size proportional to tissue volume (log scale on both axes) for the all scenarios scenario at 2140 MHz for Ella.}
\label{fig:4_7}
\end{figure}

The tissue Mass vs SAR Bubble Plot Log Scale (Thelonious) can be found in Figure~\ref{fig:4_8}. \\\\

\begin{figure}[H]
\centering
\includegraphics[width=\columnwidth]{../../../plots/far_field/thelonious/bubble/bubble_mass_vs_mass_avg_sar_all_2140MHz_log.pdf}
\caption{The bubble plot shows tissue mass vs Mass Avg SAR with bubble size proportional to tissue volume (log scale on both axes) for the all scenarios scenario at 2140 MHz for Thelonious.}
\label{fig:4_8}
\end{figure}

\subsection{Mass vs Mass Avg SAR All 2450 MHz Log}

The tissue Mass vs SAR Bubble Plot Log Scale (Duke) can be found in Figure~\ref{fig:4_9}. \\\\

\begin{figure}[H]
\centering
\includegraphics[width=\columnwidth]{../../../plots/far_field/duke/bubble/bubble_mass_vs_mass_avg_sar_all_2450MHz_log.pdf}
\caption{The bubble plot shows tissue mass vs Mass Avg SAR with bubble size proportional to tissue volume (log scale on both axes) for the all scenarios scenario at 2450 MHz for Duke.}
\label{fig:4_9}
\end{figure}

The tissue Mass vs SAR Bubble Plot Log Scale (Eartha) can be found in Figure~\ref{fig:4_10}. \\\\

\begin{figure}[H]
\centering
\includegraphics[width=\columnwidth]{../../../plots/far_field/eartha/bubble/bubble_mass_vs_mass_avg_sar_all_2450MHz_log.pdf}
\caption{The bubble plot shows tissue mass vs Mass Avg SAR with bubble size proportional to tissue volume (log scale on both axes) for the all scenarios scenario at 2450 MHz for Eartha.}
\label{fig:4_10}
\end{figure}

The tissue Mass vs SAR Bubble Plot Log Scale (Ella) can be found in Figure~\ref{fig:4_11}. \\\\

\begin{figure}[H]
\centering
\includegraphics[width=\columnwidth]{../../../plots/far_field/ella/bubble/bubble_mass_vs_mass_avg_sar_all_2450MHz_log.pdf}
\caption{The bubble plot shows tissue mass vs Mass Avg SAR with bubble size proportional to tissue volume (log scale on both axes) for the all scenarios scenario at 2450 MHz for Ella.}
\label{fig:4_11}
\end{figure}

The tissue Mass vs SAR Bubble Plot Log Scale (Thelonious) can be found in Figure~\ref{fig:4_12}. \\\\

\begin{figure}[H]
\centering
\includegraphics[width=\columnwidth]{../../../plots/far_field/thelonious/bubble/bubble_mass_vs_mass_avg_sar_all_2450MHz_log.pdf}
\caption{The bubble plot shows tissue mass vs Mass Avg SAR with bubble size proportional to tissue volume (log scale on both axes) for the all scenarios scenario at 2450 MHz for Thelonious.}
\label{fig:4_12}
\end{figure}

\subsection{Mass vs Mass Avg SAR All 3500 MHz Log}

The tissue Mass vs SAR Bubble Plot Log Scale (Duke) can be found in Figure~\ref{fig:4_13}. \\\\

\begin{figure}[H]
\centering
\includegraphics[width=\columnwidth]{../../../plots/far_field/duke/bubble/bubble_mass_vs_mass_avg_sar_all_3500MHz_log.pdf}
\caption{The bubble plot shows tissue mass vs Mass Avg SAR with bubble size proportional to tissue volume (log scale on both axes) for the all scenarios scenario at 3500 MHz for Duke.}
\label{fig:4_13}
\end{figure}

The tissue Mass vs SAR Bubble Plot Log Scale (Eartha) can be found in Figure~\ref{fig:4_14}. \\\\

\begin{figure}[H]
\centering
\includegraphics[width=\columnwidth]{../../../plots/far_field/eartha/bubble/bubble_mass_vs_mass_avg_sar_all_3500MHz_log.pdf}
\caption{The bubble plot shows tissue mass vs Mass Avg SAR with bubble size proportional to tissue volume (log scale on both axes) for the all scenarios scenario at 3500 MHz for Eartha.}
\label{fig:4_14}
\end{figure}

The tissue Mass vs SAR Bubble Plot Log Scale (Ella) can be found in Figure~\ref{fig:4_15}. \\\\

\begin{figure}[H]
\centering
\includegraphics[width=\columnwidth]{../../../plots/far_field/ella/bubble/bubble_mass_vs_mass_avg_sar_all_3500MHz_log.pdf}
\caption{The bubble plot shows tissue mass vs Mass Avg SAR with bubble size proportional to tissue volume (log scale on both axes) for the all scenarios scenario at 3500 MHz for Ella.}
\label{fig:4_15}
\end{figure}

The tissue Mass vs SAR Bubble Plot Log Scale (Thelonious) can be found in Figure~\ref{fig:4_16}. \\\\

\begin{figure}[H]
\centering
\includegraphics[width=\columnwidth]{../../../plots/far_field/thelonious/bubble/bubble_mass_vs_mass_avg_sar_all_3500MHz_log.pdf}
\caption{The bubble plot shows tissue mass vs Mass Avg SAR with bubble size proportional to tissue volume (log scale on both axes) for the all scenarios scenario at 3500 MHz for Thelonious.}
\label{fig:4_16}
\end{figure}

\subsection{Mass vs Mass Avg SAR All 450 MHz Log}

The tissue Mass vs SAR Bubble Plot Log Scale (Duke) can be found in Figure~\ref{fig:4_17}. \\\\

\begin{figure}[H]
\centering
\includegraphics[width=\columnwidth]{../../../plots/far_field/duke/bubble/bubble_mass_vs_mass_avg_sar_all_450MHz_log.pdf}
\caption{The bubble plot shows tissue mass vs Mass Avg SAR with bubble size proportional to tissue volume (log scale on both axes) for the all scenarios scenario at 450 MHz for Duke.}
\label{fig:4_17}
\end{figure}

The tissue Mass vs SAR Bubble Plot Log Scale (Eartha) can be found in Figure~\ref{fig:4_18}. \\\\

\begin{figure}[H]
\centering
\includegraphics[width=\columnwidth]{../../../plots/far_field/eartha/bubble/bubble_mass_vs_mass_avg_sar_all_450MHz_log.pdf}
\caption{The bubble plot shows tissue mass vs Mass Avg SAR with bubble size proportional to tissue volume (log scale on both axes) for the all scenarios scenario at 450 MHz for Eartha.}
\label{fig:4_18}
\end{figure}

The tissue Mass vs SAR Bubble Plot Log Scale (Ella) can be found in Figure~\ref{fig:4_19}. \\\\

\begin{figure}[H]
\centering
\includegraphics[width=\columnwidth]{../../../plots/far_field/ella/bubble/bubble_mass_vs_mass_avg_sar_all_450MHz_log.pdf}
\caption{The bubble plot shows tissue mass vs Mass Avg SAR with bubble size proportional to tissue volume (log scale on both axes) for the all scenarios scenario at 450 MHz for Ella.}
\label{fig:4_19}
\end{figure}

The tissue Mass vs SAR Bubble Plot Log Scale (Thelonious) can be found in Figure~\ref{fig:4_20}. \\\\

\begin{figure}[H]
\centering
\includegraphics[width=\columnwidth]{../../../plots/far_field/thelonious/bubble/bubble_mass_vs_mass_avg_sar_all_450MHz_log.pdf}
\caption{The bubble plot shows tissue mass vs Mass Avg SAR with bubble size proportional to tissue volume (log scale on both axes) for the all scenarios scenario at 450 MHz for Thelonious.}
\label{fig:4_20}
\end{figure}

\subsection{Mass vs Mass Avg SAR All 5200 MHz Log}

The tissue Mass vs SAR Bubble Plot Log Scale (Duke) can be found in Figure~\ref{fig:4_21}. \\\\

\begin{figure}[H]
\centering
\includegraphics[width=\columnwidth]{../../../plots/far_field/duke/bubble/bubble_mass_vs_mass_avg_sar_all_5200MHz_log.pdf}
\caption{The bubble plot shows tissue mass vs Mass Avg SAR with bubble size proportional to tissue volume (log scale on both axes) for the all scenarios scenario at 5200 MHz for Duke.}
\label{fig:4_21}
\end{figure}

The tissue Mass vs SAR Bubble Plot Log Scale (Eartha) can be found in Figure~\ref{fig:4_22}. \\\\

\begin{figure}[H]
\centering
\includegraphics[width=\columnwidth]{../../../plots/far_field/eartha/bubble/bubble_mass_vs_mass_avg_sar_all_5200MHz_log.pdf}
\caption{The bubble plot shows tissue mass vs Mass Avg SAR with bubble size proportional to tissue volume (log scale on both axes) for the all scenarios scenario at 5200 MHz for Eartha.}
\label{fig:4_22}
\end{figure}

The tissue Mass vs SAR Bubble Plot Log Scale (Ella) can be found in Figure~\ref{fig:4_23}. \\\\

\begin{figure}[H]
\centering
\includegraphics[width=\columnwidth]{../../../plots/far_field/ella/bubble/bubble_mass_vs_mass_avg_sar_all_5200MHz_log.pdf}
\caption{The bubble plot shows tissue mass vs Mass Avg SAR with bubble size proportional to tissue volume (log scale on both axes) for the all scenarios scenario at 5200 MHz for Ella.}
\label{fig:4_23}
\end{figure}

The tissue Mass vs SAR Bubble Plot Log Scale (Thelonious) can be found in Figure~\ref{fig:4_24}. \\\\

\begin{figure}[H]
\centering
\includegraphics[width=\columnwidth]{../../../plots/far_field/thelonious/bubble/bubble_mass_vs_mass_avg_sar_all_5200MHz_log.pdf}
\caption{The bubble plot shows tissue mass vs Mass Avg SAR with bubble size proportional to tissue volume (log scale on both axes) for the all scenarios scenario at 5200 MHz for Thelonious.}
\label{fig:4_24}
\end{figure}

\subsection{Mass vs Mass Avg SAR All 5800 MHz Log}

The tissue Mass vs SAR Bubble Plot Log Scale (Duke) can be found in Figure~\ref{fig:4_25}. \\\\

\begin{figure}[H]
\centering
\includegraphics[width=\columnwidth]{../../../plots/far_field/duke/bubble/bubble_mass_vs_mass_avg_sar_all_5800MHz_log.pdf}
\caption{The bubble plot shows tissue mass vs Mass Avg SAR with bubble size proportional to tissue volume (log scale on both axes) for the all scenarios scenario at 5800 MHz for Duke.}
\label{fig:4_25}
\end{figure}

The tissue Mass vs SAR Bubble Plot Log Scale (Eartha) can be found in Figure~\ref{fig:4_26}. \\\\

\begin{figure}[H]
\centering
\includegraphics[width=\columnwidth]{../../../plots/far_field/eartha/bubble/bubble_mass_vs_mass_avg_sar_all_5800MHz_log.pdf}
\caption{The bubble plot shows tissue mass vs Mass Avg SAR with bubble size proportional to tissue volume (log scale on both axes) for the all scenarios scenario at 5800 MHz for Eartha.}
\label{fig:4_26}
\end{figure}

The tissue Mass vs SAR Bubble Plot Log Scale (Ella) can be found in Figure~\ref{fig:4_27}. \\\\

\begin{figure}[H]
\centering
\includegraphics[width=\columnwidth]{../../../plots/far_field/ella/bubble/bubble_mass_vs_mass_avg_sar_all_5800MHz_log.pdf}
\caption{The bubble plot shows tissue mass vs Mass Avg SAR with bubble size proportional to tissue volume (log scale on both axes) for the all scenarios scenario at 5800 MHz for Ella.}
\label{fig:4_27}
\end{figure}

The tissue Mass vs SAR Bubble Plot Log Scale (Thelonious) can be found in Figure~\ref{fig:4_28}. \\\\

\begin{figure}[H]
\centering
\includegraphics[width=\columnwidth]{../../../plots/far_field/thelonious/bubble/bubble_mass_vs_mass_avg_sar_all_5800MHz_log.pdf}
\caption{The bubble plot shows tissue mass vs Mass Avg SAR with bubble size proportional to tissue volume (log scale on both axes) for the all scenarios scenario at 5800 MHz for Thelonious.}
\label{fig:4_28}
\end{figure}

\subsection{Mass vs Mass Avg SAR All 700 MHz Log}

The tissue Mass vs SAR Bubble Plot Log Scale (Duke) can be found in Figure~\ref{fig:4_29}. \\\\

\begin{figure}[H]
\centering
\includegraphics[width=\columnwidth]{../../../plots/far_field/duke/bubble/bubble_mass_vs_mass_avg_sar_all_700MHz_log.pdf}
\caption{The bubble plot shows tissue mass vs Mass Avg SAR with bubble size proportional to tissue volume (log scale on both axes) for the all scenarios scenario at 700 MHz for Duke.}
\label{fig:4_29}
\end{figure}

The tissue Mass vs SAR Bubble Plot Log Scale (Eartha) can be found in Figure~\ref{fig:4_30}. \\\\

\begin{figure}[H]
\centering
\includegraphics[width=\columnwidth]{../../../plots/far_field/eartha/bubble/bubble_mass_vs_mass_avg_sar_all_700MHz_log.pdf}
\caption{The bubble plot shows tissue mass vs Mass Avg SAR with bubble size proportional to tissue volume (log scale on both axes) for the all scenarios scenario at 700 MHz for Eartha.}
\label{fig:4_30}
\end{figure}

The tissue Mass vs SAR Bubble Plot Log Scale (Ella) can be found in Figure~\ref{fig:4_31}. \\\\

\begin{figure}[H]
\centering
\includegraphics[width=\columnwidth]{../../../plots/far_field/ella/bubble/bubble_mass_vs_mass_avg_sar_all_700MHz_log.pdf}
\caption{The bubble plot shows tissue mass vs Mass Avg SAR with bubble size proportional to tissue volume (log scale on both axes) for the all scenarios scenario at 700 MHz for Ella.}
\label{fig:4_31}
\end{figure}

The tissue Mass vs SAR Bubble Plot Log Scale (Thelonious) can be found in Figure~\ref{fig:4_32}. \\\\

\begin{figure}[H]
\centering
\includegraphics[width=\columnwidth]{../../../plots/far_field/thelonious/bubble/bubble_mass_vs_mass_avg_sar_all_700MHz_log.pdf}
\caption{The bubble plot shows tissue mass vs Mass Avg SAR with bubble size proportional to tissue volume (log scale on both axes) for the all scenarios scenario at 700 MHz for Thelonious.}
\label{fig:4_32}
\end{figure}

\subsection{Mass vs Mass Avg SAR All 835 MHz Log}

The tissue Mass vs SAR Bubble Plot Log Scale (Duke) can be found in Figure~\ref{fig:4_33}. \\\\

\begin{figure}[H]
\centering
\includegraphics[width=\columnwidth]{../../../plots/far_field/duke/bubble/bubble_mass_vs_mass_avg_sar_all_835MHz_log.pdf}
\caption{The bubble plot shows tissue mass vs Mass Avg SAR with bubble size proportional to tissue volume (log scale on both axes) for the all scenarios scenario at 835 MHz for Duke.}
\label{fig:4_33}
\end{figure}

The tissue Mass vs SAR Bubble Plot Log Scale (Eartha) can be found in Figure~\ref{fig:4_34}. \\\\

\begin{figure}[H]
\centering
\includegraphics[width=\columnwidth]{../../../plots/far_field/eartha/bubble/bubble_mass_vs_mass_avg_sar_all_835MHz_log.pdf}
\caption{The bubble plot shows tissue mass vs Mass Avg SAR with bubble size proportional to tissue volume (log scale on both axes) for the all scenarios scenario at 835 MHz for Eartha.}
\label{fig:4_34}
\end{figure}

The tissue Mass vs SAR Bubble Plot Log Scale (Ella) can be found in Figure~\ref{fig:4_35}. \\\\

\begin{figure}[H]
\centering
\includegraphics[width=\columnwidth]{../../../plots/far_field/ella/bubble/bubble_mass_vs_mass_avg_sar_all_835MHz_log.pdf}
\caption{The bubble plot shows tissue mass vs Mass Avg SAR with bubble size proportional to tissue volume (log scale on both axes) for the all scenarios scenario at 835 MHz for Ella.}
\label{fig:4_35}
\end{figure}

The tissue Mass vs SAR Bubble Plot Log Scale (Thelonious) can be found in Figure~\ref{fig:4_36}. \\\\

\begin{figure}[H]
\centering
\includegraphics[width=\columnwidth]{../../../plots/far_field/thelonious/bubble/bubble_mass_vs_mass_avg_sar_all_835MHz_log.pdf}
\caption{The bubble plot shows tissue mass vs Mass Avg SAR with bubble size proportional to tissue volume (log scale on both axes) for the all scenarios scenario at 835 MHz for Thelonious.}
\label{fig:4_36}
\end{figure}

\subsection{Mass vs Mass Avg SAR All All MHz Log}

The tissue Mass vs SAR Bubble Plot Log Scale (Duke) can be found in Figure~\ref{fig:4_37}. \\\\

\begin{figure}[H]
\centering
\includegraphics[width=\columnwidth]{../../../plots/far_field/duke/bubble/bubble_mass_vs_mass_avg_sar_all_allMHz_log.pdf}
\caption{The bubble plot shows tissue mass vs Mass Avg SAR with bubble size proportional to tissue volume (log scale on both axes) for the all scenarios scenario for Duke.}
\label{fig:4_37}
\end{figure}

The tissue Mass vs SAR Bubble Plot Log Scale (Eartha) can be found in Figure~\ref{fig:4_38}. \\\\

\begin{figure}[H]
\centering
\includegraphics[width=\columnwidth]{../../../plots/far_field/eartha/bubble/bubble_mass_vs_mass_avg_sar_all_allMHz_log.pdf}
\caption{The bubble plot shows tissue mass vs Mass Avg SAR with bubble size proportional to tissue volume (log scale on both axes) for the all scenarios scenario for Eartha.}
\label{fig:4_38}
\end{figure}

The tissue Mass vs SAR Bubble Plot Log Scale (Ella) can be found in Figure~\ref{fig:4_39}. \\\\

\begin{figure}[H]
\centering
\includegraphics[width=\columnwidth]{../../../plots/far_field/ella/bubble/bubble_mass_vs_mass_avg_sar_all_allMHz_log.pdf}
\caption{The bubble plot shows tissue mass vs Mass Avg SAR with bubble size proportional to tissue volume (log scale on both axes) for the all scenarios scenario for Ella.}
\label{fig:4_39}
\end{figure}

The tissue Mass vs SAR Bubble Plot Log Scale (Thelonious) can be found in Figure~\ref{fig:4_40}. \\\\

\begin{figure}[H]
\centering
\includegraphics[width=\columnwidth]{../../../plots/far_field/thelonious/bubble/bubble_mass_vs_mass_avg_sar_all_allMHz_log.pdf}
\caption{The bubble plot shows tissue mass vs Mass Avg SAR with bubble size proportional to tissue volume (log scale on both axes) for the all scenarios scenario for Thelonious.}
\label{fig:4_40}
\end{figure}

\subsection{Mass vs Max Local SAR All 1450 MHz Log}

The tissue Mass vs SAR Bubble Plot Log Scale (Duke) can be found in Figure~\ref{fig:4_41}. \\\\

\begin{figure}[H]
\centering
\includegraphics[width=\columnwidth]{../../../plots/far_field/duke/bubble/bubble_mass_vs_max_local_sar_all_1450MHz_log.pdf}
\caption{The bubble plot shows tissue mass vs Max Local SAR with bubble size proportional to tissue volume (log scale on both axes) for the all scenarios scenario at 1450 MHz for Duke.}
\label{fig:4_41}
\end{figure}

The tissue Mass vs SAR Bubble Plot Log Scale (Eartha) can be found in Figure~\ref{fig:4_42}. \\\\

\begin{figure}[H]
\centering
\includegraphics[width=\columnwidth]{../../../plots/far_field/eartha/bubble/bubble_mass_vs_max_local_sar_all_1450MHz_log.pdf}
\caption{The bubble plot shows tissue mass vs Max Local SAR with bubble size proportional to tissue volume (log scale on both axes) for the all scenarios scenario at 1450 MHz for Eartha.}
\label{fig:4_42}
\end{figure}

The tissue Mass vs SAR Bubble Plot Log Scale (Ella) can be found in Figure~\ref{fig:4_43}. \\\\

\begin{figure}[H]
\centering
\includegraphics[width=\columnwidth]{../../../plots/far_field/ella/bubble/bubble_mass_vs_max_local_sar_all_1450MHz_log.pdf}
\caption{The bubble plot shows tissue mass vs Max Local SAR with bubble size proportional to tissue volume (log scale on both axes) for the all scenarios scenario at 1450 MHz for Ella.}
\label{fig:4_43}
\end{figure}

The tissue Mass vs SAR Bubble Plot Log Scale (Thelonious) can be found in Figure~\ref{fig:4_44}. \\\\

\begin{figure}[H]
\centering
\includegraphics[width=\columnwidth]{../../../plots/far_field/thelonious/bubble/bubble_mass_vs_max_local_sar_all_1450MHz_log.pdf}
\caption{The bubble plot shows tissue mass vs Max Local SAR with bubble size proportional to tissue volume (log scale on both axes) for the all scenarios scenario at 1450 MHz for Thelonious.}
\label{fig:4_44}
\end{figure}

\subsection{Mass vs Max Local SAR All 2140 MHz Log}

The tissue Mass vs SAR Bubble Plot Log Scale (Duke) can be found in Figure~\ref{fig:4_45}. \\\\

\begin{figure}[H]
\centering
\includegraphics[width=\columnwidth]{../../../plots/far_field/duke/bubble/bubble_mass_vs_max_local_sar_all_2140MHz_log.pdf}
\caption{The bubble plot shows tissue mass vs Max Local SAR with bubble size proportional to tissue volume (log scale on both axes) for the all scenarios scenario at 2140 MHz for Duke.}
\label{fig:4_45}
\end{figure}

The tissue Mass vs SAR Bubble Plot Log Scale (Eartha) can be found in Figure~\ref{fig:4_46}. \\\\

\begin{figure}[H]
\centering
\includegraphics[width=\columnwidth]{../../../plots/far_field/eartha/bubble/bubble_mass_vs_max_local_sar_all_2140MHz_log.pdf}
\caption{The bubble plot shows tissue mass vs Max Local SAR with bubble size proportional to tissue volume (log scale on both axes) for the all scenarios scenario at 2140 MHz for Eartha.}
\label{fig:4_46}
\end{figure}

The tissue Mass vs SAR Bubble Plot Log Scale (Ella) can be found in Figure~\ref{fig:4_47}. \\\\

\begin{figure}[H]
\centering
\includegraphics[width=\columnwidth]{../../../plots/far_field/ella/bubble/bubble_mass_vs_max_local_sar_all_2140MHz_log.pdf}
\caption{The bubble plot shows tissue mass vs Max Local SAR with bubble size proportional to tissue volume (log scale on both axes) for the all scenarios scenario at 2140 MHz for Ella.}
\label{fig:4_47}
\end{figure}

The tissue Mass vs SAR Bubble Plot Log Scale (Thelonious) can be found in Figure~\ref{fig:4_48}. \\\\

\begin{figure}[H]
\centering
\includegraphics[width=\columnwidth]{../../../plots/far_field/thelonious/bubble/bubble_mass_vs_max_local_sar_all_2140MHz_log.pdf}
\caption{The bubble plot shows tissue mass vs Max Local SAR with bubble size proportional to tissue volume (log scale on both axes) for the all scenarios scenario at 2140 MHz for Thelonious.}
\label{fig:4_48}
\end{figure}

\subsection{Mass vs Max Local SAR All 2450 MHz Log}

The tissue Mass vs SAR Bubble Plot Log Scale (Duke) can be found in Figure~\ref{fig:4_49}. \\\\

\begin{figure}[H]
\centering
\includegraphics[width=\columnwidth]{../../../plots/far_field/duke/bubble/bubble_mass_vs_max_local_sar_all_2450MHz_log.pdf}
\caption{The bubble plot shows tissue mass vs Max Local SAR with bubble size proportional to tissue volume (log scale on both axes) for the all scenarios scenario at 2450 MHz for Duke.}
\label{fig:4_49}
\end{figure}

The tissue Mass vs SAR Bubble Plot Log Scale (Eartha) can be found in Figure~\ref{fig:4_50}. \\\\

\begin{figure}[H]
\centering
\includegraphics[width=\columnwidth]{../../../plots/far_field/eartha/bubble/bubble_mass_vs_max_local_sar_all_2450MHz_log.pdf}
\caption{The bubble plot shows tissue mass vs Max Local SAR with bubble size proportional to tissue volume (log scale on both axes) for the all scenarios scenario at 2450 MHz for Eartha.}
\label{fig:4_50}
\end{figure}

The tissue Mass vs SAR Bubble Plot Log Scale (Ella) can be found in Figure~\ref{fig:4_51}. \\\\

\begin{figure}[H]
\centering
\includegraphics[width=\columnwidth]{../../../plots/far_field/ella/bubble/bubble_mass_vs_max_local_sar_all_2450MHz_log.pdf}
\caption{The bubble plot shows tissue mass vs Max Local SAR with bubble size proportional to tissue volume (log scale on both axes) for the all scenarios scenario at 2450 MHz for Ella.}
\label{fig:4_51}
\end{figure}

The tissue Mass vs SAR Bubble Plot Log Scale (Thelonious) can be found in Figure~\ref{fig:4_52}. \\\\

\begin{figure}[H]
\centering
\includegraphics[width=\columnwidth]{../../../plots/far_field/thelonious/bubble/bubble_mass_vs_max_local_sar_all_2450MHz_log.pdf}
\caption{The bubble plot shows tissue mass vs Max Local SAR with bubble size proportional to tissue volume (log scale on both axes) for the all scenarios scenario at 2450 MHz for Thelonious.}
\label{fig:4_52}
\end{figure}

\subsection{Mass vs Max Local SAR All 3500 MHz Log}

The tissue Mass vs SAR Bubble Plot Log Scale (Duke) can be found in Figure~\ref{fig:4_53}. \\\\

\begin{figure}[H]
\centering
\includegraphics[width=\columnwidth]{../../../plots/far_field/duke/bubble/bubble_mass_vs_max_local_sar_all_3500MHz_log.pdf}
\caption{The bubble plot shows tissue mass vs Max Local SAR with bubble size proportional to tissue volume (log scale on both axes) for the all scenarios scenario at 3500 MHz for Duke.}
\label{fig:4_53}
\end{figure}

The tissue Mass vs SAR Bubble Plot Log Scale (Eartha) can be found in Figure~\ref{fig:4_54}. \\\\

\begin{figure}[H]
\centering
\includegraphics[width=\columnwidth]{../../../plots/far_field/eartha/bubble/bubble_mass_vs_max_local_sar_all_3500MHz_log.pdf}
\caption{The bubble plot shows tissue mass vs Max Local SAR with bubble size proportional to tissue volume (log scale on both axes) for the all scenarios scenario at 3500 MHz for Eartha.}
\label{fig:4_54}
\end{figure}

The tissue Mass vs SAR Bubble Plot Log Scale (Ella) can be found in Figure~\ref{fig:4_55}. \\\\

\begin{figure}[H]
\centering
\includegraphics[width=\columnwidth]{../../../plots/far_field/ella/bubble/bubble_mass_vs_max_local_sar_all_3500MHz_log.pdf}
\caption{The bubble plot shows tissue mass vs Max Local SAR with bubble size proportional to tissue volume (log scale on both axes) for the all scenarios scenario at 3500 MHz for Ella.}
\label{fig:4_55}
\end{figure}

The tissue Mass vs SAR Bubble Plot Log Scale (Thelonious) can be found in Figure~\ref{fig:4_56}. \\\\

\begin{figure}[H]
\centering
\includegraphics[width=\columnwidth]{../../../plots/far_field/thelonious/bubble/bubble_mass_vs_max_local_sar_all_3500MHz_log.pdf}
\caption{The bubble plot shows tissue mass vs Max Local SAR with bubble size proportional to tissue volume (log scale on both axes) for the all scenarios scenario at 3500 MHz for Thelonious.}
\label{fig:4_56}
\end{figure}

\subsection{Mass vs Max Local SAR All 450 MHz Log}

The tissue Mass vs SAR Bubble Plot Log Scale (Duke) can be found in Figure~\ref{fig:4_57}. \\\\

\begin{figure}[H]
\centering
\includegraphics[width=\columnwidth]{../../../plots/far_field/duke/bubble/bubble_mass_vs_max_local_sar_all_450MHz_log.pdf}
\caption{The bubble plot shows tissue mass vs Max Local SAR with bubble size proportional to tissue volume (log scale on both axes) for the all scenarios scenario at 450 MHz for Duke.}
\label{fig:4_57}
\end{figure}

The tissue Mass vs SAR Bubble Plot Log Scale (Eartha) can be found in Figure~\ref{fig:4_58}. \\\\

\begin{figure}[H]
\centering
\includegraphics[width=\columnwidth]{../../../plots/far_field/eartha/bubble/bubble_mass_vs_max_local_sar_all_450MHz_log.pdf}
\caption{The bubble plot shows tissue mass vs Max Local SAR with bubble size proportional to tissue volume (log scale on both axes) for the all scenarios scenario at 450 MHz for Eartha.}
\label{fig:4_58}
\end{figure}

The tissue Mass vs SAR Bubble Plot Log Scale (Ella) can be found in Figure~\ref{fig:4_59}. \\\\

\begin{figure}[H]
\centering
\includegraphics[width=\columnwidth]{../../../plots/far_field/ella/bubble/bubble_mass_vs_max_local_sar_all_450MHz_log.pdf}
\caption{The bubble plot shows tissue mass vs Max Local SAR with bubble size proportional to tissue volume (log scale on both axes) for the all scenarios scenario at 450 MHz for Ella.}
\label{fig:4_59}
\end{figure}

The tissue Mass vs SAR Bubble Plot Log Scale (Thelonious) can be found in Figure~\ref{fig:4_60}. \\\\

\begin{figure}[H]
\centering
\includegraphics[width=\columnwidth]{../../../plots/far_field/thelonious/bubble/bubble_mass_vs_max_local_sar_all_450MHz_log.pdf}
\caption{The bubble plot shows tissue mass vs Max Local SAR with bubble size proportional to tissue volume (log scale on both axes) for the all scenarios scenario at 450 MHz for Thelonious.}
\label{fig:4_60}
\end{figure}

\subsection{Mass vs Max Local SAR All 5200 MHz Log}

The tissue Mass vs SAR Bubble Plot Log Scale (Duke) can be found in Figure~\ref{fig:4_61}. \\\\

\begin{figure}[H]
\centering
\includegraphics[width=\columnwidth]{../../../plots/far_field/duke/bubble/bubble_mass_vs_max_local_sar_all_5200MHz_log.pdf}
\caption{The bubble plot shows tissue mass vs Max Local SAR with bubble size proportional to tissue volume (log scale on both axes) for the all scenarios scenario at 5200 MHz for Duke.}
\label{fig:4_61}
\end{figure}

The tissue Mass vs SAR Bubble Plot Log Scale (Eartha) can be found in Figure~\ref{fig:4_62}. \\\\

\begin{figure}[H]
\centering
\includegraphics[width=\columnwidth]{../../../plots/far_field/eartha/bubble/bubble_mass_vs_max_local_sar_all_5200MHz_log.pdf}
\caption{The bubble plot shows tissue mass vs Max Local SAR with bubble size proportional to tissue volume (log scale on both axes) for the all scenarios scenario at 5200 MHz for Eartha.}
\label{fig:4_62}
\end{figure}

The tissue Mass vs SAR Bubble Plot Log Scale (Ella) can be found in Figure~\ref{fig:4_63}. \\\\

\begin{figure}[H]
\centering
\includegraphics[width=\columnwidth]{../../../plots/far_field/ella/bubble/bubble_mass_vs_max_local_sar_all_5200MHz_log.pdf}
\caption{The bubble plot shows tissue mass vs Max Local SAR with bubble size proportional to tissue volume (log scale on both axes) for the all scenarios scenario at 5200 MHz for Ella.}
\label{fig:4_63}
\end{figure}

The tissue Mass vs SAR Bubble Plot Log Scale (Thelonious) can be found in Figure~\ref{fig:4_64}. \\\\

\begin{figure}[H]
\centering
\includegraphics[width=\columnwidth]{../../../plots/far_field/thelonious/bubble/bubble_mass_vs_max_local_sar_all_5200MHz_log.pdf}
\caption{The bubble plot shows tissue mass vs Max Local SAR with bubble size proportional to tissue volume (log scale on both axes) for the all scenarios scenario at 5200 MHz for Thelonious.}
\label{fig:4_64}
\end{figure}

\subsection{Mass vs Max Local SAR All 5800 MHz Log}

The tissue Mass vs SAR Bubble Plot Log Scale (Duke) can be found in Figure~\ref{fig:4_65}. \\\\

\begin{figure}[H]
\centering
\includegraphics[width=\columnwidth]{../../../plots/far_field/duke/bubble/bubble_mass_vs_max_local_sar_all_5800MHz_log.pdf}
\caption{The bubble plot shows tissue mass vs Max Local SAR with bubble size proportional to tissue volume (log scale on both axes) for the all scenarios scenario at 5800 MHz for Duke.}
\label{fig:4_65}
\end{figure}

The tissue Mass vs SAR Bubble Plot Log Scale (Eartha) can be found in Figure~\ref{fig:4_66}. \\\\

\begin{figure}[H]
\centering
\includegraphics[width=\columnwidth]{../../../plots/far_field/eartha/bubble/bubble_mass_vs_max_local_sar_all_5800MHz_log.pdf}
\caption{The bubble plot shows tissue mass vs Max Local SAR with bubble size proportional to tissue volume (log scale on both axes) for the all scenarios scenario at 5800 MHz for Eartha.}
\label{fig:4_66}
\end{figure}

The tissue Mass vs SAR Bubble Plot Log Scale (Ella) can be found in Figure~\ref{fig:4_67}. \\\\

\begin{figure}[H]
\centering
\includegraphics[width=\columnwidth]{../../../plots/far_field/ella/bubble/bubble_mass_vs_max_local_sar_all_5800MHz_log.pdf}
\caption{The bubble plot shows tissue mass vs Max Local SAR with bubble size proportional to tissue volume (log scale on both axes) for the all scenarios scenario at 5800 MHz for Ella.}
\label{fig:4_67}
\end{figure}

The tissue Mass vs SAR Bubble Plot Log Scale (Thelonious) can be found in Figure~\ref{fig:4_68}. \\\\

\begin{figure}[H]
\centering
\includegraphics[width=\columnwidth]{../../../plots/far_field/thelonious/bubble/bubble_mass_vs_max_local_sar_all_5800MHz_log.pdf}
\caption{The bubble plot shows tissue mass vs Max Local SAR with bubble size proportional to tissue volume (log scale on both axes) for the all scenarios scenario at 5800 MHz for Thelonious.}
\label{fig:4_68}
\end{figure}

\subsection{Mass vs Max Local SAR All 700 MHz Log}

The tissue Mass vs SAR Bubble Plot Log Scale (Duke) can be found in Figure~\ref{fig:4_69}. \\\\

\begin{figure}[H]
\centering
\includegraphics[width=\columnwidth]{../../../plots/far_field/duke/bubble/bubble_mass_vs_max_local_sar_all_700MHz_log.pdf}
\caption{The bubble plot shows tissue mass vs Max Local SAR with bubble size proportional to tissue volume (log scale on both axes) for the all scenarios scenario at 700 MHz for Duke.}
\label{fig:4_69}
\end{figure}

The tissue Mass vs SAR Bubble Plot Log Scale (Eartha) can be found in Figure~\ref{fig:4_70}. \\\\

\begin{figure}[H]
\centering
\includegraphics[width=\columnwidth]{../../../plots/far_field/eartha/bubble/bubble_mass_vs_max_local_sar_all_700MHz_log.pdf}
\caption{The bubble plot shows tissue mass vs Max Local SAR with bubble size proportional to tissue volume (log scale on both axes) for the all scenarios scenario at 700 MHz for Eartha.}
\label{fig:4_70}
\end{figure}

The tissue Mass vs SAR Bubble Plot Log Scale (Ella) can be found in Figure~\ref{fig:4_71}. \\\\

\begin{figure}[H]
\centering
\includegraphics[width=\columnwidth]{../../../plots/far_field/ella/bubble/bubble_mass_vs_max_local_sar_all_700MHz_log.pdf}
\caption{The bubble plot shows tissue mass vs Max Local SAR with bubble size proportional to tissue volume (log scale on both axes) for the all scenarios scenario at 700 MHz for Ella.}
\label{fig:4_71}
\end{figure}

The tissue Mass vs SAR Bubble Plot Log Scale (Thelonious) can be found in Figure~\ref{fig:4_72}. \\\\

\begin{figure}[H]
\centering
\includegraphics[width=\columnwidth]{../../../plots/far_field/thelonious/bubble/bubble_mass_vs_max_local_sar_all_700MHz_log.pdf}
\caption{The bubble plot shows tissue mass vs Max Local SAR with bubble size proportional to tissue volume (log scale on both axes) for the all scenarios scenario at 700 MHz for Thelonious.}
\label{fig:4_72}
\end{figure}

\subsection{Mass vs Max Local SAR All 835 MHz Log}

The tissue Mass vs SAR Bubble Plot Log Scale (Duke) can be found in Figure~\ref{fig:4_73}. \\\\

\begin{figure}[H]
\centering
\includegraphics[width=\columnwidth]{../../../plots/far_field/duke/bubble/bubble_mass_vs_max_local_sar_all_835MHz_log.pdf}
\caption{The bubble plot shows tissue mass vs Max Local SAR with bubble size proportional to tissue volume (log scale on both axes) for the all scenarios scenario at 835 MHz for Duke.}
\label{fig:4_73}
\end{figure}

The tissue Mass vs SAR Bubble Plot Log Scale (Eartha) can be found in Figure~\ref{fig:4_74}. \\\\

\begin{figure}[H]
\centering
\includegraphics[width=\columnwidth]{../../../plots/far_field/eartha/bubble/bubble_mass_vs_max_local_sar_all_835MHz_log.pdf}
\caption{The bubble plot shows tissue mass vs Max Local SAR with bubble size proportional to tissue volume (log scale on both axes) for the all scenarios scenario at 835 MHz for Eartha.}
\label{fig:4_74}
\end{figure}

The tissue Mass vs SAR Bubble Plot Log Scale (Ella) can be found in Figure~\ref{fig:4_75}. \\\\

\begin{figure}[H]
\centering
\includegraphics[width=\columnwidth]{../../../plots/far_field/ella/bubble/bubble_mass_vs_max_local_sar_all_835MHz_log.pdf}
\caption{The bubble plot shows tissue mass vs Max Local SAR with bubble size proportional to tissue volume (log scale on both axes) for the all scenarios scenario at 835 MHz for Ella.}
\label{fig:4_75}
\end{figure}

The tissue Mass vs SAR Bubble Plot Log Scale (Thelonious) can be found in Figure~\ref{fig:4_76}. \\\\

\begin{figure}[H]
\centering
\includegraphics[width=\columnwidth]{../../../plots/far_field/thelonious/bubble/bubble_mass_vs_max_local_sar_all_835MHz_log.pdf}
\caption{The bubble plot shows tissue mass vs Max Local SAR with bubble size proportional to tissue volume (log scale on both axes) for the all scenarios scenario at 835 MHz for Thelonious.}
\label{fig:4_76}
\end{figure}

\subsection{Mass vs Max Local SAR All All MHz Log}

The tissue Mass vs SAR Bubble Plot Log Scale (Duke) can be found in Figure~\ref{fig:4_77}. \\\\

\begin{figure}[H]
\centering
\includegraphics[width=\columnwidth]{../../../plots/far_field/duke/bubble/bubble_mass_vs_max_local_sar_all_allMHz_log.pdf}
\caption{The bubble plot shows tissue mass vs Max Local SAR with bubble size proportional to tissue volume (log scale on both axes) for the all scenarios scenario for Duke.}
\label{fig:4_77}
\end{figure}

The tissue Mass vs SAR Bubble Plot Log Scale (Eartha) can be found in Figure~\ref{fig:4_78}. \\\\

\begin{figure}[H]
\centering
\includegraphics[width=\columnwidth]{../../../plots/far_field/eartha/bubble/bubble_mass_vs_max_local_sar_all_allMHz_log.pdf}
\caption{The bubble plot shows tissue mass vs Max Local SAR with bubble size proportional to tissue volume (log scale on both axes) for the all scenarios scenario for Eartha.}
\label{fig:4_78}
\end{figure}

The tissue Mass vs SAR Bubble Plot Log Scale (Ella) can be found in Figure~\ref{fig:4_79}. \\\\

\begin{figure}[H]
\centering
\includegraphics[width=\columnwidth]{../../../plots/far_field/ella/bubble/bubble_mass_vs_max_local_sar_all_allMHz_log.pdf}
\caption{The bubble plot shows tissue mass vs Max Local SAR with bubble size proportional to tissue volume (log scale on both axes) for the all scenarios scenario for Ella.}
\label{fig:4_79}
\end{figure}

The tissue Mass vs SAR Bubble Plot Log Scale (Thelonious) can be found in Figure~\ref{fig:4_80}. \\\\

\begin{figure}[H]
\centering
\includegraphics[width=\columnwidth]{../../../plots/far_field/thelonious/bubble/bubble_mass_vs_max_local_sar_all_allMHz_log.pdf}
\caption{The bubble plot shows tissue mass vs Max Local SAR with bubble size proportional to tissue volume (log scale on both axes) for the all scenarios scenario for Thelonious.}
\label{fig:4_80}
\end{figure}

\subsection{Mass vs Peak SAR 10g All 1450 MHz Log}

The tissue Mass vs SAR Bubble Plot Log Scale (Duke) can be found in Figure~\ref{fig:4_81}. \\\\

\begin{figure}[H]
\centering
\includegraphics[width=\columnwidth]{../../../plots/far_field/duke/bubble/bubble_mass_vs_peak_sar_10g_all_1450MHz_log.pdf}
\caption{The bubble plot shows tissue mass vs Peak SAR 10g with bubble size proportional to tissue volume (log scale on both axes) for the all scenarios scenario at 1450 MHz for Duke.}
\label{fig:4_81}
\end{figure}

The tissue Mass vs SAR Bubble Plot Log Scale (Eartha) can be found in Figure~\ref{fig:4_82}. \\\\

\begin{figure}[H]
\centering
\includegraphics[width=\columnwidth]{../../../plots/far_field/eartha/bubble/bubble_mass_vs_peak_sar_10g_all_1450MHz_log.pdf}
\caption{The bubble plot shows tissue mass vs Peak SAR 10g with bubble size proportional to tissue volume (log scale on both axes) for the all scenarios scenario at 1450 MHz for Eartha.}
\label{fig:4_82}
\end{figure}

The tissue Mass vs SAR Bubble Plot Log Scale (Ella) can be found in Figure~\ref{fig:4_83}. \\\\

\begin{figure}[H]
\centering
\includegraphics[width=\columnwidth]{../../../plots/far_field/ella/bubble/bubble_mass_vs_peak_sar_10g_all_1450MHz_log.pdf}
\caption{The bubble plot shows tissue mass vs Peak SAR 10g with bubble size proportional to tissue volume (log scale on both axes) for the all scenarios scenario at 1450 MHz for Ella.}
\label{fig:4_83}
\end{figure}

The tissue Mass vs SAR Bubble Plot Log Scale (Thelonious) can be found in Figure~\ref{fig:4_84}. \\\\

\begin{figure}[H]
\centering
\includegraphics[width=\columnwidth]{../../../plots/far_field/thelonious/bubble/bubble_mass_vs_peak_sar_10g_all_1450MHz_log.pdf}
\caption{The bubble plot shows tissue mass vs Peak SAR 10g with bubble size proportional to tissue volume (log scale on both axes) for the all scenarios scenario at 1450 MHz for Thelonious.}
\label{fig:4_84}
\end{figure}

\subsection{Mass vs Peak SAR 10g All 2140 MHz Log}

The tissue Mass vs SAR Bubble Plot Log Scale (Duke) can be found in Figure~\ref{fig:4_85}. \\\\

\begin{figure}[H]
\centering
\includegraphics[width=\columnwidth]{../../../plots/far_field/duke/bubble/bubble_mass_vs_peak_sar_10g_all_2140MHz_log.pdf}
\caption{The bubble plot shows tissue mass vs Peak SAR 10g with bubble size proportional to tissue volume (log scale on both axes) for the all scenarios scenario at 2140 MHz for Duke.}
\label{fig:4_85}
\end{figure}

The tissue Mass vs SAR Bubble Plot Log Scale (Eartha) can be found in Figure~\ref{fig:4_86}. \\\\

\begin{figure}[H]
\centering
\includegraphics[width=\columnwidth]{../../../plots/far_field/eartha/bubble/bubble_mass_vs_peak_sar_10g_all_2140MHz_log.pdf}
\caption{The bubble plot shows tissue mass vs Peak SAR 10g with bubble size proportional to tissue volume (log scale on both axes) for the all scenarios scenario at 2140 MHz for Eartha.}
\label{fig:4_86}
\end{figure}

The tissue Mass vs SAR Bubble Plot Log Scale (Ella) can be found in Figure~\ref{fig:4_87}. \\\\

\begin{figure}[H]
\centering
\includegraphics[width=\columnwidth]{../../../plots/far_field/ella/bubble/bubble_mass_vs_peak_sar_10g_all_2140MHz_log.pdf}
\caption{The bubble plot shows tissue mass vs Peak SAR 10g with bubble size proportional to tissue volume (log scale on both axes) for the all scenarios scenario at 2140 MHz for Ella.}
\label{fig:4_87}
\end{figure}

The tissue Mass vs SAR Bubble Plot Log Scale (Thelonious) can be found in Figure~\ref{fig:4_88}. \\\\

\begin{figure}[H]
\centering
\includegraphics[width=\columnwidth]{../../../plots/far_field/thelonious/bubble/bubble_mass_vs_peak_sar_10g_all_2140MHz_log.pdf}
\caption{The bubble plot shows tissue mass vs Peak SAR 10g with bubble size proportional to tissue volume (log scale on both axes) for the all scenarios scenario at 2140 MHz for Thelonious.}
\label{fig:4_88}
\end{figure}

\subsection{Mass vs Peak SAR 10g All 2450 MHz Log}

The tissue Mass vs SAR Bubble Plot Log Scale (Duke) can be found in Figure~\ref{fig:4_89}. \\\\

\begin{figure}[H]
\centering
\includegraphics[width=\columnwidth]{../../../plots/far_field/duke/bubble/bubble_mass_vs_peak_sar_10g_all_2450MHz_log.pdf}
\caption{The bubble plot shows tissue mass vs Peak SAR 10g with bubble size proportional to tissue volume (log scale on both axes) for the all scenarios scenario at 2450 MHz for Duke.}
\label{fig:4_89}
\end{figure}

The tissue Mass vs SAR Bubble Plot Log Scale (Eartha) can be found in Figure~\ref{fig:4_90}. \\\\

\begin{figure}[H]
\centering
\includegraphics[width=\columnwidth]{../../../plots/far_field/eartha/bubble/bubble_mass_vs_peak_sar_10g_all_2450MHz_log.pdf}
\caption{The bubble plot shows tissue mass vs Peak SAR 10g with bubble size proportional to tissue volume (log scale on both axes) for the all scenarios scenario at 2450 MHz for Eartha.}
\label{fig:4_90}
\end{figure}

The tissue Mass vs SAR Bubble Plot Log Scale (Ella) can be found in Figure~\ref{fig:4_91}. \\\\

\begin{figure}[H]
\centering
\includegraphics[width=\columnwidth]{../../../plots/far_field/ella/bubble/bubble_mass_vs_peak_sar_10g_all_2450MHz_log.pdf}
\caption{The bubble plot shows tissue mass vs Peak SAR 10g with bubble size proportional to tissue volume (log scale on both axes) for the all scenarios scenario at 2450 MHz for Ella.}
\label{fig:4_91}
\end{figure}

The tissue Mass vs SAR Bubble Plot Log Scale (Thelonious) can be found in Figure~\ref{fig:4_92}. \\\\

\begin{figure}[H]
\centering
\includegraphics[width=\columnwidth]{../../../plots/far_field/thelonious/bubble/bubble_mass_vs_peak_sar_10g_all_2450MHz_log.pdf}
\caption{The bubble plot shows tissue mass vs Peak SAR 10g with bubble size proportional to tissue volume (log scale on both axes) for the all scenarios scenario at 2450 MHz for Thelonious.}
\label{fig:4_92}
\end{figure}

\subsection{Mass vs Peak SAR 10g All 3500 MHz Log}

The tissue Mass vs SAR Bubble Plot Log Scale (Duke) can be found in Figure~\ref{fig:4_93}. \\\\

\begin{figure}[H]
\centering
\includegraphics[width=\columnwidth]{../../../plots/far_field/duke/bubble/bubble_mass_vs_peak_sar_10g_all_3500MHz_log.pdf}
\caption{The bubble plot shows tissue mass vs Peak SAR 10g with bubble size proportional to tissue volume (log scale on both axes) for the all scenarios scenario at 3500 MHz for Duke.}
\label{fig:4_93}
\end{figure}

The tissue Mass vs SAR Bubble Plot Log Scale (Eartha) can be found in Figure~\ref{fig:4_94}. \\\\

\begin{figure}[H]
\centering
\includegraphics[width=\columnwidth]{../../../plots/far_field/eartha/bubble/bubble_mass_vs_peak_sar_10g_all_3500MHz_log.pdf}
\caption{The bubble plot shows tissue mass vs Peak SAR 10g with bubble size proportional to tissue volume (log scale on both axes) for the all scenarios scenario at 3500 MHz for Eartha.}
\label{fig:4_94}
\end{figure}

The tissue Mass vs SAR Bubble Plot Log Scale (Ella) can be found in Figure~\ref{fig:4_95}. \\\\

\begin{figure}[H]
\centering
\includegraphics[width=\columnwidth]{../../../plots/far_field/ella/bubble/bubble_mass_vs_peak_sar_10g_all_3500MHz_log.pdf}
\caption{The bubble plot shows tissue mass vs Peak SAR 10g with bubble size proportional to tissue volume (log scale on both axes) for the all scenarios scenario at 3500 MHz for Ella.}
\label{fig:4_95}
\end{figure}

The tissue Mass vs SAR Bubble Plot Log Scale (Thelonious) can be found in Figure~\ref{fig:4_96}. \\\\

\begin{figure}[H]
\centering
\includegraphics[width=\columnwidth]{../../../plots/far_field/thelonious/bubble/bubble_mass_vs_peak_sar_10g_all_3500MHz_log.pdf}
\caption{The bubble plot shows tissue mass vs Peak SAR 10g with bubble size proportional to tissue volume (log scale on both axes) for the all scenarios scenario at 3500 MHz for Thelonious.}
\label{fig:4_96}
\end{figure}

\subsection{Mass vs Peak SAR 10g All 450 MHz Log}

The tissue Mass vs SAR Bubble Plot Log Scale (Duke) can be found in Figure~\ref{fig:4_97}. \\\\

\begin{figure}[H]
\centering
\includegraphics[width=\columnwidth]{../../../plots/far_field/duke/bubble/bubble_mass_vs_peak_sar_10g_all_450MHz_log.pdf}
\caption{The bubble plot shows tissue mass vs Peak SAR 10g with bubble size proportional to tissue volume (log scale on both axes) for the all scenarios scenario at 450 MHz for Duke.}
\label{fig:4_97}
\end{figure}

The tissue Mass vs SAR Bubble Plot Log Scale (Eartha) can be found in Figure~\ref{fig:4_98}. \\\\

\begin{figure}[H]
\centering
\includegraphics[width=\columnwidth]{../../../plots/far_field/eartha/bubble/bubble_mass_vs_peak_sar_10g_all_450MHz_log.pdf}
\caption{The bubble plot shows tissue mass vs Peak SAR 10g with bubble size proportional to tissue volume (log scale on both axes) for the all scenarios scenario at 450 MHz for Eartha.}
\label{fig:4_98}
\end{figure}

The tissue Mass vs SAR Bubble Plot Log Scale (Ella) can be found in Figure~\ref{fig:4_99}. \\\\

\begin{figure}[H]
\centering
\includegraphics[width=\columnwidth]{../../../plots/far_field/ella/bubble/bubble_mass_vs_peak_sar_10g_all_450MHz_log.pdf}
\caption{The bubble plot shows tissue mass vs Peak SAR 10g with bubble size proportional to tissue volume (log scale on both axes) for the all scenarios scenario at 450 MHz for Ella.}
\label{fig:4_99}
\end{figure}

The tissue Mass vs SAR Bubble Plot Log Scale (Thelonious) can be found in Figure~\ref{fig:4_100}. \\\\

\begin{figure}[H]
\centering
\includegraphics[width=\columnwidth]{../../../plots/far_field/thelonious/bubble/bubble_mass_vs_peak_sar_10g_all_450MHz_log.pdf}
\caption{The bubble plot shows tissue mass vs Peak SAR 10g with bubble size proportional to tissue volume (log scale on both axes) for the all scenarios scenario at 450 MHz for Thelonious.}
\label{fig:4_100}
\end{figure}

\subsection{Mass vs Peak SAR 10g All 5200 MHz Log}

The tissue Mass vs SAR Bubble Plot Log Scale (Duke) can be found in Figure~\ref{fig:4_101}. \\\\

\begin{figure}[H]
\centering
\includegraphics[width=\columnwidth]{../../../plots/far_field/duke/bubble/bubble_mass_vs_peak_sar_10g_all_5200MHz_log.pdf}
\caption{The bubble plot shows tissue mass vs Peak SAR 10g with bubble size proportional to tissue volume (log scale on both axes) for the all scenarios scenario at 5200 MHz for Duke.}
\label{fig:4_101}
\end{figure}

The tissue Mass vs SAR Bubble Plot Log Scale (Eartha) can be found in Figure~\ref{fig:4_102}. \\\\

\begin{figure}[H]
\centering
\includegraphics[width=\columnwidth]{../../../plots/far_field/eartha/bubble/bubble_mass_vs_peak_sar_10g_all_5200MHz_log.pdf}
\caption{The bubble plot shows tissue mass vs Peak SAR 10g with bubble size proportional to tissue volume (log scale on both axes) for the all scenarios scenario at 5200 MHz for Eartha.}
\label{fig:4_102}
\end{figure}

The tissue Mass vs SAR Bubble Plot Log Scale (Ella) can be found in Figure~\ref{fig:4_103}. \\\\

\begin{figure}[H]
\centering
\includegraphics[width=\columnwidth]{../../../plots/far_field/ella/bubble/bubble_mass_vs_peak_sar_10g_all_5200MHz_log.pdf}
\caption{The bubble plot shows tissue mass vs Peak SAR 10g with bubble size proportional to tissue volume (log scale on both axes) for the all scenarios scenario at 5200 MHz for Ella.}
\label{fig:4_103}
\end{figure}

The tissue Mass vs SAR Bubble Plot Log Scale (Thelonious) can be found in Figure~\ref{fig:4_104}. \\\\

\begin{figure}[H]
\centering
\includegraphics[width=\columnwidth]{../../../plots/far_field/thelonious/bubble/bubble_mass_vs_peak_sar_10g_all_5200MHz_log.pdf}
\caption{The bubble plot shows tissue mass vs Peak SAR 10g with bubble size proportional to tissue volume (log scale on both axes) for the all scenarios scenario at 5200 MHz for Thelonious.}
\label{fig:4_104}
\end{figure}

\subsection{Mass vs Peak SAR 10g All 5800 MHz Log}

The tissue Mass vs SAR Bubble Plot Log Scale (Duke) can be found in Figure~\ref{fig:4_105}. \\\\

\begin{figure}[H]
\centering
\includegraphics[width=\columnwidth]{../../../plots/far_field/duke/bubble/bubble_mass_vs_peak_sar_10g_all_5800MHz_log.pdf}
\caption{The bubble plot shows tissue mass vs Peak SAR 10g with bubble size proportional to tissue volume (log scale on both axes) for the all scenarios scenario at 5800 MHz for Duke.}
\label{fig:4_105}
\end{figure}

The tissue Mass vs SAR Bubble Plot Log Scale (Eartha) can be found in Figure~\ref{fig:4_106}. \\\\

\begin{figure}[H]
\centering
\includegraphics[width=\columnwidth]{../../../plots/far_field/eartha/bubble/bubble_mass_vs_peak_sar_10g_all_5800MHz_log.pdf}
\caption{The bubble plot shows tissue mass vs Peak SAR 10g with bubble size proportional to tissue volume (log scale on both axes) for the all scenarios scenario at 5800 MHz for Eartha.}
\label{fig:4_106}
\end{figure}

The tissue Mass vs SAR Bubble Plot Log Scale (Ella) can be found in Figure~\ref{fig:4_107}. \\\\

\begin{figure}[H]
\centering
\includegraphics[width=\columnwidth]{../../../plots/far_field/ella/bubble/bubble_mass_vs_peak_sar_10g_all_5800MHz_log.pdf}
\caption{The bubble plot shows tissue mass vs Peak SAR 10g with bubble size proportional to tissue volume (log scale on both axes) for the all scenarios scenario at 5800 MHz for Ella.}
\label{fig:4_107}
\end{figure}

The tissue Mass vs SAR Bubble Plot Log Scale (Thelonious) can be found in Figure~\ref{fig:4_108}. \\\\

\begin{figure}[H]
\centering
\includegraphics[width=\columnwidth]{../../../plots/far_field/thelonious/bubble/bubble_mass_vs_peak_sar_10g_all_5800MHz_log.pdf}
\caption{The bubble plot shows tissue mass vs Peak SAR 10g with bubble size proportional to tissue volume (log scale on both axes) for the all scenarios scenario at 5800 MHz for Thelonious.}
\label{fig:4_108}
\end{figure}

\subsection{Mass vs Peak SAR 10g All 700 MHz Log}

The tissue Mass vs SAR Bubble Plot Log Scale (Duke) can be found in Figure~\ref{fig:4_109}. \\\\

\begin{figure}[H]
\centering
\includegraphics[width=\columnwidth]{../../../plots/far_field/duke/bubble/bubble_mass_vs_peak_sar_10g_all_700MHz_log.pdf}
\caption{The bubble plot shows tissue mass vs Peak SAR 10g with bubble size proportional to tissue volume (log scale on both axes) for the all scenarios scenario at 700 MHz for Duke.}
\label{fig:4_109}
\end{figure}

The tissue Mass vs SAR Bubble Plot Log Scale (Eartha) can be found in Figure~\ref{fig:4_110}. \\\\

\begin{figure}[H]
\centering
\includegraphics[width=\columnwidth]{../../../plots/far_field/eartha/bubble/bubble_mass_vs_peak_sar_10g_all_700MHz_log.pdf}
\caption{The bubble plot shows tissue mass vs Peak SAR 10g with bubble size proportional to tissue volume (log scale on both axes) for the all scenarios scenario at 700 MHz for Eartha.}
\label{fig:4_110}
\end{figure}

The tissue Mass vs SAR Bubble Plot Log Scale (Ella) can be found in Figure~\ref{fig:4_111}. \\\\

\begin{figure}[H]
\centering
\includegraphics[width=\columnwidth]{../../../plots/far_field/ella/bubble/bubble_mass_vs_peak_sar_10g_all_700MHz_log.pdf}
\caption{The bubble plot shows tissue mass vs Peak SAR 10g with bubble size proportional to tissue volume (log scale on both axes) for the all scenarios scenario at 700 MHz for Ella.}
\label{fig:4_111}
\end{figure}

The tissue Mass vs SAR Bubble Plot Log Scale (Thelonious) can be found in Figure~\ref{fig:4_112}. \\\\

\begin{figure}[H]
\centering
\includegraphics[width=\columnwidth]{../../../plots/far_field/thelonious/bubble/bubble_mass_vs_peak_sar_10g_all_700MHz_log.pdf}
\caption{The bubble plot shows tissue mass vs Peak SAR 10g with bubble size proportional to tissue volume (log scale on both axes) for the all scenarios scenario at 700 MHz for Thelonious.}
\label{fig:4_112}
\end{figure}

\subsection{Mass vs Peak SAR 10g All 835 MHz Log}

The tissue Mass vs SAR Bubble Plot Log Scale (Duke) can be found in Figure~\ref{fig:4_113}. \\\\

\begin{figure}[H]
\centering
\includegraphics[width=\columnwidth]{../../../plots/far_field/duke/bubble/bubble_mass_vs_peak_sar_10g_all_835MHz_log.pdf}
\caption{The bubble plot shows tissue mass vs Peak SAR 10g with bubble size proportional to tissue volume (log scale on both axes) for the all scenarios scenario at 835 MHz for Duke.}
\label{fig:4_113}
\end{figure}

The tissue Mass vs SAR Bubble Plot Log Scale (Eartha) can be found in Figure~\ref{fig:4_114}. \\\\

\begin{figure}[H]
\centering
\includegraphics[width=\columnwidth]{../../../plots/far_field/eartha/bubble/bubble_mass_vs_peak_sar_10g_all_835MHz_log.pdf}
\caption{The bubble plot shows tissue mass vs Peak SAR 10g with bubble size proportional to tissue volume (log scale on both axes) for the all scenarios scenario at 835 MHz for Eartha.}
\label{fig:4_114}
\end{figure}

The tissue Mass vs SAR Bubble Plot Log Scale (Ella) can be found in Figure~\ref{fig:4_115}. \\\\

\begin{figure}[H]
\centering
\includegraphics[width=\columnwidth]{../../../plots/far_field/ella/bubble/bubble_mass_vs_peak_sar_10g_all_835MHz_log.pdf}
\caption{The bubble plot shows tissue mass vs Peak SAR 10g with bubble size proportional to tissue volume (log scale on both axes) for the all scenarios scenario at 835 MHz for Ella.}
\label{fig:4_115}
\end{figure}

The tissue Mass vs SAR Bubble Plot Log Scale (Thelonious) can be found in Figure~\ref{fig:4_116}. \\\\

\begin{figure}[H]
\centering
\includegraphics[width=\columnwidth]{../../../plots/far_field/thelonious/bubble/bubble_mass_vs_peak_sar_10g_all_835MHz_log.pdf}
\caption{The bubble plot shows tissue mass vs Peak SAR 10g with bubble size proportional to tissue volume (log scale on both axes) for the all scenarios scenario at 835 MHz for Thelonious.}
\label{fig:4_116}
\end{figure}

\subsection{Mass vs Peak SAR 10g All All MHz Log}

The tissue Mass vs SAR Bubble Plot Log Scale (Duke) can be found in Figure~\ref{fig:4_117}. \\\\

\begin{figure}[H]
\centering
\includegraphics[width=\columnwidth]{../../../plots/far_field/duke/bubble/bubble_mass_vs_peak_sar_10g_all_allMHz_log.pdf}
\caption{The bubble plot shows tissue mass vs Peak SAR 10g with bubble size proportional to tissue volume (log scale on both axes) for the all scenarios scenario for Duke.}
\label{fig:4_117}
\end{figure}

The tissue Mass vs SAR Bubble Plot Log Scale (Eartha) can be found in Figure~\ref{fig:4_118}. \\\\

\begin{figure}[H]
\centering
\includegraphics[width=\columnwidth]{../../../plots/far_field/eartha/bubble/bubble_mass_vs_peak_sar_10g_all_allMHz_log.pdf}
\caption{The bubble plot shows tissue mass vs Peak SAR 10g with bubble size proportional to tissue volume (log scale on both axes) for the all scenarios scenario for Eartha.}
\label{fig:4_118}
\end{figure}

The tissue Mass vs SAR Bubble Plot Log Scale (Ella) can be found in Figure~\ref{fig:4_119}. \\\\

\begin{figure}[H]
\centering
\includegraphics[width=\columnwidth]{../../../plots/far_field/ella/bubble/bubble_mass_vs_peak_sar_10g_all_allMHz_log.pdf}
\caption{The bubble plot shows tissue mass vs Peak SAR 10g with bubble size proportional to tissue volume (log scale on both axes) for the all scenarios scenario for Ella.}
\label{fig:4_119}
\end{figure}

The tissue Mass vs SAR Bubble Plot Log Scale (Thelonious) can be found in Figure~\ref{fig:4_120}. \\\\

\begin{figure}[H]
\centering
\includegraphics[width=\columnwidth]{../../../plots/far_field/thelonious/bubble/bubble_mass_vs_peak_sar_10g_all_allMHz_log.pdf}
\caption{The bubble plot shows tissue mass vs Peak SAR 10g with bubble size proportional to tissue volume (log scale on both axes) for the all scenarios scenario for Thelonious.}
\label{fig:4_120}
\end{figure}

\newpage

\section{Spatial Analysis}

\subsection{Peak Location 2D}

The peak SAR location 2D projections (Duke) can be found in Figure~\ref{fig:5_1}. \\\\

\begin{figure}[H]
\centering
\includegraphics[width=\columnwidth]{../../../plots/far_field/duke/spatial/peak_location_2d.pdf}
\caption{The 2D projections show peak SAR locations onto Transverse-XY, Sagittal-XZ, and Coronal-YZ planes for the all scenarios scenario for Duke. Each point represents a peak location, colored by peak SAR value.}
\label{fig:5_1}
\end{figure}

The peak SAR location 2D projections (Eartha) can be found in Figure~\ref{fig:5_2}. \\\\

\begin{figure}[H]
\centering
\includegraphics[width=\columnwidth]{../../../plots/far_field/eartha/spatial/peak_location_2d.pdf}
\caption{The 2D projections show peak SAR locations onto Transverse-XY, Sagittal-XZ, and Coronal-YZ planes for the all scenarios scenario for Eartha. Each point represents a peak location, colored by peak SAR value.}
\label{fig:5_2}
\end{figure}

The peak SAR location 2D projections (Ella) can be found in Figure~\ref{fig:5_3}. \\\\

\begin{figure}[H]
\centering
\includegraphics[width=\columnwidth]{../../../plots/far_field/ella/spatial/peak_location_2d.pdf}
\caption{The 2D projections show peak SAR locations onto Transverse-XY, Sagittal-XZ, and Coronal-YZ planes for the all scenarios scenario for Ella. Each point represents a peak location, colored by peak SAR value.}
\label{fig:5_3}
\end{figure}

The peak SAR location 2D projections (Thelonious) can be found in Figure~\ref{fig:5_4}. \\\\

\begin{figure}[H]
\centering
\includegraphics[width=\columnwidth]{../../../plots/far_field/thelonious/spatial/peak_location_2d.pdf}
\caption{The 2D projections show peak SAR locations onto Transverse-XY, Sagittal-XZ, and Coronal-YZ planes for the all scenarios scenario for Thelonious. Each point represents a peak location, colored by peak SAR value.}
\label{fig:5_4}
\end{figure}

\newpage

\section{Correlation Analysis}

\subsection{Correlation Matrix Tissue Groups}

The tissue group correlation matrix (Duke) can be found in Figure~\ref{fig:6_1}. \\\\

\begin{figure}[H]
\centering
\includegraphics[width=\columnwidth]{../../../plots/far_field/duke/correlation/correlation_matrix_tissue_groups.pdf}
\caption{The heatmap shows Pearson correlation coefficients between different tissue group SAR values for the all scenarios scenario for Duke. Values range from -1 (perfect negative correlation) to +1 (perfect positive correlation).}
\label{fig:6_1}
\end{figure}

The tissue group correlation matrix (Eartha) can be found in Figure~\ref{fig:6_2}. \\\\

\begin{figure}[H]
\centering
\includegraphics[width=\columnwidth]{../../../plots/far_field/eartha/correlation/correlation_matrix_tissue_groups.pdf}
\caption{The heatmap shows Pearson correlation coefficients between different tissue group SAR values for the all scenarios scenario for Eartha. Values range from -1 (perfect negative correlation) to +1 (perfect positive correlation).}
\label{fig:6_2}
\end{figure}

The tissue group correlation matrix (Ella) can be found in Figure~\ref{fig:6_3}. \\\\

\begin{figure}[H]
\centering
\includegraphics[width=\columnwidth]{../../../plots/far_field/ella/correlation/correlation_matrix_tissue_groups.pdf}
\caption{The heatmap shows Pearson correlation coefficients between different tissue group SAR values for the all scenarios scenario for Ella. Values range from -1 (perfect negative correlation) to +1 (perfect positive correlation).}
\label{fig:6_3}
\end{figure}

The tissue group correlation matrix (Thelonious) can be found in Figure~\ref{fig:6_4}. \\\\

\begin{figure}[H]
\centering
\includegraphics[width=\columnwidth]{../../../plots/far_field/thelonious/correlation/correlation_matrix_tissue_groups.pdf}
\caption{The heatmap shows Pearson correlation coefficients between different tissue group SAR values for the all scenarios scenario for Thelonious. Values range from -1 (perfect negative correlation) to +1 (perfect positive correlation).}
\label{fig:6_4}
\end{figure}

\newpage

\section{Ranking Plots}

\subsection{Ranking Top 20 Mass Avg SAR All All MHz}

The top 20 tissues by Mass Avg SAR (Duke) can be found in Figure~\ref{fig:7_1}. \\\\

\begin{figure}[H]
\centering
\includegraphics[width=\columnwidth]{../../../plots/far_field/duke/ranking/ranking_top20_mass_avg_sar_all_allMHz.pdf}
\caption{The horizontal bar chart ranks the top 20 tissues by Mass Avg SAR (mW kg$^{-1}$) for the all scenarios scenario for Duke.}
\label{fig:7_1}
\end{figure}

The top 20 tissues by Mass Avg SAR (Eartha) can be found in Figure~\ref{fig:7_2}. \\\\

\begin{figure}[H]
\centering
\includegraphics[width=\columnwidth]{../../../plots/far_field/eartha/ranking/ranking_top20_mass_avg_sar_all_allMHz.pdf}
\caption{The horizontal bar chart ranks the top 20 tissues by Mass Avg SAR (mW kg$^{-1}$) for the all scenarios scenario for Eartha.}
\label{fig:7_2}
\end{figure}

The top 20 tissues by Mass Avg SAR (Ella) can be found in Figure~\ref{fig:7_3}. \\\\

\begin{figure}[H]
\centering
\includegraphics[width=\columnwidth]{../../../plots/far_field/ella/ranking/ranking_top20_mass_avg_sar_all_allMHz.pdf}
\caption{The horizontal bar chart ranks the top 20 tissues by Mass Avg SAR (mW kg$^{-1}$) for the all scenarios scenario for Ella.}
\label{fig:7_3}
\end{figure}

The top 20 tissues by Mass Avg SAR (Thelonious) can be found in Figure~\ref{fig:7_4}. \\\\

\begin{figure}[H]
\centering
\includegraphics[width=\columnwidth]{../../../plots/far_field/thelonious/ranking/ranking_top20_mass_avg_sar_all_allMHz.pdf}
\caption{The horizontal bar chart ranks the top 20 tissues by Mass Avg SAR (mW kg$^{-1}$) for the all scenarios scenario for Thelonious.}
\label{fig:7_4}
\end{figure}

\subsection{Ranking Top 20 Max Local SAR All All MHz}

The top 20 tissues by Max Local SAR (Duke) can be found in Figure~\ref{fig:7_5}. \\\\

\begin{figure}[H]
\centering
\includegraphics[width=\columnwidth]{../../../plots/far_field/duke/ranking/ranking_top20_max_local_sar_all_allMHz.pdf}
\caption{The horizontal bar chart ranks the top 20 tissues by Max Local SAR (mW kg$^{-1}$) for the all scenarios scenario for Duke. Note: values marked with * are outliers that exceed the display scale; the actual value is shown.}
\label{fig:7_5}
\end{figure}

The top 20 tissues by Max Local SAR (Eartha) can be found in Figure~\ref{fig:7_6}. \\\\

\begin{figure}[H]
\centering
\includegraphics[width=\columnwidth]{../../../plots/far_field/eartha/ranking/ranking_top20_max_local_sar_all_allMHz.pdf}
\caption{The horizontal bar chart ranks the top 20 tissues by Max Local SAR (mW kg$^{-1}$) for the all scenarios scenario for Eartha. Note: values marked with * are outliers that exceed the display scale; the actual value is shown.}
\label{fig:7_6}
\end{figure}

The top 20 tissues by Max Local SAR (Ella) can be found in Figure~\ref{fig:7_7}. \\\\

\begin{figure}[H]
\centering
\includegraphics[width=\columnwidth]{../../../plots/far_field/ella/ranking/ranking_top20_max_local_sar_all_allMHz.pdf}
\caption{The horizontal bar chart ranks the top 20 tissues by Max Local SAR (mW kg$^{-1}$) for the all scenarios scenario for Ella.}
\label{fig:7_7}
\end{figure}

The top 20 tissues by Max Local SAR (Thelonious) can be found in Figure~\ref{fig:7_8}. \\\\

\begin{figure}[H]
\centering
\includegraphics[width=\columnwidth]{../../../plots/far_field/thelonious/ranking/ranking_top20_max_local_sar_all_allMHz.pdf}
\caption{The horizontal bar chart ranks the top 20 tissues by Max Local SAR (mW kg$^{-1}$) for the all scenarios scenario for Thelonious.}
\label{fig:7_8}
\end{figure}

\subsection{Ranking Top 20 Peak SAR 10g All All MHz}

The top 20 tissues by Peak SAR 10g (Duke) can be found in Figure~\ref{fig:7_9}. \\\\

\begin{figure}[H]
\centering
\includegraphics[width=\columnwidth]{../../../plots/far_field/duke/ranking/ranking_top20_peak_sar_10g_all_allMHz.pdf}
\caption{The horizontal bar chart ranks the top 20 tissues by Peak SAR 10g (mW kg$^{-1}$) for the all scenarios scenario for Duke.}
\label{fig:7_9}
\end{figure}

The top 20 tissues by Peak SAR 10g (Eartha) can be found in Figure~\ref{fig:7_10}. \\\\

\begin{figure}[H]
\centering
\includegraphics[width=\columnwidth]{../../../plots/far_field/eartha/ranking/ranking_top20_peak_sar_10g_all_allMHz.pdf}
\caption{The horizontal bar chart ranks the top 20 tissues by Peak SAR 10g (mW kg$^{-1}$) for the all scenarios scenario for Eartha.}
\label{fig:7_10}
\end{figure}

The top 20 tissues by Peak SAR 10g (Ella) can be found in Figure~\ref{fig:7_11}. \\\\

\begin{figure}[H]
\centering
\includegraphics[width=\columnwidth]{../../../plots/far_field/ella/ranking/ranking_top20_peak_sar_10g_all_allMHz.pdf}
\caption{The horizontal bar chart ranks the top 20 tissues by Peak SAR 10g (mW kg$^{-1}$) for the all scenarios scenario for Ella.}
\label{fig:7_11}
\end{figure}

The top 20 tissues by Peak SAR 10g (Thelonious) can be found in Figure~\ref{fig:7_12}. \\\\

\begin{figure}[H]
\centering
\includegraphics[width=\columnwidth]{../../../plots/far_field/thelonious/ranking/ranking_top20_peak_sar_10g_all_allMHz.pdf}
\caption{The horizontal bar chart ranks the top 20 tissues by Peak SAR 10g (mW kg$^{-1}$) for the all scenarios scenario for Thelonious.}
\label{fig:7_12}
\end{figure}

\subsection{Ranking Top 20 Total Loss All All MHz}

The top 20 tissues by Total Loss (Duke) can be found in Figure~\ref{fig:7_13}. \\\\

\begin{figure}[H]
\centering
\includegraphics[width=\columnwidth]{../../../plots/far_field/duke/ranking/ranking_top20_Total Loss_all_allMHz.pdf}
\caption{The horizontal bar chart ranks the top 20 tissues by Total Loss ($\mu$W kg$^{{-1}}$) for the all scenarios scenario for Duke.}
\label{fig:7_13}
\end{figure}

The top 20 tissues by Total Loss (Eartha) can be found in Figure~\ref{fig:7_14}. \\\\

\begin{figure}[H]
\centering
\includegraphics[width=\columnwidth]{../../../plots/far_field/eartha/ranking/ranking_top20_Total Loss_all_allMHz.pdf}
\caption{The horizontal bar chart ranks the top 20 tissues by Total Loss ($\mu$W kg$^{{-1}}$) for the all scenarios scenario for Eartha.}
\label{fig:7_14}
\end{figure}

The top 20 tissues by Total Loss (Ella) can be found in Figure~\ref{fig:7_15}. \\\\

\begin{figure}[H]
\centering
\includegraphics[width=\columnwidth]{../../../plots/far_field/ella/ranking/ranking_top20_Total Loss_all_allMHz.pdf}
\caption{The horizontal bar chart ranks the top 20 tissues by Total Loss ($\mu$W kg$^{{-1}}$) for the all scenarios scenario for Ella.}
\label{fig:7_15}
\end{figure}

The top 20 tissues by Total Loss (Thelonious) can be found in Figure~\ref{fig:7_16}. \\\\

\begin{figure}[H]
\centering
\includegraphics[width=\columnwidth]{../../../plots/far_field/thelonious/ranking/ranking_top20_Total Loss_all_allMHz.pdf}
\caption{The horizontal bar chart ranks the top 20 tissues by Total Loss ($\mu$W kg$^{{-1}}$) for the all scenarios scenario for Thelonious.}
\label{fig:7_16}
\end{figure}

\newpage

\section{Power Analysis}

\subsection{Power Absorption All All MHz}

The power absorption distribution (Duke) can be found in Figure~\ref{fig:8_1}. \\\\

\begin{figure}[H]
\centering
\includegraphics[width=\columnwidth]{../../../plots/far_field/duke/power/power_absorption_all_allMHz.pdf}
\caption{The power absorption distribution analysis shows pie chart of total power loss by tissue group and stacked bar chart by frequency for the all scenarios scenario for Duke.}
\label{fig:8_1}
\end{figure}

The power absorption distribution (Eartha) can be found in Figure~\ref{fig:8_2}. \\\\

\begin{figure}[H]
\centering
\includegraphics[width=\columnwidth]{../../../plots/far_field/eartha/power/power_absorption_all_allMHz.pdf}
\caption{The power absorption distribution analysis shows pie chart of total power loss by tissue group and stacked bar chart by frequency for the all scenarios scenario for Eartha.}
\label{fig:8_2}
\end{figure}

The power absorption distribution (Ella) can be found in Figure~\ref{fig:8_3}. \\\\

\begin{figure}[H]
\centering
\includegraphics[width=\columnwidth]{../../../plots/far_field/ella/power/power_absorption_all_allMHz.pdf}
\caption{The power absorption distribution analysis shows pie chart of total power loss by tissue group and stacked bar chart by frequency for the all scenarios scenario for Ella.}
\label{fig:8_3}
\end{figure}

The power absorption distribution (Thelonious) can be found in Figure~\ref{fig:8_4}. \\\\

\begin{figure}[H]
\centering
\includegraphics[width=\columnwidth]{../../../plots/far_field/thelonious/power/power_absorption_all_allMHz.pdf}
\caption{The power absorption distribution analysis shows pie chart of total power loss by tissue group and stacked bar chart by frequency for the all scenarios scenario for Thelonious.}
\label{fig:8_4}
\end{figure}

\subsection{Power Balance Overview}

The power balance overview (Duke) can be found in Figure~\ref{fig:8_5}. \\\\

\begin{figure}[H]
\centering
\includegraphics[width=\columnwidth]{../../../plots/far_field/duke/power/power_balance_overview.pdf}
\caption{The power balance overview shows heatmaps of power balance percentage, input power, and radiated power across scenarios and frequencies for Duke.}
\label{fig:8_5}
\end{figure}

The power balance overview (Eartha) can be found in Figure~\ref{fig:8_6}. \\\\

\begin{figure}[H]
\centering
\includegraphics[width=\columnwidth]{../../../plots/far_field/eartha/power/power_balance_overview.pdf}
\caption{The power balance overview shows heatmaps of power balance percentage, input power, and radiated power across scenarios and frequencies for Eartha.}
\label{fig:8_6}
\end{figure}

The power balance overview (Ella) can be found in Figure~\ref{fig:8_7}. \\\\

\begin{figure}[H]
\centering
\includegraphics[width=\columnwidth]{../../../plots/far_field/ella/power/power_balance_overview.pdf}
\caption{The power balance overview shows heatmaps of power balance percentage, input power, and radiated power across scenarios and frequencies for Ella.}
\label{fig:8_7}
\end{figure}

The power balance overview (Thelonious) can be found in Figure~\ref{fig:8_8}. \\\\

\begin{figure}[H]
\centering
\includegraphics[width=\columnwidth]{../../../plots/far_field/thelonious/power/power_balance_overview.pdf}
\caption{The power balance overview shows heatmaps of power balance percentage, input power, and radiated power across scenarios and frequencies for Thelonious.}
\label{fig:8_8}
\end{figure}

\subsection{Power Efficiency All}

The power efficiency trends (Duke) can be found in Figure~\ref{fig:8_9}. \\\\

\begin{figure}[H]
\centering
\includegraphics[width=\columnwidth]{../../../plots/far_field/duke/power/power_efficiency_all.pdf}
\caption{The power efficiency analysis shows antenna efficiency and power component breakdown (dielectric loss, SIBC loss, radiated power) as a function of frequency for the all scenarios scenario for Duke.}
\label{fig:8_9}
\end{figure}

The power efficiency trends (Eartha) can be found in Figure~\ref{fig:8_10}. \\\\

\begin{figure}[H]
\centering
\includegraphics[width=\columnwidth]{../../../plots/far_field/eartha/power/power_efficiency_all.pdf}
\caption{The power efficiency analysis shows antenna efficiency and power component breakdown (dielectric loss, SIBC loss, radiated power) as a function of frequency for the all scenarios scenario for Eartha.}
\label{fig:8_10}
\end{figure}

The power efficiency trends (Ella) can be found in Figure~\ref{fig:8_11}. \\\\

\begin{figure}[H]
\centering
\includegraphics[width=\columnwidth]{../../../plots/far_field/ella/power/power_efficiency_all.pdf}
\caption{The power efficiency analysis shows antenna efficiency and power component breakdown (dielectric loss, SIBC loss, radiated power) as a function of frequency for the all scenarios scenario for Ella.}
\label{fig:8_11}
\end{figure}

The power efficiency trends (Thelonious) can be found in Figure~\ref{fig:8_12}. \\\\

\begin{figure}[H]
\centering
\includegraphics[width=\columnwidth]{../../../plots/far_field/thelonious/power/power_efficiency_all.pdf}
\caption{The power efficiency analysis shows antenna efficiency and power component breakdown (dielectric loss, SIBC loss, radiated power) as a function of frequency for the all scenarios scenario for Thelonious.}
\label{fig:8_12}
\end{figure}

\subsection{Power Loss Breakdown}

The power loss breakdown (Duke) can be found in Figure~\ref{fig:8_13}. \\\\

\begin{figure}[H]
\centering
\includegraphics[width=\columnwidth]{../../../plots/far_field/duke/power/power_loss_breakdown.pdf}
\caption{The power loss breakdown shows stacked bar chart of dielectric and SIBC power losses summed across frequencies for each scenario for Duke.}
\label{fig:8_13}
\end{figure}

The power loss breakdown (Eartha) can be found in Figure~\ref{fig:8_14}. \\\\

\begin{figure}[H]
\centering
\includegraphics[width=\columnwidth]{../../../plots/far_field/eartha/power/power_loss_breakdown.pdf}
\caption{The power loss breakdown shows stacked bar chart of dielectric and SIBC power losses summed across frequencies for each scenario for Eartha.}
\label{fig:8_14}
\end{figure}

The power loss breakdown (Ella) can be found in Figure~\ref{fig:8_15}. \\\\

\begin{figure}[H]
\centering
\includegraphics[width=\columnwidth]{../../../plots/far_field/ella/power/power_loss_breakdown.pdf}
\caption{The power loss breakdown shows stacked bar chart of dielectric and SIBC power losses summed across frequencies for each scenario for Ella.}
\label{fig:8_15}
\end{figure}

The power loss breakdown (Thelonious) can be found in Figure~\ref{fig:8_16}. \\\\

\begin{figure}[H]
\centering
\includegraphics[width=\columnwidth]{../../../plots/far_field/thelonious/power/power_loss_breakdown.pdf}
\caption{The power loss breakdown shows stacked bar chart of dielectric and SIBC power losses summed across frequencies for each scenario for Thelonious.}
\label{fig:8_16}
\end{figure}

\newpage

\section{Penetration Depth}

\subsection{Penetration Ratio SAR vs Frequency All}

The SAR penetration depth brain to skin SAR ratio versus frequency (Duke) can be found in Figure~\ref{fig:9_1}. \\\\

\begin{figure}[H]
\centering
\includegraphics[width=\columnwidth]{../../../plots/far_field/duke/penetration/penetration_ratio_SAR_vs_frequency_all.pdf}
\caption{The line plot shows the SAR penetration depth ratio (Brain/Skin) for SAR as a function of frequency for the all scenarios scenario for Duke. Higher ratios indicate deeper penetration into brain tissue relative to skin.}
\label{fig:9_1}
\end{figure}

The SAR penetration depth brain to skin SAR ratio versus frequency (Eartha) can be found in Figure~\ref{fig:9_2}. \\\\

\begin{figure}[H]
\centering
\includegraphics[width=\columnwidth]{../../../plots/far_field/eartha/penetration/penetration_ratio_SAR_vs_frequency_all.pdf}
\caption{The line plot shows the SAR penetration depth ratio (Brain/Skin) for SAR as a function of frequency for the all scenarios scenario for Eartha. Higher ratios indicate deeper penetration into brain tissue relative to skin.}
\label{fig:9_2}
\end{figure}

The SAR penetration depth brain to skin SAR ratio versus frequency (Ella) can be found in Figure~\ref{fig:9_3}. \\\\

\begin{figure}[H]
\centering
\includegraphics[width=\columnwidth]{../../../plots/far_field/ella/penetration/penetration_ratio_SAR_vs_frequency_all.pdf}
\caption{The line plot shows the SAR penetration depth ratio (Brain/Skin) for SAR as a function of frequency for the all scenarios scenario for Ella. Higher ratios indicate deeper penetration into brain tissue relative to skin.}
\label{fig:9_3}
\end{figure}

The SAR penetration depth brain to skin SAR ratio versus frequency (Thelonious) can be found in Figure~\ref{fig:9_4}. \\\\

\begin{figure}[H]
\centering
\includegraphics[width=\columnwidth]{../../../plots/far_field/thelonious/penetration/penetration_ratio_SAR_vs_frequency_all.pdf}
\caption{The line plot shows the SAR penetration depth ratio (Brain/Skin) for SAR as a function of frequency for the all scenarios scenario for Thelonious. Higher ratios indicate deeper penetration into brain tissue relative to skin.}
\label{fig:9_4}
\end{figure}

\subsection{Penetration Ratio psSAR10g vs Frequency All}

The SAR penetration depth brain to skin psSAR10g ratio versus frequency (Duke) can be found in Figure~\ref{fig:9_5}. \\\\

\begin{figure}[H]
\centering
\includegraphics[width=\columnwidth]{../../../plots/far_field/duke/penetration/penetration_ratio_psSAR10g_vs_frequency_all.pdf}
\caption{The line plot shows the SAR penetration depth ratio (Brain/Skin) for psSAR10g as a function of frequency for the all scenarios scenario for Duke. Higher ratios indicate deeper penetration into brain tissue relative to skin.}
\label{fig:9_5}
\end{figure}

The SAR penetration depth brain to skin psSAR10g ratio versus frequency (Eartha) can be found in Figure~\ref{fig:9_6}. \\\\

\begin{figure}[H]
\centering
\includegraphics[width=\columnwidth]{../../../plots/far_field/eartha/penetration/penetration_ratio_psSAR10g_vs_frequency_all.pdf}
\caption{The line plot shows the SAR penetration depth ratio (Brain/Skin) for psSAR10g as a function of frequency for the all scenarios scenario for Eartha. Higher ratios indicate deeper penetration into brain tissue relative to skin.}
\label{fig:9_6}
\end{figure}

The SAR penetration depth brain to skin psSAR10g ratio versus frequency (Ella) can be found in Figure~\ref{fig:9_7}. \\\\

\begin{figure}[H]
\centering
\includegraphics[width=\columnwidth]{../../../plots/far_field/ella/penetration/penetration_ratio_psSAR10g_vs_frequency_all.pdf}
\caption{The line plot shows the SAR penetration depth ratio (Brain/Skin) for psSAR10g as a function of frequency for the all scenarios scenario for Ella. Higher ratios indicate deeper penetration into brain tissue relative to skin.}
\label{fig:9_7}
\end{figure}

The SAR penetration depth brain to skin psSAR10g ratio versus frequency (Thelonious) can be found in Figure~\ref{fig:9_8}. \\\\

\begin{figure}[H]
\centering
\includegraphics[width=\columnwidth]{../../../plots/far_field/thelonious/penetration/penetration_ratio_psSAR10g_vs_frequency_all.pdf}
\caption{The line plot shows the SAR penetration depth ratio (Brain/Skin) for psSAR10g as a function of frequency for the all scenarios scenario for Thelonious. Higher ratios indicate deeper penetration into brain tissue relative to skin.}
\label{fig:9_8}
\end{figure}

\newpage

\section{Tissue Analysis}

\subsection{Distribution Mass Volume All}

The tissue mass/volume distribution (Duke) can be found in Figure~\ref{fig:10_1}. \\\\

\begin{figure}[H]
\centering
\includegraphics[width=\columnwidth]{../../../plots/far_field/duke/tissue_analysis/distribution_mass_volume_all.pdf}
\caption{The distribution analysis shows histograms of tissue total mass and total volume (log scale), and a scatter plot of volume vs mass with water density reference line for the all scenarios scenario for Duke.}
\label{fig:10_1}
\end{figure}

The tissue mass/volume distribution (Eartha) can be found in Figure~\ref{fig:10_2}. \\\\

\begin{figure}[H]
\centering
\includegraphics[width=\columnwidth]{../../../plots/far_field/eartha/tissue_analysis/distribution_mass_volume_all.pdf}
\caption{The distribution analysis shows histograms of tissue total mass and total volume (log scale), and a scatter plot of volume vs mass with water density reference line for the all scenarios scenario for Eartha.}
\label{fig:10_2}
\end{figure}

The tissue mass/volume distribution (Ella) can be found in Figure~\ref{fig:10_3}. \\\\

\begin{figure}[H]
\centering
\includegraphics[width=\columnwidth]{../../../plots/far_field/ella/tissue_analysis/distribution_mass_volume_all.pdf}
\caption{The distribution analysis shows histograms of tissue total mass and total volume (log scale), and a scatter plot of volume vs mass with water density reference line for the all scenarios scenario for Ella.}
\label{fig:10_3}
\end{figure}

The tissue mass/volume distribution (Thelonious) can be found in Figure~\ref{fig:10_4}. \\\\

\begin{figure}[H]
\centering
\includegraphics[width=\columnwidth]{../../../plots/far_field/thelonious/tissue_analysis/distribution_mass_volume_all.pdf}
\caption{The distribution analysis shows histograms of tissue total mass and total volume (log scale), and a scatter plot of volume vs mass with water density reference line for the all scenarios scenario for Thelonious.}
\label{fig:10_4}
\end{figure}

\subsection{Scatter Max Local vs psSAR10g All All MHz}

The max local SAR vs psSAR10g (Duke) can be found in Figure~\ref{fig:10_5}. \\\\

\begin{figure}[H]
\centering
\includegraphics[width=\columnwidth]{../../../plots/far_field/duke/tissue_analysis/scatter_MaxLocal_vs_psSAR10g_all_allMHz.pdf}
\caption{The scatter plot compares Max Local SAR and psSAR10g values for tissues in the all scenarios scenario for Duke. The red dashed line represents y=x (no spatial averaging). Points are colored by frequency when multiple frequencies are present.}
\label{fig:10_5}
\end{figure}

The max local SAR vs psSAR10g (Eartha) can be found in Figure~\ref{fig:10_6}. \\\\

\begin{figure}[H]
\centering
\includegraphics[width=\columnwidth]{../../../plots/far_field/eartha/tissue_analysis/scatter_MaxLocal_vs_psSAR10g_all_allMHz.pdf}
\caption{The scatter plot compares Max Local SAR and psSAR10g values for tissues in the all scenarios scenario for Eartha. The red dashed line represents y=x (no spatial averaging). Points are colored by frequency when multiple frequencies are present.}
\label{fig:10_6}
\end{figure}

The max local SAR vs psSAR10g (Ella) can be found in Figure~\ref{fig:10_7}. \\\\

\begin{figure}[H]
\centering
\includegraphics[width=\columnwidth]{../../../plots/far_field/ella/tissue_analysis/scatter_MaxLocal_vs_psSAR10g_all_allMHz.pdf}
\caption{The scatter plot compares Max Local SAR and psSAR10g values for tissues in the all scenarios scenario for Ella. The red dashed line represents y=x (no spatial averaging). Points are colored by frequency when multiple frequencies are present.}
\label{fig:10_7}
\end{figure}

The max local SAR vs psSAR10g (Thelonious) can be found in Figure~\ref{fig:10_8}. \\\\

\begin{figure}[H]
\centering
\includegraphics[width=\columnwidth]{../../../plots/far_field/thelonious/tissue_analysis/scatter_MaxLocal_vs_psSAR10g_all_allMHz.pdf}
\caption{The scatter plot compares Max Local SAR and psSAR10g values for tissues in the all scenarios scenario for Thelonious. The red dashed line represents y=x (no spatial averaging). Points are colored by frequency when multiple frequencies are present.}
\label{fig:10_8}
\end{figure}

\subsection{Tissue Frequency Response Cerebrospinal Fluid All}

The frequency response for Cerebrospinal fluid (Duke) can be found in Figure~\ref{fig:10_9}. \\\\

\begin{figure}[H]
\centering
\includegraphics[width=\columnwidth]{../../../plots/far_field/duke/tissue_analysis/tissue_frequency_response_Cerebrospinal_fluid_all.pdf}
\caption{The line plot shows the frequency response of SAR metrics (Min Local SAR, Mass-Averaged SAR, Max Local SAR) for Cerebrospinal\_fluid in the all scenarios scenario for Duke.}
\label{fig:10_9}
\end{figure}

The frequency response for Cerebrospinal fluid (Eartha) can be found in Figure~\ref{fig:10_10}. \\\\

\begin{figure}[H]
\centering
\includegraphics[width=\columnwidth]{../../../plots/far_field/eartha/tissue_analysis/tissue_frequency_response_Cerebrospinal_fluid_all.pdf}
\caption{The line plot shows the frequency response of SAR metrics (Min Local SAR, Mass-Averaged SAR, Max Local SAR) for Cerebrospinal\_fluid in the all scenarios scenario for Eartha.}
\label{fig:10_10}
\end{figure}

The frequency response for Cerebrospinal fluid (Ella) can be found in Figure~\ref{fig:10_11}. \\\\

\begin{figure}[H]
\centering
\includegraphics[width=\columnwidth]{../../../plots/far_field/ella/tissue_analysis/tissue_frequency_response_Cerebrospinal_fluid_all.pdf}
\caption{The line plot shows the frequency response of SAR metrics (Min Local SAR, Mass-Averaged SAR, Max Local SAR) for Cerebrospinal\_fluid in the all scenarios scenario for Ella.}
\label{fig:10_11}
\end{figure}

The frequency response for Cerebrospinal fluid (Thelonious) can be found in Figure~\ref{fig:10_12}. \\\\

\begin{figure}[H]
\centering
\includegraphics[width=\columnwidth]{../../../plots/far_field/thelonious/tissue_analysis/tissue_frequency_response_Cerebrospinal_fluid_all.pdf}
\caption{The line plot shows the frequency response of SAR metrics (Min Local SAR, Mass-Averaged SAR, Max Local SAR) for Cerebrospinal\_fluid in the all scenarios scenario for Thelonious.}
\label{fig:10_12}
\end{figure}

\subsection{Tissue Frequency Response Connective Tissue All}

The frequency response for Connective tissue (Duke) can be found in Figure~\ref{fig:10_13}. \\\\

\begin{figure}[H]
\centering
\includegraphics[width=\columnwidth]{../../../plots/far_field/duke/tissue_analysis/tissue_frequency_response_Connective_tissue_all.pdf}
\caption{The line plot shows the frequency response of SAR metrics (Min Local SAR, Mass-Averaged SAR, Max Local SAR) for Connective\_tissue in the all scenarios scenario for Duke.}
\label{fig:10_13}
\end{figure}

The frequency response for Connective tissue (Eartha) can be found in Figure~\ref{fig:10_14}. \\\\

\begin{figure}[H]
\centering
\includegraphics[width=\columnwidth]{../../../plots/far_field/eartha/tissue_analysis/tissue_frequency_response_Connective_tissue_all.pdf}
\caption{The line plot shows the frequency response of SAR metrics (Min Local SAR, Mass-Averaged SAR, Max Local SAR) for Connective\_tissue in the all scenarios scenario for Eartha.}
\label{fig:10_14}
\end{figure}

The frequency response for Connective tissue (Thelonious) can be found in Figure~\ref{fig:10_15}. \\\\

\begin{figure}[H]
\centering
\includegraphics[width=\columnwidth]{../../../plots/far_field/thelonious/tissue_analysis/tissue_frequency_response_Connective_tissue_all.pdf}
\caption{The line plot shows the frequency response of SAR metrics (Min Local SAR, Mass-Averaged SAR, Max Local SAR) for Connective\_tissue in the all scenarios scenario for Thelonious.}
\label{fig:10_15}
\end{figure}

\subsection{Tissue Frequency Response Cornea All}

The frequency response for Cornea (Duke) can be found in Figure~\ref{fig:10_16}. \\\\

\begin{figure}[H]
\centering
\includegraphics[width=\columnwidth]{../../../plots/far_field/duke/tissue_analysis/tissue_frequency_response_Cornea_all.pdf}
\caption{The line plot shows the frequency response of SAR metrics (Min Local SAR, Mass-Averaged SAR, Max Local SAR) for Cornea in the all scenarios scenario for Duke.}
\label{fig:10_16}
\end{figure}

The frequency response for Cornea (Eartha) can be found in Figure~\ref{fig:10_17}. \\\\

\begin{figure}[H]
\centering
\includegraphics[width=\columnwidth]{../../../plots/far_field/eartha/tissue_analysis/tissue_frequency_response_Cornea_all.pdf}
\caption{The line plot shows the frequency response of SAR metrics (Min Local SAR, Mass-Averaged SAR, Max Local SAR) for Cornea in the all scenarios scenario for Eartha.}
\label{fig:10_17}
\end{figure}

The frequency response for Cornea (Ella) can be found in Figure~\ref{fig:10_18}. \\\\

\begin{figure}[H]
\centering
\includegraphics[width=\columnwidth]{../../../plots/far_field/ella/tissue_analysis/tissue_frequency_response_Cornea_all.pdf}
\caption{The line plot shows the frequency response of SAR metrics (Min Local SAR, Mass-Averaged SAR, Max Local SAR) for Cornea in the all scenarios scenario for Ella.}
\label{fig:10_18}
\end{figure}

The frequency response for Cornea (Thelonious) can be found in Figure~\ref{fig:10_19}. \\\\

\begin{figure}[H]
\centering
\includegraphics[width=\columnwidth]{../../../plots/far_field/thelonious/tissue_analysis/tissue_frequency_response_Cornea_all.pdf}
\caption{The line plot shows the frequency response of SAR metrics (Min Local SAR, Mass-Averaged SAR, Max Local SAR) for Cornea in the all scenarios scenario for Thelonious.}
\label{fig:10_19}
\end{figure}

\subsection{Tissue Frequency Response Ear Cartilage All}

The frequency response for Ear cartilage (Duke) can be found in Figure~\ref{fig:10_20}. \\\\

\begin{figure}[H]
\centering
\includegraphics[width=\columnwidth]{../../../plots/far_field/duke/tissue_analysis/tissue_frequency_response_Ear_cartilage_all.pdf}
\caption{The line plot shows the frequency response of SAR metrics (Min Local SAR, Mass-Averaged SAR, Max Local SAR) for Ear\_cartilage in the all scenarios scenario for Duke.}
\label{fig:10_20}
\end{figure}

The frequency response for Ear cartilage (Eartha) can be found in Figure~\ref{fig:10_21}. \\\\

\begin{figure}[H]
\centering
\includegraphics[width=\columnwidth]{../../../plots/far_field/eartha/tissue_analysis/tissue_frequency_response_Ear_cartilage_all.pdf}
\caption{The line plot shows the frequency response of SAR metrics (Min Local SAR, Mass-Averaged SAR, Max Local SAR) for Ear\_cartilage in the all scenarios scenario for Eartha.}
\label{fig:10_21}
\end{figure}

The frequency response for Ear cartilage (Ella) can be found in Figure~\ref{fig:10_22}. \\\\

\begin{figure}[H]
\centering
\includegraphics[width=\columnwidth]{../../../plots/far_field/ella/tissue_analysis/tissue_frequency_response_Ear_cartilage_all.pdf}
\caption{The line plot shows the frequency response of SAR metrics (Min Local SAR, Mass-Averaged SAR, Max Local SAR) for Ear\_cartilage in the all scenarios scenario for Ella.}
\label{fig:10_22}
\end{figure}

The frequency response for Ear cartilage (Thelonious) can be found in Figure~\ref{fig:10_23}. \\\\

\begin{figure}[H]
\centering
\includegraphics[width=\columnwidth]{../../../plots/far_field/thelonious/tissue_analysis/tissue_frequency_response_Ear_cartilage_all.pdf}
\caption{The line plot shows the frequency response of SAR metrics (Min Local SAR, Mass-Averaged SAR, Max Local SAR) for Ear\_cartilage in the all scenarios scenario for Thelonious.}
\label{fig:10_23}
\end{figure}

\subsection{Tissue Frequency Response Ear Skin All}

The frequency response for Ear skin (Duke) can be found in Figure~\ref{fig:10_24}. \\\\

\begin{figure}[H]
\centering
\includegraphics[width=\columnwidth]{../../../plots/far_field/duke/tissue_analysis/tissue_frequency_response_Ear_skin_all.pdf}
\caption{The line plot shows the frequency response of SAR metrics (Min Local SAR, Mass-Averaged SAR, Max Local SAR) for Ear\_skin in the all scenarios scenario for Duke.}
\label{fig:10_24}
\end{figure}

The frequency response for Ear skin (Eartha) can be found in Figure~\ref{fig:10_25}. \\\\

\begin{figure}[H]
\centering
\includegraphics[width=\columnwidth]{../../../plots/far_field/eartha/tissue_analysis/tissue_frequency_response_Ear_skin_all.pdf}
\caption{The line plot shows the frequency response of SAR metrics (Min Local SAR, Mass-Averaged SAR, Max Local SAR) for Ear\_skin in the all scenarios scenario for Eartha.}
\label{fig:10_25}
\end{figure}

The frequency response for Ear skin (Ella) can be found in Figure~\ref{fig:10_26}. \\\\

\begin{figure}[H]
\centering
\includegraphics[width=\columnwidth]{../../../plots/far_field/ella/tissue_analysis/tissue_frequency_response_Ear_skin_all.pdf}
\caption{The line plot shows the frequency response of SAR metrics (Min Local SAR, Mass-Averaged SAR, Max Local SAR) for Ear\_skin in the all scenarios scenario for Ella.}
\label{fig:10_26}
\end{figure}

The frequency response for Ear skin (Thelonious) can be found in Figure~\ref{fig:10_27}. \\\\

\begin{figure}[H]
\centering
\includegraphics[width=\columnwidth]{../../../plots/far_field/thelonious/tissue_analysis/tissue_frequency_response_Ear_skin_all.pdf}
\caption{The line plot shows the frequency response of SAR metrics (Min Local SAR, Mass-Averaged SAR, Max Local SAR) for Ear\_skin in the all scenarios scenario for Thelonious.}
\label{fig:10_27}
\end{figure}

\subsection{Tissue Frequency Response Eye Sclera All}

The frequency response for Eye Sclera (Duke) can be found in Figure~\ref{fig:10_28}. \\\\

\begin{figure}[H]
\centering
\includegraphics[width=\columnwidth]{../../../plots/far_field/duke/tissue_analysis/tissue_frequency_response_Eye_Sclera_all.pdf}
\caption{The line plot shows the frequency response of SAR metrics (Min Local SAR, Mass-Averaged SAR, Max Local SAR) for Eye\_Sclera in the all scenarios scenario for Duke.}
\label{fig:10_28}
\end{figure}

The frequency response for Eye Sclera (Eartha) can be found in Figure~\ref{fig:10_29}. \\\\

\begin{figure}[H]
\centering
\includegraphics[width=\columnwidth]{../../../plots/far_field/eartha/tissue_analysis/tissue_frequency_response_Eye_Sclera_all.pdf}
\caption{The line plot shows the frequency response of SAR metrics (Min Local SAR, Mass-Averaged SAR, Max Local SAR) for Eye\_Sclera in the all scenarios scenario for Eartha.}
\label{fig:10_29}
\end{figure}

The frequency response for Eye Sclera (Ella) can be found in Figure~\ref{fig:10_30}. \\\\

\begin{figure}[H]
\centering
\includegraphics[width=\columnwidth]{../../../plots/far_field/ella/tissue_analysis/tissue_frequency_response_Eye_Sclera_all.pdf}
\caption{The line plot shows the frequency response of SAR metrics (Min Local SAR, Mass-Averaged SAR, Max Local SAR) for Eye\_Sclera in the all scenarios scenario for Ella.}
\label{fig:10_30}
\end{figure}

\subsection{Tissue Frequency Response Eye Vitreous Humor All}

The frequency response for Eye vitreous humor (Duke) can be found in Figure~\ref{fig:10_31}. \\\\

\begin{figure}[H]
\centering
\includegraphics[width=\columnwidth]{../../../plots/far_field/duke/tissue_analysis/tissue_frequency_response_Eye_vitreous_humor_all.pdf}
\caption{The line plot shows the frequency response of SAR metrics (Min Local SAR, Mass-Averaged SAR, Max Local SAR) for Eye\_vitreous\_humor in the all scenarios scenario for Duke.}
\label{fig:10_31}
\end{figure}

The frequency response for Eye vitreous humor (Eartha) can be found in Figure~\ref{fig:10_32}. \\\\

\begin{figure}[H]
\centering
\includegraphics[width=\columnwidth]{../../../plots/far_field/eartha/tissue_analysis/tissue_frequency_response_Eye_vitreous_humor_all.pdf}
\caption{The line plot shows the frequency response of SAR metrics (Min Local SAR, Mass-Averaged SAR, Max Local SAR) for Eye\_vitreous\_humor in the all scenarios scenario for Eartha.}
\label{fig:10_32}
\end{figure}

The frequency response for Eye vitreous humor (Ella) can be found in Figure~\ref{fig:10_33}. \\\\

\begin{figure}[H]
\centering
\includegraphics[width=\columnwidth]{../../../plots/far_field/ella/tissue_analysis/tissue_frequency_response_Eye_vitreous_humor_all.pdf}
\caption{The line plot shows the frequency response of SAR metrics (Min Local SAR, Mass-Averaged SAR, Max Local SAR) for Eye\_vitreous\_humor in the all scenarios scenario for Ella.}
\label{fig:10_33}
\end{figure}

The frequency response for Eye vitreous humor (Thelonious) can be found in Figure~\ref{fig:10_34}. \\\\

\begin{figure}[H]
\centering
\includegraphics[width=\columnwidth]{../../../plots/far_field/thelonious/tissue_analysis/tissue_frequency_response_Eye_vitreous_humor_all.pdf}
\caption{The line plot shows the frequency response of SAR metrics (Min Local SAR, Mass-Averaged SAR, Max Local SAR) for Eye\_vitreous\_humor in the all scenarios scenario for Thelonious.}
\label{fig:10_34}
\end{figure}

\subsection{Tissue Frequency Response Meniscus All}

The frequency response for Meniscus (Eartha) can be found in Figure~\ref{fig:10_35}. \\\\

\begin{figure}[H]
\centering
\includegraphics[width=\columnwidth]{../../../plots/far_field/eartha/tissue_analysis/tissue_frequency_response_Meniscus_all.pdf}
\caption{The line plot shows the frequency response of SAR metrics (Min Local SAR, Mass-Averaged SAR, Max Local SAR) for Meniscus in the all scenarios scenario for Eartha.}
\label{fig:10_35}
\end{figure}

The frequency response for Meniscus (Thelonious) can be found in Figure~\ref{fig:10_36}. \\\\

\begin{figure}[H]
\centering
\includegraphics[width=\columnwidth]{../../../plots/far_field/thelonious/tissue_analysis/tissue_frequency_response_Meniscus_all.pdf}
\caption{The line plot shows the frequency response of SAR metrics (Min Local SAR, Mass-Averaged SAR, Max Local SAR) for Meniscus in the all scenarios scenario for Thelonious.}
\label{fig:10_36}
\end{figure}

\subsection{Tissue Frequency Response Mucosa All}

The frequency response for Mucosa (Ella) can be found in Figure~\ref{fig:10_37}. \\\\

\begin{figure}[H]
\centering
\includegraphics[width=\columnwidth]{../../../plots/far_field/ella/tissue_analysis/tissue_frequency_response_Mucosa_all.pdf}
\caption{The line plot shows the frequency response of SAR metrics (Min Local SAR, Mass-Averaged SAR, Max Local SAR) for Mucosa in the all scenarios scenario for Ella.}
\label{fig:10_37}
\end{figure}

\subsection{Tissue Frequency Response Skin All}

The frequency response for Skin (Duke) can be found in Figure~\ref{fig:10_38}. \\\\

\begin{figure}[H]
\centering
\includegraphics[width=\columnwidth]{../../../plots/far_field/duke/tissue_analysis/tissue_frequency_response_Skin_all.pdf}
\caption{The line plot shows the frequency response of SAR metrics (Min Local SAR, Mass-Averaged SAR, Max Local SAR) for Skin in the all scenarios scenario for Duke.}
\label{fig:10_38}
\end{figure}

The frequency response for Skin (Eartha) can be found in Figure~\ref{fig:10_39}. \\\\

\begin{figure}[H]
\centering
\includegraphics[width=\columnwidth]{../../../plots/far_field/eartha/tissue_analysis/tissue_frequency_response_Skin_all.pdf}
\caption{The line plot shows the frequency response of SAR metrics (Min Local SAR, Mass-Averaged SAR, Max Local SAR) for Skin in the all scenarios scenario for Eartha.}
\label{fig:10_39}
\end{figure}

The frequency response for Skin (Ella) can be found in Figure~\ref{fig:10_40}. \\\\

\begin{figure}[H]
\centering
\includegraphics[width=\columnwidth]{../../../plots/far_field/ella/tissue_analysis/tissue_frequency_response_Skin_all.pdf}
\caption{The line plot shows the frequency response of SAR metrics (Min Local SAR, Mass-Averaged SAR, Max Local SAR) for Skin in the all scenarios scenario for Ella.}
\label{fig:10_40}
\end{figure}

The frequency response for Skin (Thelonious) can be found in Figure~\ref{fig:10_41}. \\\\

\begin{figure}[H]
\centering
\includegraphics[width=\columnwidth]{../../../plots/far_field/thelonious/tissue_analysis/tissue_frequency_response_Skin_all.pdf}
\caption{The line plot shows the frequency response of SAR metrics (Min Local SAR, Mass-Averaged SAR, Max Local SAR) for Skin in the all scenarios scenario for Thelonious.}
\label{fig:10_41}
\end{figure}

\subsection{Tissue Frequency Response Tendon Ligament All}

The frequency response for Tendon Ligament (Eartha) can be found in Figure~\ref{fig:10_42}. \\\\

\begin{figure}[H]
\centering
\includegraphics[width=\columnwidth]{../../../plots/far_field/eartha/tissue_analysis/tissue_frequency_response_Tendon_Ligament_all.pdf}
\caption{The line plot shows the frequency response of SAR metrics (Min Local SAR, Mass-Averaged SAR, Max Local SAR) for Tendon\_Ligament in the all scenarios scenario for Eartha.}
\label{fig:10_42}
\end{figure}

The frequency response for Tendon Ligament (Ella) can be found in Figure~\ref{fig:10_43}. \\\\

\begin{figure}[H]
\centering
\includegraphics[width=\columnwidth]{../../../plots/far_field/ella/tissue_analysis/tissue_frequency_response_Tendon_Ligament_all.pdf}
\caption{The line plot shows the frequency response of SAR metrics (Min Local SAR, Mass-Averaged SAR, Max Local SAR) for Tendon\_Ligament in the all scenarios scenario for Ella.}
\label{fig:10_43}
\end{figure}

The frequency response for Tendon Ligament (Thelonious) can be found in Figure~\ref{fig:10_44}. \\\\

\begin{figure}[H]
\centering
\includegraphics[width=\columnwidth]{../../../plots/far_field/thelonious/tissue_analysis/tissue_frequency_response_Tendon_Ligament_all.pdf}
\caption{The line plot shows the frequency response of SAR metrics (Min Local SAR, Mass-Averaged SAR, Max Local SAR) for Tendon\_Ligament in the all scenarios scenario for Thelonious.}
\label{fig:10_44}
\end{figure}

\subsection{Tissue Frequency Response Testis All}

The frequency response for Testis (Duke) can be found in Figure~\ref{fig:10_45}. \\\\

\begin{figure}[H]
\centering
\includegraphics[width=\columnwidth]{../../../plots/far_field/duke/tissue_analysis/tissue_frequency_response_Testis_all.pdf}
\caption{The line plot shows the frequency response of SAR metrics (Min Local SAR, Mass-Averaged SAR, Max Local SAR) for Testis in the all scenarios scenario for Duke.}
\label{fig:10_45}
\end{figure}

The frequency response for Testis (Thelonious) can be found in Figure~\ref{fig:10_46}. \\\\

\begin{figure}[H]
\centering
\includegraphics[width=\columnwidth]{../../../plots/far_field/thelonious/tissue_analysis/tissue_frequency_response_Testis_all.pdf}
\caption{The line plot shows the frequency response of SAR metrics (Min Local SAR, Mass-Averaged SAR, Max Local SAR) for Testis in the all scenarios scenario for Thelonious.}
\label{fig:10_46}
\end{figure}

\subsection{Tissue Frequency Response Thyroid Gland All}

The frequency response for Thyroid gland (Duke) can be found in Figure~\ref{fig:10_47}. \\\\

\begin{figure}[H]
\centering
\includegraphics[width=\columnwidth]{../../../plots/far_field/duke/tissue_analysis/tissue_frequency_response_Thyroid_gland_all.pdf}
\caption{The line plot shows the frequency response of SAR metrics (Min Local SAR, Mass-Averaged SAR, Max Local SAR) for Thyroid\_gland in the all scenarios scenario for Duke.}
\label{fig:10_47}
\end{figure}

\subsection{Tissue Frequency Response Vein All}

The frequency response for Vein (Ella) can be found in Figure~\ref{fig:10_48}. \\\\

\begin{figure}[H]
\centering
\includegraphics[width=\columnwidth]{../../../plots/far_field/ella/tissue_analysis/tissue_frequency_response_Vein_all.pdf}
\caption{The line plot shows the frequency response of SAR metrics (Min Local SAR, Mass-Averaged SAR, Max Local SAR) for Vein in the all scenarios scenario for Ella.}
\label{fig:10_48}
\end{figure}

\newpage

\section{Cumulative Distribution Functions}

\subsection{Brain Direction All All MHz}

The cumulative distribution function for Brain SAR (Duke) can be found in Figure~\ref{fig:11_1}. \\\\

\begin{figure}[H]
\centering
\includegraphics[width=\columnwidth]{../../../plots/far_field/duke/cdf/cdf__brain_direction_all_allMHz.pdf}
\caption{The cumulative distribution function (CDF) plot shows the probability distribution of Brain SAR values for Duke. Grouped by direction.}
\label{fig:11_1}
\end{figure}

The cumulative distribution function for Brain SAR (Eartha) can be found in Figure~\ref{fig:11_2}. \\\\

\begin{figure}[H]
\centering
\includegraphics[width=\columnwidth]{../../../plots/far_field/eartha/cdf/cdf__brain_direction_all_allMHz.pdf}
\caption{The cumulative distribution function (CDF) plot shows the probability distribution of Brain SAR values for Eartha. Grouped by direction.}
\label{fig:11_2}
\end{figure}

The cumulative distribution function for Brain SAR (Ella) can be found in Figure~\ref{fig:11_3}. \\\\

\begin{figure}[H]
\centering
\includegraphics[width=\columnwidth]{../../../plots/far_field/ella/cdf/cdf__brain_direction_all_allMHz.pdf}
\caption{The cumulative distribution function (CDF) plot shows the probability distribution of Brain SAR values for Ella. Grouped by direction.}
\label{fig:11_3}
\end{figure}

The cumulative distribution function for Brain SAR (Thelonious) can be found in Figure~\ref{fig:11_4}. \\\\

\begin{figure}[H]
\centering
\includegraphics[width=\columnwidth]{../../../plots/far_field/thelonious/cdf/cdf__brain_direction_all_allMHz.pdf}
\caption{The cumulative distribution function (CDF) plot shows the probability distribution of Brain SAR values for Thelonious. Grouped by direction.}
\label{fig:11_4}
\end{figure}

\subsection{Brain Frequency Mhz All All MHz}

The cumulative distribution function for Brain SAR (Duke) can be found in Figure~\ref{fig:11_5}. \\\\

\begin{figure}[H]
\centering
\includegraphics[width=\columnwidth]{../../../plots/far_field/duke/cdf/cdf__brain_frequency_mhz_all_allMHz.pdf}
\caption{The cumulative distribution function (CDF) plot shows the probability distribution of Brain SAR values for Duke. Grouped by frequency.}
\label{fig:11_5}
\end{figure}

The cumulative distribution function for Brain SAR (Eartha) can be found in Figure~\ref{fig:11_6}. \\\\

\begin{figure}[H]
\centering
\includegraphics[width=\columnwidth]{../../../plots/far_field/eartha/cdf/cdf__brain_frequency_mhz_all_allMHz.pdf}
\caption{The cumulative distribution function (CDF) plot shows the probability distribution of Brain SAR values for Eartha. Grouped by frequency.}
\label{fig:11_6}
\end{figure}

The cumulative distribution function for Brain SAR (Ella) can be found in Figure~\ref{fig:11_7}. \\\\

\begin{figure}[H]
\centering
\includegraphics[width=\columnwidth]{../../../plots/far_field/ella/cdf/cdf__brain_frequency_mhz_all_allMHz.pdf}
\caption{The cumulative distribution function (CDF) plot shows the probability distribution of Brain SAR values for Ella. Grouped by frequency.}
\label{fig:11_7}
\end{figure}

The cumulative distribution function for Brain SAR (Thelonious) can be found in Figure~\ref{fig:11_8}. \\\\

\begin{figure}[H]
\centering
\includegraphics[width=\columnwidth]{../../../plots/far_field/thelonious/cdf/cdf__brain_frequency_mhz_all_allMHz.pdf}
\caption{The cumulative distribution function (CDF) plot shows the probability distribution of Brain SAR values for Thelonious. Grouped by frequency.}
\label{fig:11_8}
\end{figure}

\subsection{Brain Polarization All All MHz}

The cumulative distribution function for Brain SAR (Duke) can be found in Figure~\ref{fig:11_9}. \\\\

\begin{figure}[H]
\centering
\includegraphics[width=\columnwidth]{../../../plots/far_field/duke/cdf/cdf__brain_polarization_all_allMHz.pdf}
\caption{The cumulative distribution function (CDF) plot shows the probability distribution of Brain SAR values for Duke. Grouped by polarization.}
\label{fig:11_9}
\end{figure}

The cumulative distribution function for Brain SAR (Eartha) can be found in Figure~\ref{fig:11_10}. \\\\

\begin{figure}[H]
\centering
\includegraphics[width=\columnwidth]{../../../plots/far_field/eartha/cdf/cdf__brain_polarization_all_allMHz.pdf}
\caption{The cumulative distribution function (CDF) plot shows the probability distribution of Brain SAR values for Eartha. Grouped by polarization.}
\label{fig:11_10}
\end{figure}

The cumulative distribution function for Brain SAR (Ella) can be found in Figure~\ref{fig:11_11}. \\\\

\begin{figure}[H]
\centering
\includegraphics[width=\columnwidth]{../../../plots/far_field/ella/cdf/cdf__brain_polarization_all_allMHz.pdf}
\caption{The cumulative distribution function (CDF) plot shows the probability distribution of Brain SAR values for Ella. Grouped by polarization.}
\label{fig:11_11}
\end{figure}

The cumulative distribution function for Brain SAR (Thelonious) can be found in Figure~\ref{fig:11_12}. \\\\

\begin{figure}[H]
\centering
\includegraphics[width=\columnwidth]{../../../plots/far_field/thelonious/cdf/cdf__brain_polarization_all_allMHz.pdf}
\caption{The cumulative distribution function (CDF) plot shows the probability distribution of Brain SAR values for Thelonious. Grouped by polarization.}
\label{fig:11_12}
\end{figure}

\subsection{Eyes Direction All All MHz}

The cumulative distribution function for Eyes SAR (Duke) can be found in Figure~\ref{fig:11_13}. \\\\

\begin{figure}[H]
\centering
\includegraphics[width=\columnwidth]{../../../plots/far_field/duke/cdf/cdf__eyes_direction_all_allMHz.pdf}
\caption{The cumulative distribution function (CDF) plot shows the probability distribution of Eyes SAR values for Duke. Grouped by direction.}
\label{fig:11_13}
\end{figure}

The cumulative distribution function for Eyes SAR (Eartha) can be found in Figure~\ref{fig:11_14}. \\\\

\begin{figure}[H]
\centering
\includegraphics[width=\columnwidth]{../../../plots/far_field/eartha/cdf/cdf__eyes_direction_all_allMHz.pdf}
\caption{The cumulative distribution function (CDF) plot shows the probability distribution of Eyes SAR values for Eartha. Grouped by direction.}
\label{fig:11_14}
\end{figure}

The cumulative distribution function for Eyes SAR (Ella) can be found in Figure~\ref{fig:11_15}. \\\\

\begin{figure}[H]
\centering
\includegraphics[width=\columnwidth]{../../../plots/far_field/ella/cdf/cdf__eyes_direction_all_allMHz.pdf}
\caption{The cumulative distribution function (CDF) plot shows the probability distribution of Eyes SAR values for Ella. Grouped by direction.}
\label{fig:11_15}
\end{figure}

The cumulative distribution function for Eyes SAR (Thelonious) can be found in Figure~\ref{fig:11_16}. \\\\

\begin{figure}[H]
\centering
\includegraphics[width=\columnwidth]{../../../plots/far_field/thelonious/cdf/cdf__eyes_direction_all_allMHz.pdf}
\caption{The cumulative distribution function (CDF) plot shows the probability distribution of Eyes SAR values for Thelonious. Grouped by direction.}
\label{fig:11_16}
\end{figure}

\subsection{Eyes Frequency Mhz All All MHz}

The cumulative distribution function for Eyes SAR (Duke) can be found in Figure~\ref{fig:11_17}. \\\\

\begin{figure}[H]
\centering
\includegraphics[width=\columnwidth]{../../../plots/far_field/duke/cdf/cdf__eyes_frequency_mhz_all_allMHz.pdf}
\caption{The cumulative distribution function (CDF) plot shows the probability distribution of Eyes SAR values for Duke. Grouped by frequency.}
\label{fig:11_17}
\end{figure}

The cumulative distribution function for Eyes SAR (Eartha) can be found in Figure~\ref{fig:11_18}. \\\\

\begin{figure}[H]
\centering
\includegraphics[width=\columnwidth]{../../../plots/far_field/eartha/cdf/cdf__eyes_frequency_mhz_all_allMHz.pdf}
\caption{The cumulative distribution function (CDF) plot shows the probability distribution of Eyes SAR values for Eartha. Grouped by frequency.}
\label{fig:11_18}
\end{figure}

The cumulative distribution function for Eyes SAR (Ella) can be found in Figure~\ref{fig:11_19}. \\\\

\begin{figure}[H]
\centering
\includegraphics[width=\columnwidth]{../../../plots/far_field/ella/cdf/cdf__eyes_frequency_mhz_all_allMHz.pdf}
\caption{The cumulative distribution function (CDF) plot shows the probability distribution of Eyes SAR values for Ella. Grouped by frequency.}
\label{fig:11_19}
\end{figure}

The cumulative distribution function for Eyes SAR (Thelonious) can be found in Figure~\ref{fig:11_20}. \\\\

\begin{figure}[H]
\centering
\includegraphics[width=\columnwidth]{../../../plots/far_field/thelonious/cdf/cdf__eyes_frequency_mhz_all_allMHz.pdf}
\caption{The cumulative distribution function (CDF) plot shows the probability distribution of Eyes SAR values for Thelonious. Grouped by frequency.}
\label{fig:11_20}
\end{figure}

\subsection{Eyes Polarization All All MHz}

The cumulative distribution function for Eyes SAR (Duke) can be found in Figure~\ref{fig:11_21}. \\\\

\begin{figure}[H]
\centering
\includegraphics[width=\columnwidth]{../../../plots/far_field/duke/cdf/cdf__eyes_polarization_all_allMHz.pdf}
\caption{The cumulative distribution function (CDF) plot shows the probability distribution of Eyes SAR values for Duke. Grouped by polarization.}
\label{fig:11_21}
\end{figure}

The cumulative distribution function for Eyes SAR (Eartha) can be found in Figure~\ref{fig:11_22}. \\\\

\begin{figure}[H]
\centering
\includegraphics[width=\columnwidth]{../../../plots/far_field/eartha/cdf/cdf__eyes_polarization_all_allMHz.pdf}
\caption{The cumulative distribution function (CDF) plot shows the probability distribution of Eyes SAR values for Eartha. Grouped by polarization.}
\label{fig:11_22}
\end{figure}

The cumulative distribution function for Eyes SAR (Ella) can be found in Figure~\ref{fig:11_23}. \\\\

\begin{figure}[H]
\centering
\includegraphics[width=\columnwidth]{../../../plots/far_field/ella/cdf/cdf__eyes_polarization_all_allMHz.pdf}
\caption{The cumulative distribution function (CDF) plot shows the probability distribution of Eyes SAR values for Ella. Grouped by polarization.}
\label{fig:11_23}
\end{figure}

The cumulative distribution function for Eyes SAR (Thelonious) can be found in Figure~\ref{fig:11_24}. \\\\

\begin{figure}[H]
\centering
\includegraphics[width=\columnwidth]{../../../plots/far_field/thelonious/cdf/cdf__eyes_polarization_all_allMHz.pdf}
\caption{The cumulative distribution function (CDF) plot shows the probability distribution of Eyes SAR values for Thelonious. Grouped by polarization.}
\label{fig:11_24}
\end{figure}

\subsection{Genitals Direction All All MHz}

The cumulative distribution function for Genitals SAR (Duke) can be found in Figure~\ref{fig:11_25}. \\\\

\begin{figure}[H]
\centering
\includegraphics[width=\columnwidth]{../../../plots/far_field/duke/cdf/cdf__genitals_direction_all_allMHz.pdf}
\caption{The cumulative distribution function (CDF) plot shows the probability distribution of Genitals SAR values for Duke. Grouped by direction.}
\label{fig:11_25}
\end{figure}

The cumulative distribution function for Genitals SAR (Eartha) can be found in Figure~\ref{fig:11_26}. \\\\

\begin{figure}[H]
\centering
\includegraphics[width=\columnwidth]{../../../plots/far_field/eartha/cdf/cdf__genitals_direction_all_allMHz.pdf}
\caption{The cumulative distribution function (CDF) plot shows the probability distribution of Genitals SAR values for Eartha. Grouped by direction.}
\label{fig:11_26}
\end{figure}

The cumulative distribution function for Genitals SAR (Ella) can be found in Figure~\ref{fig:11_27}. \\\\

\begin{figure}[H]
\centering
\includegraphics[width=\columnwidth]{../../../plots/far_field/ella/cdf/cdf__genitals_direction_all_allMHz.pdf}
\caption{The cumulative distribution function (CDF) plot shows the probability distribution of Genitals SAR values for Ella. Grouped by direction.}
\label{fig:11_27}
\end{figure}

The cumulative distribution function for Genitals SAR (Thelonious) can be found in Figure~\ref{fig:11_28}. \\\\

\begin{figure}[H]
\centering
\includegraphics[width=\columnwidth]{../../../plots/far_field/thelonious/cdf/cdf__genitals_direction_all_allMHz.pdf}
\caption{The cumulative distribution function (CDF) plot shows the probability distribution of Genitals SAR values for Thelonious. Grouped by direction.}
\label{fig:11_28}
\end{figure}

\subsection{Genitals Frequency Mhz All All MHz}

The cumulative distribution function for Genitals SAR (Duke) can be found in Figure~\ref{fig:11_29}. \\\\

\begin{figure}[H]
\centering
\includegraphics[width=\columnwidth]{../../../plots/far_field/duke/cdf/cdf__genitals_frequency_mhz_all_allMHz.pdf}
\caption{The cumulative distribution function (CDF) plot shows the probability distribution of Genitals SAR values for Duke. Grouped by frequency.}
\label{fig:11_29}
\end{figure}

The cumulative distribution function for Genitals SAR (Eartha) can be found in Figure~\ref{fig:11_30}. \\\\

\begin{figure}[H]
\centering
\includegraphics[width=\columnwidth]{../../../plots/far_field/eartha/cdf/cdf__genitals_frequency_mhz_all_allMHz.pdf}
\caption{The cumulative distribution function (CDF) plot shows the probability distribution of Genitals SAR values for Eartha. Grouped by frequency.}
\label{fig:11_30}
\end{figure}

The cumulative distribution function for Genitals SAR (Ella) can be found in Figure~\ref{fig:11_31}. \\\\

\begin{figure}[H]
\centering
\includegraphics[width=\columnwidth]{../../../plots/far_field/ella/cdf/cdf__genitals_frequency_mhz_all_allMHz.pdf}
\caption{The cumulative distribution function (CDF) plot shows the probability distribution of Genitals SAR values for Ella. Grouped by frequency.}
\label{fig:11_31}
\end{figure}

The cumulative distribution function for Genitals SAR (Thelonious) can be found in Figure~\ref{fig:11_32}. \\\\

\begin{figure}[H]
\centering
\includegraphics[width=\columnwidth]{../../../plots/far_field/thelonious/cdf/cdf__genitals_frequency_mhz_all_allMHz.pdf}
\caption{The cumulative distribution function (CDF) plot shows the probability distribution of Genitals SAR values for Thelonious. Grouped by frequency.}
\label{fig:11_32}
\end{figure}

\subsection{Genitals Polarization All All MHz}

The cumulative distribution function for Genitals SAR (Duke) can be found in Figure~\ref{fig:11_33}. \\\\

\begin{figure}[H]
\centering
\includegraphics[width=\columnwidth]{../../../plots/far_field/duke/cdf/cdf__genitals_polarization_all_allMHz.pdf}
\caption{The cumulative distribution function (CDF) plot shows the probability distribution of Genitals SAR values for Duke. Grouped by polarization.}
\label{fig:11_33}
\end{figure}

The cumulative distribution function for Genitals SAR (Eartha) can be found in Figure~\ref{fig:11_34}. \\\\

\begin{figure}[H]
\centering
\includegraphics[width=\columnwidth]{../../../plots/far_field/eartha/cdf/cdf__genitals_polarization_all_allMHz.pdf}
\caption{The cumulative distribution function (CDF) plot shows the probability distribution of Genitals SAR values for Eartha. Grouped by polarization.}
\label{fig:11_34}
\end{figure}

The cumulative distribution function for Genitals SAR (Ella) can be found in Figure~\ref{fig:11_35}. \\\\

\begin{figure}[H]
\centering
\includegraphics[width=\columnwidth]{../../../plots/far_field/ella/cdf/cdf__genitals_polarization_all_allMHz.pdf}
\caption{The cumulative distribution function (CDF) plot shows the probability distribution of Genitals SAR values for Ella. Grouped by polarization.}
\label{fig:11_35}
\end{figure}

The cumulative distribution function for Genitals SAR (Thelonious) can be found in Figure~\ref{fig:11_36}. \\\\

\begin{figure}[H]
\centering
\includegraphics[width=\columnwidth]{../../../plots/far_field/thelonious/cdf/cdf__genitals_polarization_all_allMHz.pdf}
\caption{The cumulative distribution function (CDF) plot shows the probability distribution of Genitals SAR values for Thelonious. Grouped by polarization.}
\label{fig:11_36}
\end{figure}

\subsection{Peak SAR Direction All All MHz}

The cumulative distribution function for Peak SAR 10g (Duke) can be found in Figure~\ref{fig:11_37}. \\\\

\begin{figure}[H]
\centering
\includegraphics[width=\columnwidth]{../../../plots/far_field/duke/cdf/cdf_peak_sar_direction_all_allMHz.pdf}
\caption{The cumulative distribution function (CDF) plot shows the probability distribution of Peak SAR (10g) values for Duke. Grouped by direction.}
\label{fig:11_37}
\end{figure}

The cumulative distribution function for Peak SAR 10g (Eartha) can be found in Figure~\ref{fig:11_38}. \\\\

\begin{figure}[H]
\centering
\includegraphics[width=\columnwidth]{../../../plots/far_field/eartha/cdf/cdf_peak_sar_direction_all_allMHz.pdf}
\caption{The cumulative distribution function (CDF) plot shows the probability distribution of Peak SAR (10g) values for Eartha. Grouped by direction.}
\label{fig:11_38}
\end{figure}

The cumulative distribution function for Peak SAR 10g (Ella) can be found in Figure~\ref{fig:11_39}. \\\\

\begin{figure}[H]
\centering
\includegraphics[width=\columnwidth]{../../../plots/far_field/ella/cdf/cdf_peak_sar_direction_all_allMHz.pdf}
\caption{The cumulative distribution function (CDF) plot shows the probability distribution of Peak SAR (10g) values for Ella. Grouped by direction.}
\label{fig:11_39}
\end{figure}

The cumulative distribution function for Peak SAR 10g (Thelonious) can be found in Figure~\ref{fig:11_40}. \\\\

\begin{figure}[H]
\centering
\includegraphics[width=\columnwidth]{../../../plots/far_field/thelonious/cdf/cdf_peak_sar_direction_all_allMHz.pdf}
\caption{The cumulative distribution function (CDF) plot shows the probability distribution of Peak SAR (10g) values for Thelonious. Grouped by direction.}
\label{fig:11_40}
\end{figure}

\subsection{Peak SAR Frequency Mhz All All MHz}

The cumulative distribution function for Peak SAR 10g (Duke) can be found in Figure~\ref{fig:11_41}. \\\\

\begin{figure}[H]
\centering
\includegraphics[width=\columnwidth]{../../../plots/far_field/duke/cdf/cdf_peak_sar_frequency_mhz_all_allMHz.pdf}
\caption{The cumulative distribution function (CDF) plot shows the probability distribution of Peak SAR (10g) values for Duke. Grouped by frequency.}
\label{fig:11_41}
\end{figure}

The cumulative distribution function for Peak SAR 10g (Eartha) can be found in Figure~\ref{fig:11_42}. \\\\

\begin{figure}[H]
\centering
\includegraphics[width=\columnwidth]{../../../plots/far_field/eartha/cdf/cdf_peak_sar_frequency_mhz_all_allMHz.pdf}
\caption{The cumulative distribution function (CDF) plot shows the probability distribution of Peak SAR (10g) values for Eartha. Grouped by frequency.}
\label{fig:11_42}
\end{figure}

The cumulative distribution function for Peak SAR 10g (Ella) can be found in Figure~\ref{fig:11_43}. \\\\

\begin{figure}[H]
\centering
\includegraphics[width=\columnwidth]{../../../plots/far_field/ella/cdf/cdf_peak_sar_frequency_mhz_all_allMHz.pdf}
\caption{The cumulative distribution function (CDF) plot shows the probability distribution of Peak SAR (10g) values for Ella. Grouped by frequency.}
\label{fig:11_43}
\end{figure}

The cumulative distribution function for Peak SAR 10g (Thelonious) can be found in Figure~\ref{fig:11_44}. \\\\

\begin{figure}[H]
\centering
\includegraphics[width=\columnwidth]{../../../plots/far_field/thelonious/cdf/cdf_peak_sar_frequency_mhz_all_allMHz.pdf}
\caption{The cumulative distribution function (CDF) plot shows the probability distribution of Peak SAR (10g) values for Thelonious. Grouped by frequency.}
\label{fig:11_44}
\end{figure}

\subsection{Peak SAR Polarization All All MHz}

The cumulative distribution function for Peak SAR 10g (Duke) can be found in Figure~\ref{fig:11_45}. \\\\

\begin{figure}[H]
\centering
\includegraphics[width=\columnwidth]{../../../plots/far_field/duke/cdf/cdf_peak_sar_polarization_all_allMHz.pdf}
\caption{The cumulative distribution function (CDF) plot shows the probability distribution of Peak SAR (10g) values for Duke. Grouped by polarization.}
\label{fig:11_45}
\end{figure}

The cumulative distribution function for Peak SAR 10g (Eartha) can be found in Figure~\ref{fig:11_46}. \\\\

\begin{figure}[H]
\centering
\includegraphics[width=\columnwidth]{../../../plots/far_field/eartha/cdf/cdf_peak_sar_polarization_all_allMHz.pdf}
\caption{The cumulative distribution function (CDF) plot shows the probability distribution of Peak SAR (10g) values for Eartha. Grouped by polarization.}
\label{fig:11_46}
\end{figure}

The cumulative distribution function for Peak SAR 10g (Ella) can be found in Figure~\ref{fig:11_47}. \\\\

\begin{figure}[H]
\centering
\includegraphics[width=\columnwidth]{../../../plots/far_field/ella/cdf/cdf_peak_sar_polarization_all_allMHz.pdf}
\caption{The cumulative distribution function (CDF) plot shows the probability distribution of Peak SAR (10g) values for Ella. Grouped by polarization.}
\label{fig:11_47}
\end{figure}

The cumulative distribution function for Peak SAR 10g (Thelonious) can be found in Figure~\ref{fig:11_48}. \\\\

\begin{figure}[H]
\centering
\includegraphics[width=\columnwidth]{../../../plots/far_field/thelonious/cdf/cdf_peak_sar_polarization_all_allMHz.pdf}
\caption{The cumulative distribution function (CDF) plot shows the probability distribution of Peak SAR (10g) values for Thelonious. Grouped by polarization.}
\label{fig:11_48}
\end{figure}

\subsection{Ps10g Brain Direction All All MHz}

The cumulative distribution function for psSAR10g Brain (Duke) can be found in Figure~\ref{fig:11_49}. \\\\

\begin{figure}[H]
\centering
\includegraphics[width=\columnwidth]{../../../plots/far_field/duke/cdf/cdf_ps10g_brain_direction_all_allMHz.pdf}
\caption{The cumulative distribution function (CDF) plot shows the probability distribution of psSAR10g Brain values for Duke. Grouped by direction.}
\label{fig:11_49}
\end{figure}

The cumulative distribution function for psSAR10g Brain (Eartha) can be found in Figure~\ref{fig:11_50}. \\\\

\begin{figure}[H]
\centering
\includegraphics[width=\columnwidth]{../../../plots/far_field/eartha/cdf/cdf_ps10g_brain_direction_all_allMHz.pdf}
\caption{The cumulative distribution function (CDF) plot shows the probability distribution of psSAR10g Brain values for Eartha. Grouped by direction.}
\label{fig:11_50}
\end{figure}

The cumulative distribution function for psSAR10g Brain (Ella) can be found in Figure~\ref{fig:11_51}. \\\\

\begin{figure}[H]
\centering
\includegraphics[width=\columnwidth]{../../../plots/far_field/ella/cdf/cdf_ps10g_brain_direction_all_allMHz.pdf}
\caption{The cumulative distribution function (CDF) plot shows the probability distribution of psSAR10g Brain values for Ella. Grouped by direction.}
\label{fig:11_51}
\end{figure}

The cumulative distribution function for psSAR10g Brain (Thelonious) can be found in Figure~\ref{fig:11_52}. \\\\

\begin{figure}[H]
\centering
\includegraphics[width=\columnwidth]{../../../plots/far_field/thelonious/cdf/cdf_ps10g_brain_direction_all_allMHz.pdf}
\caption{The cumulative distribution function (CDF) plot shows the probability distribution of psSAR10g Brain values for Thelonious. Grouped by direction.}
\label{fig:11_52}
\end{figure}

\subsection{Ps10g Brain Frequency Mhz All All MHz}

The cumulative distribution function for psSAR10g Brain (Duke) can be found in Figure~\ref{fig:11_53}. \\\\

\begin{figure}[H]
\centering
\includegraphics[width=\columnwidth]{../../../plots/far_field/duke/cdf/cdf_ps10g_brain_frequency_mhz_all_allMHz.pdf}
\caption{The cumulative distribution function (CDF) plot shows the probability distribution of psSAR10g Brain values for Duke. Grouped by frequency.}
\label{fig:11_53}
\end{figure}

The cumulative distribution function for psSAR10g Brain (Eartha) can be found in Figure~\ref{fig:11_54}. \\\\

\begin{figure}[H]
\centering
\includegraphics[width=\columnwidth]{../../../plots/far_field/eartha/cdf/cdf_ps10g_brain_frequency_mhz_all_allMHz.pdf}
\caption{The cumulative distribution function (CDF) plot shows the probability distribution of psSAR10g Brain values for Eartha. Grouped by frequency.}
\label{fig:11_54}
\end{figure}

The cumulative distribution function for psSAR10g Brain (Ella) can be found in Figure~\ref{fig:11_55}. \\\\

\begin{figure}[H]
\centering
\includegraphics[width=\columnwidth]{../../../plots/far_field/ella/cdf/cdf_ps10g_brain_frequency_mhz_all_allMHz.pdf}
\caption{The cumulative distribution function (CDF) plot shows the probability distribution of psSAR10g Brain values for Ella. Grouped by frequency.}
\label{fig:11_55}
\end{figure}

The cumulative distribution function for psSAR10g Brain (Thelonious) can be found in Figure~\ref{fig:11_56}. \\\\

\begin{figure}[H]
\centering
\includegraphics[width=\columnwidth]{../../../plots/far_field/thelonious/cdf/cdf_ps10g_brain_frequency_mhz_all_allMHz.pdf}
\caption{The cumulative distribution function (CDF) plot shows the probability distribution of psSAR10g Brain values for Thelonious. Grouped by frequency.}
\label{fig:11_56}
\end{figure}

\subsection{Ps10g Brain Polarization All All MHz}

The cumulative distribution function for psSAR10g Brain (Duke) can be found in Figure~\ref{fig:11_57}. \\\\

\begin{figure}[H]
\centering
\includegraphics[width=\columnwidth]{../../../plots/far_field/duke/cdf/cdf_ps10g_brain_polarization_all_allMHz.pdf}
\caption{The cumulative distribution function (CDF) plot shows the probability distribution of psSAR10g Brain values for Duke. Grouped by polarization.}
\label{fig:11_57}
\end{figure}

The cumulative distribution function for psSAR10g Brain (Eartha) can be found in Figure~\ref{fig:11_58}. \\\\

\begin{figure}[H]
\centering
\includegraphics[width=\columnwidth]{../../../plots/far_field/eartha/cdf/cdf_ps10g_brain_polarization_all_allMHz.pdf}
\caption{The cumulative distribution function (CDF) plot shows the probability distribution of psSAR10g Brain values for Eartha. Grouped by polarization.}
\label{fig:11_58}
\end{figure}

The cumulative distribution function for psSAR10g Brain (Ella) can be found in Figure~\ref{fig:11_59}. \\\\

\begin{figure}[H]
\centering
\includegraphics[width=\columnwidth]{../../../plots/far_field/ella/cdf/cdf_ps10g_brain_polarization_all_allMHz.pdf}
\caption{The cumulative distribution function (CDF) plot shows the probability distribution of psSAR10g Brain values for Ella. Grouped by polarization.}
\label{fig:11_59}
\end{figure}

The cumulative distribution function for psSAR10g Brain (Thelonious) can be found in Figure~\ref{fig:11_60}. \\\\

\begin{figure}[H]
\centering
\includegraphics[width=\columnwidth]{../../../plots/far_field/thelonious/cdf/cdf_ps10g_brain_polarization_all_allMHz.pdf}
\caption{The cumulative distribution function (CDF) plot shows the probability distribution of psSAR10g Brain values for Thelonious. Grouped by polarization.}
\label{fig:11_60}
\end{figure}

\subsection{Ps10g Eyes Direction All All MHz}

The cumulative distribution function for psSAR10g Eyes (Duke) can be found in Figure~\ref{fig:11_61}. \\\\

\begin{figure}[H]
\centering
\includegraphics[width=\columnwidth]{../../../plots/far_field/duke/cdf/cdf_ps10g_eyes_direction_all_allMHz.pdf}
\caption{The cumulative distribution function (CDF) plot shows the probability distribution of psSAR10g Eyes values for Duke. Grouped by direction.}
\label{fig:11_61}
\end{figure}

The cumulative distribution function for psSAR10g Eyes (Eartha) can be found in Figure~\ref{fig:11_62}. \\\\

\begin{figure}[H]
\centering
\includegraphics[width=\columnwidth]{../../../plots/far_field/eartha/cdf/cdf_ps10g_eyes_direction_all_allMHz.pdf}
\caption{The cumulative distribution function (CDF) plot shows the probability distribution of psSAR10g Eyes values for Eartha. Grouped by direction.}
\label{fig:11_62}
\end{figure}

The cumulative distribution function for psSAR10g Eyes (Ella) can be found in Figure~\ref{fig:11_63}. \\\\

\begin{figure}[H]
\centering
\includegraphics[width=\columnwidth]{../../../plots/far_field/ella/cdf/cdf_ps10g_eyes_direction_all_allMHz.pdf}
\caption{The cumulative distribution function (CDF) plot shows the probability distribution of psSAR10g Eyes values for Ella. Grouped by direction.}
\label{fig:11_63}
\end{figure}

The cumulative distribution function for psSAR10g Eyes (Thelonious) can be found in Figure~\ref{fig:11_64}. \\\\

\begin{figure}[H]
\centering
\includegraphics[width=\columnwidth]{../../../plots/far_field/thelonious/cdf/cdf_ps10g_eyes_direction_all_allMHz.pdf}
\caption{The cumulative distribution function (CDF) plot shows the probability distribution of psSAR10g Eyes values for Thelonious. Grouped by direction.}
\label{fig:11_64}
\end{figure}

\subsection{Ps10g Eyes Frequency Mhz All All MHz}

The cumulative distribution function for psSAR10g Eyes (Duke) can be found in Figure~\ref{fig:11_65}. \\\\

\begin{figure}[H]
\centering
\includegraphics[width=\columnwidth]{../../../plots/far_field/duke/cdf/cdf_ps10g_eyes_frequency_mhz_all_allMHz.pdf}
\caption{The cumulative distribution function (CDF) plot shows the probability distribution of psSAR10g Eyes values for Duke. Grouped by frequency.}
\label{fig:11_65}
\end{figure}

The cumulative distribution function for psSAR10g Eyes (Eartha) can be found in Figure~\ref{fig:11_66}. \\\\

\begin{figure}[H]
\centering
\includegraphics[width=\columnwidth]{../../../plots/far_field/eartha/cdf/cdf_ps10g_eyes_frequency_mhz_all_allMHz.pdf}
\caption{The cumulative distribution function (CDF) plot shows the probability distribution of psSAR10g Eyes values for Eartha. Grouped by frequency.}
\label{fig:11_66}
\end{figure}

The cumulative distribution function for psSAR10g Eyes (Ella) can be found in Figure~\ref{fig:11_67}. \\\\

\begin{figure}[H]
\centering
\includegraphics[width=\columnwidth]{../../../plots/far_field/ella/cdf/cdf_ps10g_eyes_frequency_mhz_all_allMHz.pdf}
\caption{The cumulative distribution function (CDF) plot shows the probability distribution of psSAR10g Eyes values for Ella. Grouped by frequency.}
\label{fig:11_67}
\end{figure}

The cumulative distribution function for psSAR10g Eyes (Thelonious) can be found in Figure~\ref{fig:11_68}. \\\\

\begin{figure}[H]
\centering
\includegraphics[width=\columnwidth]{../../../plots/far_field/thelonious/cdf/cdf_ps10g_eyes_frequency_mhz_all_allMHz.pdf}
\caption{The cumulative distribution function (CDF) plot shows the probability distribution of psSAR10g Eyes values for Thelonious. Grouped by frequency.}
\label{fig:11_68}
\end{figure}

\subsection{Ps10g Eyes Polarization All All MHz}

The cumulative distribution function for psSAR10g Eyes (Duke) can be found in Figure~\ref{fig:11_69}. \\\\

\begin{figure}[H]
\centering
\includegraphics[width=\columnwidth]{../../../plots/far_field/duke/cdf/cdf_ps10g_eyes_polarization_all_allMHz.pdf}
\caption{The cumulative distribution function (CDF) plot shows the probability distribution of psSAR10g Eyes values for Duke. Grouped by polarization.}
\label{fig:11_69}
\end{figure}

The cumulative distribution function for psSAR10g Eyes (Eartha) can be found in Figure~\ref{fig:11_70}. \\\\

\begin{figure}[H]
\centering
\includegraphics[width=\columnwidth]{../../../plots/far_field/eartha/cdf/cdf_ps10g_eyes_polarization_all_allMHz.pdf}
\caption{The cumulative distribution function (CDF) plot shows the probability distribution of psSAR10g Eyes values for Eartha. Grouped by polarization.}
\label{fig:11_70}
\end{figure}

The cumulative distribution function for psSAR10g Eyes (Ella) can be found in Figure~\ref{fig:11_71}. \\\\

\begin{figure}[H]
\centering
\includegraphics[width=\columnwidth]{../../../plots/far_field/ella/cdf/cdf_ps10g_eyes_polarization_all_allMHz.pdf}
\caption{The cumulative distribution function (CDF) plot shows the probability distribution of psSAR10g Eyes values for Ella. Grouped by polarization.}
\label{fig:11_71}
\end{figure}

The cumulative distribution function for psSAR10g Eyes (Thelonious) can be found in Figure~\ref{fig:11_72}. \\\\

\begin{figure}[H]
\centering
\includegraphics[width=\columnwidth]{../../../plots/far_field/thelonious/cdf/cdf_ps10g_eyes_polarization_all_allMHz.pdf}
\caption{The cumulative distribution function (CDF) plot shows the probability distribution of psSAR10g Eyes values for Thelonious. Grouped by polarization.}
\label{fig:11_72}
\end{figure}

\subsection{Ps10g Genitals Direction All All MHz}

The cumulative distribution function for psSAR10g Genitals (Duke) can be found in Figure~\ref{fig:11_73}. \\\\

\begin{figure}[H]
\centering
\includegraphics[width=\columnwidth]{../../../plots/far_field/duke/cdf/cdf_ps10g_genitals_direction_all_allMHz.pdf}
\caption{The cumulative distribution function (CDF) plot shows the probability distribution of psSAR10g Genitals values for Duke. Grouped by direction.}
\label{fig:11_73}
\end{figure}

The cumulative distribution function for psSAR10g Genitals (Eartha) can be found in Figure~\ref{fig:11_74}. \\\\

\begin{figure}[H]
\centering
\includegraphics[width=\columnwidth]{../../../plots/far_field/eartha/cdf/cdf_ps10g_genitals_direction_all_allMHz.pdf}
\caption{The cumulative distribution function (CDF) plot shows the probability distribution of psSAR10g Genitals values for Eartha. Grouped by direction.}
\label{fig:11_74}
\end{figure}

The cumulative distribution function for psSAR10g Genitals (Ella) can be found in Figure~\ref{fig:11_75}. \\\\

\begin{figure}[H]
\centering
\includegraphics[width=\columnwidth]{../../../plots/far_field/ella/cdf/cdf_ps10g_genitals_direction_all_allMHz.pdf}
\caption{The cumulative distribution function (CDF) plot shows the probability distribution of psSAR10g Genitals values for Ella. Grouped by direction.}
\label{fig:11_75}
\end{figure}

The cumulative distribution function for psSAR10g Genitals (Thelonious) can be found in Figure~\ref{fig:11_76}. \\\\

\begin{figure}[H]
\centering
\includegraphics[width=\columnwidth]{../../../plots/far_field/thelonious/cdf/cdf_ps10g_genitals_direction_all_allMHz.pdf}
\caption{The cumulative distribution function (CDF) plot shows the probability distribution of psSAR10g Genitals values for Thelonious. Grouped by direction.}
\label{fig:11_76}
\end{figure}

\subsection{Ps10g Genitals Frequency Mhz All All MHz}

The cumulative distribution function for psSAR10g Genitals (Duke) can be found in Figure~\ref{fig:11_77}. \\\\

\begin{figure}[H]
\centering
\includegraphics[width=\columnwidth]{../../../plots/far_field/duke/cdf/cdf_ps10g_genitals_frequency_mhz_all_allMHz.pdf}
\caption{The cumulative distribution function (CDF) plot shows the probability distribution of psSAR10g Genitals values for Duke. Grouped by frequency.}
\label{fig:11_77}
\end{figure}

The cumulative distribution function for psSAR10g Genitals (Eartha) can be found in Figure~\ref{fig:11_78}. \\\\

\begin{figure}[H]
\centering
\includegraphics[width=\columnwidth]{../../../plots/far_field/eartha/cdf/cdf_ps10g_genitals_frequency_mhz_all_allMHz.pdf}
\caption{The cumulative distribution function (CDF) plot shows the probability distribution of psSAR10g Genitals values for Eartha. Grouped by frequency.}
\label{fig:11_78}
\end{figure}

The cumulative distribution function for psSAR10g Genitals (Ella) can be found in Figure~\ref{fig:11_79}. \\\\

\begin{figure}[H]
\centering
\includegraphics[width=\columnwidth]{../../../plots/far_field/ella/cdf/cdf_ps10g_genitals_frequency_mhz_all_allMHz.pdf}
\caption{The cumulative distribution function (CDF) plot shows the probability distribution of psSAR10g Genitals values for Ella. Grouped by frequency.}
\label{fig:11_79}
\end{figure}

The cumulative distribution function for psSAR10g Genitals (Thelonious) can be found in Figure~\ref{fig:11_80}. \\\\

\begin{figure}[H]
\centering
\includegraphics[width=\columnwidth]{../../../plots/far_field/thelonious/cdf/cdf_ps10g_genitals_frequency_mhz_all_allMHz.pdf}
\caption{The cumulative distribution function (CDF) plot shows the probability distribution of psSAR10g Genitals values for Thelonious. Grouped by frequency.}
\label{fig:11_80}
\end{figure}

\subsection{Ps10g Genitals Polarization All All MHz}

The cumulative distribution function for psSAR10g Genitals (Duke) can be found in Figure~\ref{fig:11_81}. \\\\

\begin{figure}[H]
\centering
\includegraphics[width=\columnwidth]{../../../plots/far_field/duke/cdf/cdf_ps10g_genitals_polarization_all_allMHz.pdf}
\caption{The cumulative distribution function (CDF) plot shows the probability distribution of psSAR10g Genitals values for Duke. Grouped by polarization.}
\label{fig:11_81}
\end{figure}

The cumulative distribution function for psSAR10g Genitals (Eartha) can be found in Figure~\ref{fig:11_82}. \\\\

\begin{figure}[H]
\centering
\includegraphics[width=\columnwidth]{../../../plots/far_field/eartha/cdf/cdf_ps10g_genitals_polarization_all_allMHz.pdf}
\caption{The cumulative distribution function (CDF) plot shows the probability distribution of psSAR10g Genitals values for Eartha. Grouped by polarization.}
\label{fig:11_82}
\end{figure}

The cumulative distribution function for psSAR10g Genitals (Ella) can be found in Figure~\ref{fig:11_83}. \\\\

\begin{figure}[H]
\centering
\includegraphics[width=\columnwidth]{../../../plots/far_field/ella/cdf/cdf_ps10g_genitals_polarization_all_allMHz.pdf}
\caption{The cumulative distribution function (CDF) plot shows the probability distribution of psSAR10g Genitals values for Ella. Grouped by polarization.}
\label{fig:11_83}
\end{figure}

The cumulative distribution function for psSAR10g Genitals (Thelonious) can be found in Figure~\ref{fig:11_84}. \\\\

\begin{figure}[H]
\centering
\includegraphics[width=\columnwidth]{../../../plots/far_field/thelonious/cdf/cdf_ps10g_genitals_polarization_all_allMHz.pdf}
\caption{The cumulative distribution function (CDF) plot shows the probability distribution of psSAR10g Genitals values for Thelonious. Grouped by polarization.}
\label{fig:11_84}
\end{figure}

\subsection{Ps10g Skin Direction All All MHz}

The cumulative distribution function for psSAR10g Skin (Duke) can be found in Figure~\ref{fig:11_85}. \\\\

\begin{figure}[H]
\centering
\includegraphics[width=\columnwidth]{../../../plots/far_field/duke/cdf/cdf_ps10g_skin_direction_all_allMHz.pdf}
\caption{The cumulative distribution function (CDF) plot shows the probability distribution of psSAR10g Skin values for Duke. Grouped by direction.}
\label{fig:11_85}
\end{figure}

The cumulative distribution function for psSAR10g Skin (Eartha) can be found in Figure~\ref{fig:11_86}. \\\\

\begin{figure}[H]
\centering
\includegraphics[width=\columnwidth]{../../../plots/far_field/eartha/cdf/cdf_ps10g_skin_direction_all_allMHz.pdf}
\caption{The cumulative distribution function (CDF) plot shows the probability distribution of psSAR10g Skin values for Eartha. Grouped by direction.}
\label{fig:11_86}
\end{figure}

The cumulative distribution function for psSAR10g Skin (Ella) can be found in Figure~\ref{fig:11_87}. \\\\

\begin{figure}[H]
\centering
\includegraphics[width=\columnwidth]{../../../plots/far_field/ella/cdf/cdf_ps10g_skin_direction_all_allMHz.pdf}
\caption{The cumulative distribution function (CDF) plot shows the probability distribution of psSAR10g Skin values for Ella. Grouped by direction.}
\label{fig:11_87}
\end{figure}

The cumulative distribution function for psSAR10g Skin (Thelonious) can be found in Figure~\ref{fig:11_88}. \\\\

\begin{figure}[H]
\centering
\includegraphics[width=\columnwidth]{../../../plots/far_field/thelonious/cdf/cdf_ps10g_skin_direction_all_allMHz.pdf}
\caption{The cumulative distribution function (CDF) plot shows the probability distribution of psSAR10g Skin values for Thelonious. Grouped by direction.}
\label{fig:11_88}
\end{figure}

\subsection{Ps10g Skin Frequency Mhz All All MHz}

The cumulative distribution function for psSAR10g Skin (Duke) can be found in Figure~\ref{fig:11_89}. \\\\

\begin{figure}[H]
\centering
\includegraphics[width=\columnwidth]{../../../plots/far_field/duke/cdf/cdf_ps10g_skin_frequency_mhz_all_allMHz.pdf}
\caption{The cumulative distribution function (CDF) plot shows the probability distribution of psSAR10g Skin values for Duke. Grouped by frequency.}
\label{fig:11_89}
\end{figure}

The cumulative distribution function for psSAR10g Skin (Eartha) can be found in Figure~\ref{fig:11_90}. \\\\

\begin{figure}[H]
\centering
\includegraphics[width=\columnwidth]{../../../plots/far_field/eartha/cdf/cdf_ps10g_skin_frequency_mhz_all_allMHz.pdf}
\caption{The cumulative distribution function (CDF) plot shows the probability distribution of psSAR10g Skin values for Eartha. Grouped by frequency.}
\label{fig:11_90}
\end{figure}

The cumulative distribution function for psSAR10g Skin (Ella) can be found in Figure~\ref{fig:11_91}. \\\\

\begin{figure}[H]
\centering
\includegraphics[width=\columnwidth]{../../../plots/far_field/ella/cdf/cdf_ps10g_skin_frequency_mhz_all_allMHz.pdf}
\caption{The cumulative distribution function (CDF) plot shows the probability distribution of psSAR10g Skin values for Ella. Grouped by frequency.}
\label{fig:11_91}
\end{figure}

The cumulative distribution function for psSAR10g Skin (Thelonious) can be found in Figure~\ref{fig:11_92}. \\\\

\begin{figure}[H]
\centering
\includegraphics[width=\columnwidth]{../../../plots/far_field/thelonious/cdf/cdf_ps10g_skin_frequency_mhz_all_allMHz.pdf}
\caption{The cumulative distribution function (CDF) plot shows the probability distribution of psSAR10g Skin values for Thelonious. Grouped by frequency.}
\label{fig:11_92}
\end{figure}

\subsection{Ps10g Skin Polarization All All MHz}

The cumulative distribution function for psSAR10g Skin (Duke) can be found in Figure~\ref{fig:11_93}. \\\\

\begin{figure}[H]
\centering
\includegraphics[width=\columnwidth]{../../../plots/far_field/duke/cdf/cdf_ps10g_skin_polarization_all_allMHz.pdf}
\caption{The cumulative distribution function (CDF) plot shows the probability distribution of psSAR10g Skin values for Duke. Grouped by polarization.}
\label{fig:11_93}
\end{figure}

The cumulative distribution function for psSAR10g Skin (Eartha) can be found in Figure~\ref{fig:11_94}. \\\\

\begin{figure}[H]
\centering
\includegraphics[width=\columnwidth]{../../../plots/far_field/eartha/cdf/cdf_ps10g_skin_polarization_all_allMHz.pdf}
\caption{The cumulative distribution function (CDF) plot shows the probability distribution of psSAR10g Skin values for Eartha. Grouped by polarization.}
\label{fig:11_94}
\end{figure}

The cumulative distribution function for psSAR10g Skin (Ella) can be found in Figure~\ref{fig:11_95}. \\\\

\begin{figure}[H]
\centering
\includegraphics[width=\columnwidth]{../../../plots/far_field/ella/cdf/cdf_ps10g_skin_polarization_all_allMHz.pdf}
\caption{The cumulative distribution function (CDF) plot shows the probability distribution of psSAR10g Skin values for Ella. Grouped by polarization.}
\label{fig:11_95}
\end{figure}

The cumulative distribution function for psSAR10g Skin (Thelonious) can be found in Figure~\ref{fig:11_96}. \\\\

\begin{figure}[H]
\centering
\includegraphics[width=\columnwidth]{../../../plots/far_field/thelonious/cdf/cdf_ps10g_skin_polarization_all_allMHz.pdf}
\caption{The cumulative distribution function (CDF) plot shows the probability distribution of psSAR10g Skin values for Thelonious. Grouped by polarization.}
\label{fig:11_96}
\end{figure}

\subsection{Skin Direction All All MHz}

The cumulative distribution function for Skin SAR (Duke) can be found in Figure~\ref{fig:11_97}. \\\\

\begin{figure}[H]
\centering
\includegraphics[width=\columnwidth]{../../../plots/far_field/duke/cdf/cdf__skin_direction_all_allMHz.pdf}
\caption{The cumulative distribution function (CDF) plot shows the probability distribution of Skin SAR values for Duke. Grouped by direction.}
\label{fig:11_97}
\end{figure}

The cumulative distribution function for Skin SAR (Eartha) can be found in Figure~\ref{fig:11_98}. \\\\

\begin{figure}[H]
\centering
\includegraphics[width=\columnwidth]{../../../plots/far_field/eartha/cdf/cdf__skin_direction_all_allMHz.pdf}
\caption{The cumulative distribution function (CDF) plot shows the probability distribution of Skin SAR values for Eartha. Grouped by direction.}
\label{fig:11_98}
\end{figure}

The cumulative distribution function for Skin SAR (Ella) can be found in Figure~\ref{fig:11_99}. \\\\

\begin{figure}[H]
\centering
\includegraphics[width=\columnwidth]{../../../plots/far_field/ella/cdf/cdf__skin_direction_all_allMHz.pdf}
\caption{The cumulative distribution function (CDF) plot shows the probability distribution of Skin SAR values for Ella. Grouped by direction.}
\label{fig:11_99}
\end{figure}

The cumulative distribution function for Skin SAR (Thelonious) can be found in Figure~\ref{fig:11_100}. \\\\

\begin{figure}[H]
\centering
\includegraphics[width=\columnwidth]{../../../plots/far_field/thelonious/cdf/cdf__skin_direction_all_allMHz.pdf}
\caption{The cumulative distribution function (CDF) plot shows the probability distribution of Skin SAR values for Thelonious. Grouped by direction.}
\label{fig:11_100}
\end{figure}

\subsection{Skin Frequency Mhz All All MHz}

The cumulative distribution function for Skin SAR (Duke) can be found in Figure~\ref{fig:11_101}. \\\\

\begin{figure}[H]
\centering
\includegraphics[width=\columnwidth]{../../../plots/far_field/duke/cdf/cdf__skin_frequency_mhz_all_allMHz.pdf}
\caption{The cumulative distribution function (CDF) plot shows the probability distribution of Skin SAR values for Duke. Grouped by frequency.}
\label{fig:11_101}
\end{figure}

The cumulative distribution function for Skin SAR (Eartha) can be found in Figure~\ref{fig:11_102}. \\\\

\begin{figure}[H]
\centering
\includegraphics[width=\columnwidth]{../../../plots/far_field/eartha/cdf/cdf__skin_frequency_mhz_all_allMHz.pdf}
\caption{The cumulative distribution function (CDF) plot shows the probability distribution of Skin SAR values for Eartha. Grouped by frequency.}
\label{fig:11_102}
\end{figure}

The cumulative distribution function for Skin SAR (Ella) can be found in Figure~\ref{fig:11_103}. \\\\

\begin{figure}[H]
\centering
\includegraphics[width=\columnwidth]{../../../plots/far_field/ella/cdf/cdf__skin_frequency_mhz_all_allMHz.pdf}
\caption{The cumulative distribution function (CDF) plot shows the probability distribution of Skin SAR values for Ella. Grouped by frequency.}
\label{fig:11_103}
\end{figure}

The cumulative distribution function for Skin SAR (Thelonious) can be found in Figure~\ref{fig:11_104}. \\\\

\begin{figure}[H]
\centering
\includegraphics[width=\columnwidth]{../../../plots/far_field/thelonious/cdf/cdf__skin_frequency_mhz_all_allMHz.pdf}
\caption{The cumulative distribution function (CDF) plot shows the probability distribution of Skin SAR values for Thelonious. Grouped by frequency.}
\label{fig:11_104}
\end{figure}

\subsection{Skin Polarization All All MHz}

The cumulative distribution function for Skin SAR (Duke) can be found in Figure~\ref{fig:11_105}. \\\\

\begin{figure}[H]
\centering
\includegraphics[width=\columnwidth]{../../../plots/far_field/duke/cdf/cdf__skin_polarization_all_allMHz.pdf}
\caption{The cumulative distribution function (CDF) plot shows the probability distribution of Skin SAR values for Duke. Grouped by polarization.}
\label{fig:11_105}
\end{figure}

The cumulative distribution function for Skin SAR (Eartha) can be found in Figure~\ref{fig:11_106}. \\\\

\begin{figure}[H]
\centering
\includegraphics[width=\columnwidth]{../../../plots/far_field/eartha/cdf/cdf__skin_polarization_all_allMHz.pdf}
\caption{The cumulative distribution function (CDF) plot shows the probability distribution of Skin SAR values for Eartha. Grouped by polarization.}
\label{fig:11_106}
\end{figure}

The cumulative distribution function for Skin SAR (Ella) can be found in Figure~\ref{fig:11_107}. \\\\

\begin{figure}[H]
\centering
\includegraphics[width=\columnwidth]{../../../plots/far_field/ella/cdf/cdf__skin_polarization_all_allMHz.pdf}
\caption{The cumulative distribution function (CDF) plot shows the probability distribution of Skin SAR values for Ella. Grouped by polarization.}
\label{fig:11_107}
\end{figure}

The cumulative distribution function for Skin SAR (Thelonious) can be found in Figure~\ref{fig:11_108}. \\\\

\begin{figure}[H]
\centering
\includegraphics[width=\columnwidth]{../../../plots/far_field/thelonious/cdf/cdf__skin_polarization_all_allMHz.pdf}
\caption{The cumulative distribution function (CDF) plot shows the probability distribution of Skin SAR values for Thelonious. Grouped by polarization.}
\label{fig:11_108}
\end{figure}

\subsection{Whole Body Direction All All MHz}

The cumulative distribution function for Whole Body SAR (Duke) can be found in Figure~\ref{fig:11_109}. \\\\

\begin{figure}[H]
\centering
\includegraphics[width=\columnwidth]{../../../plots/far_field/duke/cdf/cdf__whole_body_direction_all_allMHz.pdf}
\caption{The cumulative distribution function (CDF) plot shows the probability distribution of Whole-Body SAR values for Duke. Grouped by direction.}
\label{fig:11_109}
\end{figure}

The cumulative distribution function for Whole Body SAR (Eartha) can be found in Figure~\ref{fig:11_110}. \\\\

\begin{figure}[H]
\centering
\includegraphics[width=\columnwidth]{../../../plots/far_field/eartha/cdf/cdf__whole_body_direction_all_allMHz.pdf}
\caption{The cumulative distribution function (CDF) plot shows the probability distribution of Whole-Body SAR values for Eartha. Grouped by direction.}
\label{fig:11_110}
\end{figure}

The cumulative distribution function for Whole Body SAR (Ella) can be found in Figure~\ref{fig:11_111}. \\\\

\begin{figure}[H]
\centering
\includegraphics[width=\columnwidth]{../../../plots/far_field/ella/cdf/cdf__whole_body_direction_all_allMHz.pdf}
\caption{The cumulative distribution function (CDF) plot shows the probability distribution of Whole-Body SAR values for Ella. Grouped by direction.}
\label{fig:11_111}
\end{figure}

The cumulative distribution function for Whole Body SAR (Thelonious) can be found in Figure~\ref{fig:11_112}. \\\\

\begin{figure}[H]
\centering
\includegraphics[width=\columnwidth]{../../../plots/far_field/thelonious/cdf/cdf__whole_body_direction_all_allMHz.pdf}
\caption{The cumulative distribution function (CDF) plot shows the probability distribution of Whole-Body SAR values for Thelonious. Grouped by direction.}
\label{fig:11_112}
\end{figure}

\subsection{Whole Body Frequency Mhz All All MHz}

The cumulative distribution function for Whole Body SAR (Duke) can be found in Figure~\ref{fig:11_113}. \\\\

\begin{figure}[H]
\centering
\includegraphics[width=\columnwidth]{../../../plots/far_field/duke/cdf/cdf__whole_body_frequency_mhz_all_allMHz.pdf}
\caption{The cumulative distribution function (CDF) plot shows the probability distribution of Whole-Body SAR values for Duke. Grouped by frequency.}
\label{fig:11_113}
\end{figure}

The cumulative distribution function for Whole Body SAR (Eartha) can be found in Figure~\ref{fig:11_114}. \\\\

\begin{figure}[H]
\centering
\includegraphics[width=\columnwidth]{../../../plots/far_field/eartha/cdf/cdf__whole_body_frequency_mhz_all_allMHz.pdf}
\caption{The cumulative distribution function (CDF) plot shows the probability distribution of Whole-Body SAR values for Eartha. Grouped by frequency.}
\label{fig:11_114}
\end{figure}

The cumulative distribution function for Whole Body SAR (Ella) can be found in Figure~\ref{fig:11_115}. \\\\

\begin{figure}[H]
\centering
\includegraphics[width=\columnwidth]{../../../plots/far_field/ella/cdf/cdf__whole_body_frequency_mhz_all_allMHz.pdf}
\caption{The cumulative distribution function (CDF) plot shows the probability distribution of Whole-Body SAR values for Ella. Grouped by frequency.}
\label{fig:11_115}
\end{figure}

The cumulative distribution function for Whole Body SAR (Thelonious) can be found in Figure~\ref{fig:11_116}. \\\\

\begin{figure}[H]
\centering
\includegraphics[width=\columnwidth]{../../../plots/far_field/thelonious/cdf/cdf__whole_body_frequency_mhz_all_allMHz.pdf}
\caption{The cumulative distribution function (CDF) plot shows the probability distribution of Whole-Body SAR values for Thelonious. Grouped by frequency.}
\label{fig:11_116}
\end{figure}

\subsection{Whole Body Polarization All All MHz}

The cumulative distribution function for Whole Body SAR (Duke) can be found in Figure~\ref{fig:11_117}. \\\\

\begin{figure}[H]
\centering
\includegraphics[width=\columnwidth]{../../../plots/far_field/duke/cdf/cdf__whole_body_polarization_all_allMHz.pdf}
\caption{The cumulative distribution function (CDF) plot shows the probability distribution of Whole-Body SAR values for Duke. Grouped by polarization.}
\label{fig:11_117}
\end{figure}

The cumulative distribution function for Whole Body SAR (Eartha) can be found in Figure~\ref{fig:11_118}. \\\\

\begin{figure}[H]
\centering
\includegraphics[width=\columnwidth]{../../../plots/far_field/eartha/cdf/cdf__whole_body_polarization_all_allMHz.pdf}
\caption{The cumulative distribution function (CDF) plot shows the probability distribution of Whole-Body SAR values for Eartha. Grouped by polarization.}
\label{fig:11_118}
\end{figure}

The cumulative distribution function for Whole Body SAR (Ella) can be found in Figure~\ref{fig:11_119}. \\\\

\begin{figure}[H]
\centering
\includegraphics[width=\columnwidth]{../../../plots/far_field/ella/cdf/cdf__whole_body_polarization_all_allMHz.pdf}
\caption{The cumulative distribution function (CDF) plot shows the probability distribution of Whole-Body SAR values for Ella. Grouped by polarization.}
\label{fig:11_119}
\end{figure}

The cumulative distribution function for Whole Body SAR (Thelonious) can be found in Figure~\ref{fig:11_120}. \\\\

\begin{figure}[H]
\centering
\includegraphics[width=\columnwidth]{../../../plots/far_field/thelonious/cdf/cdf__whole_body_polarization_all_allMHz.pdf}
\caption{The cumulative distribution function (CDF) plot shows the probability distribution of Whole-Body SAR values for Thelonious. Grouped by polarization.}
\label{fig:11_120}
\end{figure}

\newpage

\section{Outlier Analysis}

\subsection{Outliers Iqr All}

The outlier detection iqr method (Duke) can be found in Figure~\ref{fig:12_1}. \\\\

\begin{figure}[H]
\centering
\includegraphics[width=\columnwidth]{../../../plots/far_field/duke/outliers/outliers_iqr_all.pdf}
\caption{The outlier detection plots show metrics Genitals SAR using iqr method for the all scenarios scenario for Duke. Outliers are marked with red 'x' markers.}
\label{fig:12_1}
\end{figure}

The outlier detection iqr method (Eartha) can be found in Figure~\ref{fig:12_2}. \\\\

\begin{figure}[H]
\centering
\includegraphics[width=\columnwidth]{../../../plots/far_field/eartha/outliers/outliers_iqr_all.pdf}
\caption{The outlier detection plots show metrics Genitals SAR using iqr method for the all scenarios scenario for Eartha. Outliers are marked with red 'x' markers.}
\label{fig:12_2}
\end{figure}

The outlier detection iqr method (Ella) can be found in Figure~\ref{fig:12_3}. \\\\

\begin{figure}[H]
\centering
\includegraphics[width=\columnwidth]{../../../plots/far_field/ella/outliers/outliers_iqr_all.pdf}
\caption{The outlier detection plots show metrics Genitals SAR using iqr method for the all scenarios scenario for Ella. Outliers are marked with red 'x' markers.}
\label{fig:12_3}
\end{figure}

The outlier detection iqr method (Thelonious) can be found in Figure~\ref{fig:12_4}. \\\\

\begin{figure}[H]
\centering
\includegraphics[width=\columnwidth]{../../../plots/far_field/thelonious/outliers/outliers_iqr_all.pdf}
\caption{The outlier detection plots show metrics Genitals SAR using iqr method for the all scenarios scenario for Thelonious. Outliers are marked with red 'x' markers.}
\label{fig:12_4}
\end{figure}

\newpage

\end{document}
